% LTeX: enabled=false
%%%%%%%% 1. DOCUMENTCLASS %%%%%%%%
% Choose the language of your thesis passing 'french' or 'english' as
% \documentclass option.
% Note1: The 'page de garde' will always be written in French.
% Note2: You will have an error if you change the language of the document and
%        compile it without cleaning the auxiliary files. Compiling it again
%        should solve the problem.
\documentclass[french,a4paper,11pt,twoside]{StyleThese}


%%%%%%%% 2. PACKAGES AND BASIC INFO %%%%%%%%
% LTeX: enabled=false
\usepackage[T1]{fontenc}
\usepackage[utf8]{inputenc}
\usepackage{datetime} % For month display
\usepackage{lmodern} % Font Latin Modern Roman
\usepackage{tabularx} % Table
\usepackage{multirow} % Table multirow

\iftoggle{ThesisInEnglish}{%
  \usepackage[english]{babel}
}{ %
  \usepackage[english,main=french]{babel}
}

\usepackage{hhline}
\usepackage[left=1.5in,right=1.3in,top=1.1in,bottom=1.1in,includefoot,includehead,headheight=30pt]{geometry}
\renewcommand{\baselinestretch}{1.05}

% Math
\usepackage{amsmath,amssymb}

%%%%%%%% Table of contents %%%%%%%%
\setcounter{secnumdepth}{3}
\setcounter{tocdepth}{2}

% Table of contents for each chapter
\usepackage[nottoc, notlof, notlot]{tocbibind}
\usepackage{minitoc}
\setcounter{minitocdepth}{2}
\mtcindent=15pt

% Use \minitoc where to put a table of contents
\let\minitocORIG\minitoc
\renewcommand{\minitoc}{\minitocORIG \vspace{1.5em}}

% Glossary / list of abbreviations
\usepackage[intoc]{nomencl}
\iftoggle{ThesisInEnglish}{%
  \renewcommand{\nomname}{Glossary}
}{ %
  \renewcommand{\nomname}{Liste des abréviations}
}
\makenomenclature

% \usepackage[acronym]{glossaries}
% \makeglossaries


%%%%%%%% Image %%%%%%%%
\usepackage{ifpdf}

\ifpdf
  \usepackage[pdftex]{graphicx}
  \DeclareGraphicsExtensions{.jpg}
  \usepackage[pagebackref,hyperindex=true]{hyperref}
  \usepackage{tikz}
  \usetikzlibrary{arrows,shapes,calc}
\else
  \usepackage{graphicx}
  \DeclareGraphicsExtensions{.ps,.eps}
  \usepackage[dvipdfm,pagebackref,hyperindex=true]{hyperref}
\fi

\usepackage{rotating} % Sideways of figures & tables
\usepackage{tablefootnote}


%%%%%%%% PDF %%%%%%%%
% Links in pdf
\usepackage{color}
\definecolor{linkcol}{rgb}{0,0,0.4}
\definecolor{citecol}{rgb}{0.5,0,0}
\definecolor{linkcol}{rgb}{0,0,0}
\definecolor{citecol}{rgb}{0,0,0}
% Change this to change the informations included in the pdf file

% Basic pdf setup
\hypersetup
{
  bookmarksopen=true,
  %pdftoolbar=false, %barre d'outils non visible
  pdfmenubar=true, %barre de menu visible
  pdfhighlight=/O, %effet d'un clic sur un lien hypertexte
  colorlinks=true, %couleurs sur les liens hypertextes
  pdfpagemode=UseNone, %aucun mode de page
  pdfpagelayout=SinglePage, %ouverture en simple page
  pdffitwindow=true, %pages ouvertes entierement dans toute la fenetre
  linkcolor=linkcol, %couleur des liens hypertextes internes
  citecolor=citecol, %couleur des liens pour les citations
  urlcolor=linkcol %couleur des liens pour les url
}


%%%%%%%% Backref in biblio %%%%%%%%
%% Nicer backref links
\iftoggle{ThesisInEnglish}{%
  \renewcommand*{\backref}[1]{}
  \renewcommand*{\backrefalt}[4]{%
  \ifcase #1 %
  (Not cited.)%
  \or
  (Cited in page~#2.)%
  \else
  (Cited in pages~#2.)%
  \fi}
  \renewcommand*{\backrefsep}{, }
  \renewcommand*{\backreftwosep}{ and~}
  \renewcommand*{\backreflastsep}{ and~}
}{%
  \renewcommand*{\backref}[1]{}
  \renewcommand*{\backrefalt}[4]{%
  \ifcase #1 %
  (Non cité.)%
  \or
  (Cité en page~#2.)%
  \else
  (Cité en pages~#2.)%
  \fi}
  \renewcommand*{\backrefsep}{, }
  \renewcommand*{\backreftwosep}{ et~}
  \renewcommand*{\backreflastsep}{ et~}
}

\usepackage{xurl} % allow break url

%%%%%%%% Fancy Header %%%%%%%%
% Fancy Header Style Options
\usepackage{fancyhdr}                   % Fancy Header and Footer
\pagestyle{fancy}                       % Sets fancy header and footer
\fancyfoot{}                            % Delete current footer settings

\fancyhead[LE,RO]{\bfseries\thepage}    % Page number (boldface) in left on even
                                        % pages and right on odd pages
\fancyhead[RE]{\bfseries\nouppercase{\leftmark}}      % Chapter in the right on even pages
\fancyhead[LO]{\bfseries\nouppercase{\rightmark}}     % Section in the left on odd pages

\let\headruleORIG\headrule
\renewcommand{\headrule}{\color{black} \headruleORIG}
\renewcommand{\headrulewidth}{1.0pt}
\usepackage{colortbl}
\arrayrulecolor{black}

\fancypagestyle{plain}{
  \fancyhead{}
  \fancyfoot{}
  \renewcommand{\headrulewidth}{0pt}
}


%%%%%%%% Clear Header %%%%%%%%
% Clear Header Style on the Last Empty Odd pages
\makeatletter
\def\cleardoublepage{\clearpage\if@twoside \ifodd\c@page\else%
  \hbox{}%
  \thispagestyle{empty}%              % Empty header styles
  \newpage%
  \if@twocolumn\hbox{}\newpage\fi\fi\fi}
\makeatother


%%%%%%%% Center Page %%%%%%%%
% centered page environment (for abstract)
\newenvironment{vcenterpage}
{\newpage\vspace*{\fill}\thispagestyle{empty}\renewcommand{\headrulewidth}{0pt}}
{\vspace*{\fill}}


%%%%%%%% End Common Format %%%%%%%%


% Loading the tlsflyleaf.sty package require some option to define the
% establishment name, the doctoral school and the PhD speciality.
% In that aim you have 2 key-value option:
%   - Ets=<value> : define the establishment name
%   - ED=<value>  : define the doctoral school and speciality
%   - ED2=<value> : define the second speciality ("double mention"). OPTIONAL.
% The full list of accepted values for each option could be find either
% in the documentation or in tlsflyleaf.sty file.
%\usepackage[ED=MITT-STICRT, Ets=INSA]{tlsflyleaf}
%\usepackage[ED=SDU2E-Ast, ED2=SDU2E-Eco, Ets=UT3]{tlsflyleaf}
\usepackage[ED=EDSYS-A, Ets=ENAC]{tlsflyleaf}

\usepackage{siunitx}
\usepackage{empheq}
\usepackage{bbold}
\usepackage{array,multirow,makecell}
\usepackage{float}
\usepackage{todonotes}
% \usepackage{algorithm,algpseudocode,algorithmicx}
\usepackage{algcompatible}
\usepackage{algorithm}
\usepackage{amsthm}
\usepackage{etoolbox}
\usepackage{subcaption}



\newcommand{\smallmat}[1]{\left[ \begin{smallmatrix}#1 \end{smallmatrix} \right]}
\newcommand{\bigmat}[1]{\begin{bmatrix}#1 \end{bmatrix}}

\newcommand{\smallm}[1]{\begin{smallmatrix}#1 \end{smallmatrix}}
\newcommand\real{{\mathbb R}}
\newcommand{\skewsym}[1]{\left[#1\right]_{\times}}
\DeclareMathOperator{\rank}{rank}
\DeclareMathOperator{\diag}{diag}
\DeclareMathSymbol{\shortminus}{\mathbin}{AMSa}{"39}

\newtheorem{proposition}{Proposition}
\newtheorem{theorem}{Théorème}
\newtheorem{remm}{Remarque}
\renewcommand{\listalgorithmname}{Liste des algorithmes}
\floatname{algorithm}{Algorithme}
\newenvironment{remark}{\begin{remm}\rm }{\hfill \hspace*{1pt} \hfill $\circ$\end{remm}}
\newcommand\numberthis{\addtocounter{equation}{1}\tag{\theequation}}
\def\NoNumber#1{{\def\alglinenumber##1{}\State #1}\addtocounter{ALG@line}{-1}}






% Setup basic string
\title{Control actif de la turbulence sur un micro drone convertible}
\author{Florian SANSOU}
\defencedate{jj/mm/aaaa}
\lab{École Nationale d’Aviation Civile}
%\cotutelle{}

% Setup custom pdf info
\makeatletter
\hypersetup {
  pdftitle={\@title},
  pdfauthor={\@author},
  pdfsubject={Thesis subject},
  pdfkeywords={key, words},
}
\makeatother

% Setup people like your boss, the jury team and the referees
% - First you need to define how number they will be in each category
%   It is done with the commands \nboss{n}, \nreferee{n} and \njudge{n}.
%   You can define more people in each category than the number given
%   but only the first "\npeople" will be print.
% - Then use the command \makesomeone{<category>}{<number>}{<name>}{<status>}{<other>}
%   where:
%     <category> should be select in ['boss', 'referee', 'judge']
%     <number>   is the rank for printing the person.
%                Only number <= "\npeople" will be printed
%     <name>     First name and las name of the people
%     <status>   Is (s)he a "charg\'e de recher" ou un "professeur d'universit\'e"...
%     <other>    What ever string you want to add (laboratory, jury member place...).
%% Boss
\nboss{2}
\makesomeone{boss}{2}{M. Gautier HATTENBERGER}{co-directeur de th\`ese}{}  % Sera affiche en second
\makesomeone{boss}{1}{M. Fabrice DEMOURANT}{Directeur de th\`ese}{} % Sera afiche en premier
%% Referee
\nreferee{2}
\makesomeone{referee}{1}{M. Paolo ROBUFFO GIORDANO}{}{}
\makesomeone{referee}{2}{M. Pascal MORIN}{}{}
%% Judges
\njudge{6}
\makesomeone{judge}{1}{Mme Sophie TARBOURIECH}{Directrice de recherche LAAS-CNRS}{Pr\'esidente du jury}
\makesomeone{judge}{2}{M. Paolo ROBUFFO GIORDANO}{Directeur de recherche IRISA-CNRS}{Rapporteur}
\makesomeone{judge}{3}{M. Pascal MORIN}{Professeur des universit\'es  Sorbonne Université}{Rapporteur}
\makesomeone{judge}{4}{Philippe CHEVREL}{Enseignant chercheur IMT ATLANTIQUE}{Examinateur}
\makesomeone{judge}{5}{M. Fabrice DEMOURANT}{Ing\'enieur de recherche ONERA}{Directeur de th\`ese}
\makesomeone{judge}{6}{M. Gautier HATTENBERGER}{Enseignant chercheur ENAC}{Co-directeur de th\`ese}

% Other package here
% ...

\sloppy
\begin{document}


%%%%%%%% 3. COVER PAGE %%%%%%%%

\makeatletter
\pdfbookmark{\@title}{title}
\makeatother

% \makeflyleaf
% \includepdf[pages=-]{chapters/couverture_these.pdf} % if you want to use generated pdf cover (e.g. ADUM), use this instead of \makeflyleaf. You'll also need \usepackage{pdfpages}
\cleardoublepage
\onehalfspacing
\dominitoc
\doparttoc


%%%%%%%% 4. ACKNOWLEDGMENTS AND TABLES OF CONTENT %%%%%%%%
\pagenumbering{roman}
% Here you can see an example of how to create text conditioned by the language
% variable. The \iftoggle command:
%
%   \iftoggle{ThesisInEnglish}{%
%   <your-text-in-english>
%   }{%
%   <your-text-in-french>
%   }
%
% will compile only one of the two blocks, depending on the variable you set at
% the beginning of this document. Language selection is managed this way in the
% formatAndDefs.tex file. You too can create sections of your thesis that is
% language dependend this way, although you probably won't need it. Another use
% of \iftoggle can be found at the end of this file.
\iftoggle{ThesisInEnglish}{%
\section*{Acknowledgments}
}{%
\section*{Remerciements}
}

A faire en dernier :-) 


\cleardoublepage
\pdfbookmark{\contentsname}{toc}
\tableofcontents


\renewcommand{\listfigurename}{Liste des figures}
\listoffigures
\addcontentsline{toc}{chapter}{\listfigurename}


\listoftables
\addcontentsline{toc}{chapter}{\listtablename}
\mtcaddchapter

\listofalgorithms
\addcontentsline{toc}{chapter}{\listalgorithmname}
\mtcaddchapter

\printnomenclature
\mtcaddchapter
% Use \mtcfixnomenclature below if you have a glossary (added with
% \printnomenclature above) and you're see a shift in the mini-table of
% contents at the begining of each chapter (example: no mini-toc in chapter 1;
% mini-toc of chapter 1 appearing in chapter 2; and so on).
%
% You should not use \mtcfixnomenclature if you have no glossary (that means,
% if you don't use \printnomenclature or if your glossary is empty).
\mtcfixnomenclature

% \printglossary
% \printglossary[type=\acronymtype]
% \mtcaddchapter



%%%%%%%% 5. MAIN CONTENT %%%%%%%%
\mainmatter

\chapter{Généralités sur les drones}
% \addstarredchapter{Introduction} %Sinon cela n'apparait pas dans la table des matières
% \markboth{Introduction}{Introduction} % headers

% \chapter{}
% \minitoc
% \newacronym{gcd}{GCD}{Greatest Common Divisor}

\section{Drone autonome}
Ces dernières années, le domaine des drones s'est considérablement développé, de nombreux progrès ont été réalisé dans la conduite de vol autonome qui permette de réaliser de nombreuse tache longue, répétitive ou dangereuse de manière plus sûre que des avions ou des systèmes télépiloté. Les drones ont fait leurs preuves dans de nombreuses applications civiles, alors qu'ils étaient auparavant conçus à des fins de surveillance et de destruction dans le secteur militaire. La possibilité d'utiliser des systèmes de vol autonomes dans le secteur civil a commencé avec l'accessibilité croissante proposée par l'industrie commerciale grâce à des solutions à faible coût pour les applications d'imagerie aérienne. L'intérêt porté aux applications d'imagerie aérienne a motivé le développement de plusieurs projets, notamment dans les domaines de l'aide humanitaire, des secours en cas de catastrophe, de la recherche et du sauvetage, des opérations de sécurité, de la surveillance, de l'agriculture de précision et de l'inspection des infrastructures civiles.

\section{Micros drone convertible}
Les micros drone sont une gamme de drone intéressante par leurs petites tailles qui leur permettent d'intervenir dans des espaces confiné ou contraint. Ils n'ont cependant qu'une charge utile restreinte souvent limitée à l'emport d'une caméra ou d'un colis de faible masse. 
    \subsection{Domaine de vol} 
    Tout l'intérêt d'un drone convertible réside dans la capacité à décoller et atterrir verticalement, tout en conservant une bonne efficacité énergétique, en vol d'avancement, conféré par une aile. Cette aile a l'avantage de généré de la portance, qui s'oppose au poids un drone et permet d'assurer sa sustentation. La contrepartie de la génération de la portance est le trainé qui s'oppose à l'avancement et doit être contré par une force de traction générée par les hélices.
    Nous pouvons définir l'efficacité énergétique comme le ratio entre le temps de vol et l'énergie électrique nécessaire pour effectuer ce vol. Nous pouvons comparer l'efficacité d'un drone quadrirotor qui assure sa sustentation uniquement grâce à des hélices (à l'instar d'un hélicoptère), a celle d'un drone à voilure fixe et à celle d'un convertible.
    \begin{table}[ht]
        \centering
        \begin{tabular}{|c|c|c|c|}
            \hline
            Type d'architecture & Vitesse & Stationnaire & Temps de vol \\
            \hline \hline
            Drone à voilure fixe & & & \\
            \hline
            Quadrirotor & & & \\
            \hline
            Drone convertible & & & \\
            
        \end{tabular}
    \end{table}
    Drone à voilure fixe, stationnaire impossible, grande efficité ernegetique vitesse mini et max
    Quadrirotor, grande manouvrabilité, faible efficaité energetique
    Convertible, dommaine de vol important, sensibilité aux pertubation, grand efficaticté energetique en vol d'avancement, faible en stationnaire 

    Nous observons qu'un drone convertible possède un domaine de vol bien plus important qu'un drone à voilure fixe qui ne va pas pouvoir voler à basse vitesse et qu'un quadrirotor dont l'autonomie vas être limité par sa consommation. Ainsi un drone convertible semble un outil tout à fait approprié pour de nombreuse mission.

    \subsection{Types d'architecture des drones convertibles}
    La conception structurelle et aérodynamique d'un drone est le facteur principal permettant des transitions stable et fluide. De plus, il est nécessaire d'optimiser l'architecture pour une mission de manière à être le plus efficace dans la tache principale du drone. Au vu de la diversité des missions, un grand nombre d'architectures ont été proposé et que nous pouvons catégoriser en quatre classes : \textit{quadplanes}, \textit{tiltrotor}, \textit{tailsitter}, \textit{tiltwing}. Nous ajoutons la catégorie \textit{quadplanes} aux trois autres catégories (\textit{tiltrotor}, \textit{tailsitter} et \textit{tiltwing}) classiquement utilisé lors des études bibliographiques sur les drones convertibles \cite{saeed_survey_2018,ducard_review_2021}.
        \subsubsection*{\textit{Quadplanes}}
        Les \textit{quadplanes} sont conçu par la fusion d'un avion et d'un quadrirotor, ce qui permet un découpage de l'actionnement en fonction de la phase de vol. Le premier système de propulsion est composé de quatre hélices générant une force verticale permet le contrôle lors de phase de décollage, d'atterrissage et de stationnaire. Le second système de propulsion est composé d'une  propulsive supplémentaire afin d'atteindre des vitesses en vol d'avancement.

        L'avantage de ce type d'architecture est sa grande robustesse. Effectivement, aucune pièce en mouvement est nécessaire pour réaliser la transition ce que réduit le risque de défaillance mécanique. L'inconvénient est le manque d'efficacité. Lors d'un vol d'avancement, la portance sera générée par l'aile, ainsi il est possible de désactiver les rotors qui génèrerons des perturbations aérodynamiques et des trainés parasites. Effectivement, les axes des moteurs se retrouvent orthogonaux au flux d'air généré par le déplacement du drone, ce qui correspond au cas le plus défavorable en termes de trainé. De plus, la surcharge engendrée par l'emport de moteur supplémentaire se traduit par une diminution de la charge utile transportable. 

        En termes de contrôle, un atout indéniable est la séparation des actionnements en fonction de phase de vol. Ainsi, l'architecture de commande sera basée sur un mécanisme de commutation permettant de choisir la loi de commande appropriée sur un critère de vitesse air. Ce critère est pertinent, car il est lié à l'efficacité de l'aile à générer de la portance induite par le flux d'air. Ainsi à basse vitesse, le drone se stabilise avec l'actionnement quadrirotor et la loi de commande associé et dans les vitesses plus importante, la commutation de loi permet de contrôler le drone en mode avion. Toutefois, le passage d'une loi à l'autre reste le point clé de la commande et demeure complexe et critique.

        \subsubsection*{\textit{Tiltrotor}}

        Les \textit{tiltrotor} nécessitent l'utilisation d'un actionneur supplémentaire afin d'effectuer la transition. Les rotors sont montés sur des arbres basculants actionnés et la transition du vol stationnaire au vol d'avancement (ou inversement) s'effectue progressivement en fonction de l'inclinaison du rotor. Ainsi l'angle entre le souffle des hélices et l'aile peut être ajusté à chaque instant. Cet angle joue un rôle important dans le contrôle des forces et des moments aérodynamiques : sa maîtrise permet de mieux gérer non seulement les performances aérodynamiques du vol lors des transitions, mais aussi la stabilité du système sur l'ensemble du domaine de vol. 
        Malgré le fait que les \textit{tiltrotor} embarqué un actionneur uniquement dédié à la transition, ce qui augmente la masse du drone, cette architecture est intéressante, car elle permet d'utiliser les mêmes actionneurs pour assurée la sustentation en stationnaire que pour générer la traction en mode avion

        \subsubsection*{\textit{Tailsitter}}
        Contrairement au \textit{tiltrotor} qui se pose sur le fuselage de l'avion, les \textit{tailsitter} se posent à la verticale. Durant la transition du mode stationnaire au vol d'avancement la structure entière bascule vers l'avant modifiant l'angle d'incidence de la voiture. Selon la configuration du \textit{tailsitter}, la transition peut être réalisée soit par la génération du moment aérodynamique créé par les élevons, soit par le couple créé par le système de propulsion. Pendant le vol d'avancement, en position horizontale, le \textit{tailsitter} vole comme un avion conventionnel sans dérive. En utilisant des techniques aérodynamiques classiques, les concepteurs peuvent optimiser le profil de l'aile du \textit{tailsitter} pour le rendre plus endurant afin de réduire la consommation d'énergie. Grâce à ce processus d'optimisation aérodynamique, le \textit{tailsitter} peut effectuer des missions de vol de plus d'une heure.
        
        Ils semblent être la configuration le plus abouti des drones convertibles, car il utilise les mêmes actionneurs dans tout le domaine de vols. Ainsi, il n'embarque aucune masse superflue.

    
        \subsubsection*{\textit{Tiltwing}} 
        \todo{citation}
        La particularité des \textit{tiltwings} réside dans le fait que les rotors sont inclinées en même temps que les ailes. Un actionneur supplémentaire et puissant est donc nécessaire pour surmonter le couple de l'aile afin de la positionner dans l'orientation souhaitée. La commande de cet actionneur doit être prise en compte lors de la conception des lois de commande. Pendant le décollage, l'atterrissage et les vols stationnaires, les ailes doivent être positionnées vers le haut afin de produire une force de poussée opposée au vecteur gravité. Dans ces phases de vol, lorsque les ailes sont orientées vers le haut, l'aéronef est plus vulnérable aux vents et les lois de commande doivent rejeter ces perturbations. Dans la littérature, il existe plusieurs configurations d'ailes basculantes et différentes approches de contrôle conçues pour stabiliser leur dynamique de vol 
        % \cite{}.

        \paragraph*{\textit{Freewing}}
        Une gamme en cours de développement à l'intérieur de l'architecture des \textit{tiltwing} est les \textit{freewing}. Ils sont actionnés comme des \textit{tiltwing} sauf au niveau de l'axe de rotation entre l'aile et le fuselage. Cette rotation est laissée libre, ce degrés de liberté permet à l'aile de changer librement son incidence.

\section{Propriétés des \textit{tailsitters} et des \textit{freewings}}
    D'un point de vue mécanique, les \textit{tailsitters} et les \textit{freewings} sont caractérisés comme des systèmes sous-actionnés avec une dynamique fortement couplée. Ces caractéristiques mécaniques rendent le processus de modélisation et d'identification difficile. Cela peut s'expliquer par le fait que, pour ce type de système, une entrée de commande donnée agit simultanément sur différentes dynamiques. Ainsi, l'identification de l'influence d'une entrée de commande donnée sur une dynamique particulière reste un processus important qui nécessite plus d'attention.
    \subsection{Actionnement}
    Dans ces deux architectures, il est courant de trouver des actionneurs basés sur des effets aérodynamiques. Ces actionneurs ont l'avantage d'être peut consommateur en énergies, ils ont mue par des servomoteurs qui consomme peu d'électricité proportionnellement aux couples qu'ils génèrent. Dans le cas des ailes volantes, les surfaces aérodynamiques sont souvent placé sur la partie arrière des ailes et peuvent être utilisé symétriquement similairement à des volets ou anti-symétriquement comme des ailerons. Nous utiliserons donc la contraction des deux mots anglais pour définir ces surfaces aérodynamiques qui porte le nom d'élevon.
    Dans les phases de stationnaire, atterrissage ou décollage, la plateforme est maintenue en vol par les hélices, ainsi il est nécessaire de dimensionner les groupes moteurs-hélices pour qui puisse générer assez de force. En fonction des configurations, les moments peuvent être obtenus par des différentiels sur l'utilisation des moteurs ou bien par des surfaces aérodynamiques. Dans le cas de surface soufflé par le flux d'air des hélices, il existe un couplage des actionneurs qui complexifie la modélisation et le contrôle de ces architectures.
    \subsection{Aérodynamique}

\section{Modélisations}
\todo{Compromis précision du modèle et commande}
\todo{Modèle de turbulence dryden}


\section{Commandes}

\section{Contexte de la thèse}
\section{Présentation de la thèse}




\chapter{Darko UAV modelling}
% \minitoc

\section{Model of the DarkO UAV}
\label{sec:model}
The DarkO UAV, designed and developed at the Ecole Nationale de l'Aviation Civile (ENAC) in Toulouse
 (France), is a clear example of a convertible UAV with a so-called tail-sitter architecture. 
 DarkO is assembled from multiple 3D printed Onyx parts (a highly robust material comprising 
 omnidirectional carbon fibres). All surfaces are interlocked on a single axis, so that the drone 
 can be easily disassembled for parts replacement or to gain access to the on-board electronics. 

The on-board autopilot is an Apogee~\footnote{\url{https://wiki.paparazziuav.org/wiki/Apogee/v1.00}} 
board manufactured at ENAC, see Fig. \ref{fig:apogee}. 
\begin{figure}[ht]
\centering
    \includegraphics[width=0.8\columnwidth]{figures/800px-Apogee_v100_top_1E.jpeg}
    \caption{Apogee v1.00 top view.}
    \label{fig:apogee}
\end{figure}
The autopilot provides the option of recording the on-board data on an SD memory card, at the control
 frequency of 500 Hz, thus allowing for effective post-processing of acquired data. 
 The communication protocol used between the autopilot and the Electronic Speed Controllers (ESCs)
  is Dshot 600. The ESCs are AIKON AK32 35A flash with an AM32 firmware. The ground-to-board
   communication is performed via a bidirectional channel based on XBee-PRO S1 modules.
\begin{figure}[ht]
\centering
    \includegraphics[width=1\columnwidth]{figures/darko.pdf}
    \caption{DarkO body frame with a schematic representation of the actuators.}
    \label{fig:darko2}
\end{figure}
DarkO's actuators consist in two propellers (T-Motor T5147) symmetrically placed at the front of the wing (shown in \textbf{black} in Fig. \ref{fig:darko2}) powered by two T-Motor F30 2300kv electric motors and two elevons, placed at the back of the wing (shown in \textcolor{cyan}{blue} in Fig. \ref{fig:darko2}) and acting as control surfaces. The elevons are driven by two MKS DS65K servomotors. Fig.~\ref{fig:darko2} shows the DarkO model, together with a NED inertial reference frame (or world frame) ``$\text{i}$'' linked to the Earth's surface,
and a body reference frame ``$\text{b}$'' attached to the drone, with $x_{\text{b}}$ corresponding to the roll axis (the propellers axe laying in the $z_{\text{b}} =0$ plane), $y_{\text{b}}$ the pitch axis (the direction of the wings), $z_{\text{b}}$ the yaw axis. Using the same notation as in \cite{lustosa:hal-03035938}, the left and right propeller/elevon are denoted by using subscripts $i=1$ (left) and $i=2$ (right). The sign convention will be defined positive for the elevons positions $\delta_{1}$, $\delta_{2}$ when they create a pitch-up moment with the propellers rotating in a opposite directions with angular speeds $\omega_{1} > 0 $ and $\omega_{2} < 0$, respectively.
\begin{table}[ht]
  \centering
    \begin{tabular}{|l|c|c|}
      \hline
      \multicolumn{1}{|c|}{Parameter or coefficient} & Value & Units  \\
      \hline
      $m$ (drone mass)  & 0.519 & \SI{}{\kilogram} \\
      \hline
      $b$ (wingspan)  & 0.542 & \SI{}{\meter} \\
      \hline
      $c$ (aerodynamic cord)  & 0.13 & \SI{}{\meter} \\
      \hline
      $\boldsymbol{B}=\diag(b,c,b)$ & $\!\!\diag(0.542, 0.13, 0.542)$ \!\! & \SI{}{\meter}\\
      \hline
      $S$ (wing area) & 0.026936 & \SI{}{\square\meter}\\
      \hline
      $S_{\text{wet}}$ (wet area) & 0.0180 & \SI{}{\square\meter}\\
      \hline
      $S_{\text{p}}$ (propeller area) & 0.0127 & \SI{}{\square\meter}\\
      \hline
      % $J_{x}$ & 0.0070 & \SI{}{\kilogram\square\meter}\\
      % \hline
      $\boldsymbol{J}=\diag(J_{x},J_{y},J_{z})$ & \!\! $\diag(0.0072,0.0004,0.0086)$\!\! & \SI{}{\kilogram\square\meter}\\
      % \hline
      % $J_{z}$ & 0.0061 & \SI{}{\kilogram\square\meter}\\
    %   \hline
    %   $J_{p}$ & 5.1116e-6 & \SI{}{\kilogram\square\meter}\\
      \hline
      $k_{\text{f}}$ (propeller thrust) & 1.7800e-8 & \SI{}{\kilogram\meter}\\
      \hline
      $k_{\text{m}}$(propeller torque) & 2.1065e-10 & \SI{}{\kilogram\square\meter}\\
      \hline
      $p_{x}$ (propeller $x$ location) & 0.065 & \SI{}{\meter}\\
      \hline
      $p_{y}$ (propeller $y$ location) & 0.162 & \SI{}{\meter}\\
      \hline
      $a_{y}$ (lift $y$ position) & 0.1504 & \SI{}{\meter}\\
      \hline
      $\xi_{\text{f}}$ (elevons lift) & 0.2 & --\\
      \hline
      $\xi_{\text{m}}$ (elevons torque) & 1.4 & --\\
      \hline
      $\rho$ (air density) & 1.225 & \SI{}{\kilogram\per\cubic\meter}\\
      \hline
      $C_{\text{d}}$ (drag coefficient) & 0.1644 & --\\
      \hline
      $C_{y}$ (lateral coefficient) & 0 & --\\
      \hline
       $C_{\ell}$ (lift coefficient) & 5.4001 & --\\
      \hline
      $\Delta_{\text{r}}$ (UAV centering) & -0.0145 & \SI{}{\meter}\\
      \hline
    \end{tabular}
    \caption{\label{tab:pars} Identified numerical parameters of the DarkO model.}
\end{table}

\subsection{Full nonlinear model}

Exploiting the modelling reported in \cite{lustosa:hal-03035938} and \cite{olszaneckibarth:hal-02542982}, an accurate model of the DarkO dynamics describes the position $\boldsymbol{p} \in \real^3$ of the origin of the body frame  and its velocity $\boldsymbol{v} = \dot{\boldsymbol{p}} \in \real^3$, in addition to its orientation, well represented by a quaternion $\boldsymbol{q} \in {\mathbb S}^3:=\{ \boldsymbol{q}\in \real^4: |\boldsymbol{q}| = 1\}$ and its angular velocity $\boldsymbol{\omega}_{\text{b}}$ represented in the body frame, which satisfies
$\boldsymbol{\dot q} = \frac{1}{2}\boldsymbol{q} \otimes \smallmat{0 \\ \boldsymbol{\omega}_{\text{b}}}$, where $\otimes$ denotes the quaternion product (see \cite{lustosa:hal-03035938,olszaneckibarth:hal-02542982} or the tutorial \cite{hamel_minhduc} for the details).
Selecting the overall state as $\boldsymbol{x}:=(\boldsymbol{p},~ \boldsymbol{v},~ \boldsymbol{q},~  \boldsymbol{\omega}_{\text{b}})$, the mathematical model derived in \cite{lustosa:hal-03035938}, depend on a set of parameters listed in Table \ref{tab:pars}, where we also report the value obtained from a system identification \cite{sansou:stage}. The dynamic model can be written as
\begin{subequations}\label{eq:dyna_orig}
  \begin{empheq}[left=\empheqlbrace]{alignat=2}
        % \boldsymbol{\dot p} & &= & \boldsymbol{v} \label{eq:dyna1}\\
        m\boldsymbol{\dot v} &&=& - m\boldsymbol{g} +  \boldsymbol{R}(\boldsymbol{q})\boldsymbol{F}_{\text{b}},\\
        \label{eq:dyna_orig_b}
        \boldsymbol{J} \boldsymbol{\dot \omega_{\text{b}}} &&= &  - \skewsym{\boldsymbol{\omega}_{\text{b}}}\boldsymbol{J}\boldsymbol{\omega}_{\text{b}} + \boldsymbol{M}_{\text{b}},
  \end{empheq}
\end{subequations}
where $\boldsymbol{g}:=\smallmat{0 & 0& 9.81}^\top$ denotes the gravity vector, $m\in \real$ is the mass, $\boldsymbol{J}\in \real^{3\times 3}$ is the diagonal moment of inertia (see Table~\ref{tab:pars}) and, partitioning the quaternion $\boldsymbol{q} \in {\mathbb S}^3$ as $\boldsymbol{q} := \left [ \eta ~ \boldsymbol{\epsilon}^\top \right]^\top$, the corresponding rotation matrix $\boldsymbol{R}(\boldsymbol{q}) \in SO(3): = \{\boldsymbol{R}\in \real^{3\times 3}: \; \boldsymbol{R}^\top \boldsymbol{R} = \mathbb{I}_{3}, \det (\boldsymbol{R})=1\}$ is defined as (see \cite{hamel_minhduc})
\begin{align}
    \label{eq:matrix_rot}
    \boldsymbol{R}(\boldsymbol{q}) := \mathbb{I}_{3} +2\eta \skewsym{\boldsymbol{\epsilon}} + 2\skewsym{\boldsymbol{\epsilon}}^{2}.
\end{align}
According to \cite{lustosa:hal-03035938} the force and moment vectors $\boldsymbol{F}_{\text{b}}$ and $\boldsymbol{M}_{\text{b}}$ in \eqref{eq:dyna_orig} depend on (i) the state $\boldsymbol{x}$, (ii) the disturbance $\boldsymbol{w} \in \real^3$, representing the wind speed in the world frame, and (iii) the actuators commands (see Figure~\ref{fig:darko2}), comprising the two propellers' rotational speeds
$\omega_1, \omega_2 \in \real$ and the two elevons' deflections $\delta_1, \delta_2\in \real$.
Let us first consider the actuators commands' effect. Each propeller generates a thrust $\boldsymbol{T}_i$ oriented in the $x$ direction of the body frame and a moment $\boldsymbol{N}_i$ about the same axis:
\begin{align}
\label{eq:thrust}
\boldsymbol{T}_{i} \!:=\! \begin{bmatrix} \tau_{i} \\ 0 \\ 0 \end{bmatrix} \!:=\!
\begin{bmatrix} k_{\text{f}}\omega_{i}^{2} \\ 0 \\ 0 \end{bmatrix}\! , \;
\boldsymbol{N}_{i} \!:=\! (-1)^{i}  \frac{k_{\text{m}} }{k_{\text{f}}}\boldsymbol{T}_{i}, \quad i=1,2 .
\end{align} 
Each elevon's position $\delta_i \in \real$ is assigned by a servo-motor that imposes an efficiency level (in terms of airstream deflection) quantified by two skew-symmetric matrices:
\begin{align}
\label{eq:elevons_efficiency}
    \boldsymbol{\Delta}^{\text{f}}_{i} \!:=\! \begin{bmatrix} 0 & 0 & \xi_{\text{f}}\delta_{i} \\ 0 & 0 & 0 \\ -\xi_{\text{f}}\delta_{i} & 0 & 0 \end{bmatrix}\! ,\;
    \boldsymbol{\Delta}^{\text{m}}_{i} \!:=\! \begin{bmatrix} 0 & 0 & \xi_{\text{m}}\delta_{i} \\ 0 & 0 & 0 \\ -\xi_{\text{m}}\delta_{i} & 0 & 0 \end{bmatrix} \!, 
\end{align}
$i=1,2$. The constant parameters $k_{\text{f}}$, $k_{\text{m}}$, $\xi_{\text{f}}$, $\xi_{\text{m}}$ appearing in \eqref{eq:thrust} and \eqref{eq:elevons_efficiency} are listed in Table~\ref{tab:pars}.\\
With the above actuation quantities, we may rearrange the dynamics given in \cite[eqns (97),~(98)]{lustosa:hal-03035938} (see also \cite{sansou:stage}) and express $\boldsymbol{F}_{\text{b}}$ and $\boldsymbol{M}_{\text{b}}$ in \eqref{eq:dyna_orig} as
%
\begin{align}
%\begin{split}
\nonumber
    \boldsymbol{F}_{\text{b}} :={}&  \boldsymbol{T}_{1} + \boldsymbol{T}_{2} + \frac{S_{\text{wet}}}{4S_{\text{p}}} \boldsymbol{\Phi}^{\text{(fv)}} \Big( (\boldsymbol{\Delta}^{\text{f}}_1 - \mathbb{I}_{3} ) \boldsymbol{T}_{1} + ( \boldsymbol{\Delta}^{\text{f}}_2 - \mathbb{I}_{3}) \boldsymbol{T}_{2}\Big) \\ 
     \label{eq:Fb}
    &+ \frac{1}{4} \rho S  \boldsymbol{\Phi}^{\text{(fv)}} \Big(\boldsymbol{\Delta}^{\text{f}}_1+ \boldsymbol{\Delta}^{\text{f}}_2 - 2 \mathbb{I}_{3} \Big) \lVert \boldsymbol{v_{\text{b}}} \rVert \boldsymbol{v_{\text{b}}}\\
    \nonumber
    &+ \frac{1}{4} \rho S \boldsymbol{\Phi}^{\text{(mv)}} \Big(\boldsymbol{\Delta}^{\text{f}}_1 + \boldsymbol{\Delta}^{\text{f}}_2 - 2\mathbb{I}_{3}\Big) \boldsymbol{B} \lVert \boldsymbol{v_{\text{b}}} \rVert  \boldsymbol{\omega}_{\text{b}}, 
%\end{split}
\end{align}
% 
\begin{align} 
\label{eq:Mb}
% \begin{split}
 \boldsymbol{M}_{\text{b}} :&=\boldsymbol{N}_{1} + \boldsymbol{N}_{2} + \skewsym{\smallm{p_x\\ p_y\\ 0}} \boldsymbol{T}_{1} + \skewsym{\smallm{p_x\\ - p_y\\ 0}} \boldsymbol{T}_{2}\\
    \nonumber
  &- \frac{S_{\text{wet}}}{4S_{\text{p}}} \bigg( \boldsymbol{B} \boldsymbol{\Phi}^{\text{(mv)}} (\boldsymbol{\Delta}^{\text{m}}_1- \mathbb{I}_{3} ) + \skewsym{\smallm{0 \\ a_y \\ 0}} \boldsymbol{\Phi}^{\text{(fv)}} (\boldsymbol{\Delta}^{\text{m}}_1 +\mathbb{I}_{3} ) \bigg) \boldsymbol{T}_{1} \\
    \nonumber
  & - \frac{S_{\text{wet}}}{4S_{\text{p}}} \bigg( \boldsymbol{B} \boldsymbol{\Phi}^{\text{(mv)}} (\boldsymbol{\Delta}^{\text{m}}_2 - \mathbb{I}_{3} ) +  \skewsym{\smallm{0 \\ - a_y \\ 0}} \boldsymbol{\Phi}^{\text{(fv)}} (\boldsymbol{\Delta}^{\text{m}}_2 + \mathbb{I}_{3}) \bigg) \boldsymbol{T}_{2} \\
    \nonumber
  & + \frac{1}{4} \rho S  \bigg( \Big(\skewsym{\smallm{0 \\ a_y \\ 0}} \!\!\! \boldsymbol{\Phi}^{\text{(fv)}}  + \boldsymbol{B} \boldsymbol{\Phi}^{\text{(mv)}} \Big) \boldsymbol{\Delta}^{\text{m}}_1 \\
    \nonumber
  &  + \Big( \skewsym{\smallm{0 \\ - a_y \\ 0}} \!\!\! \boldsymbol{\Phi}^{\text{(fv)}} + \boldsymbol{B} \boldsymbol{\Phi}^{\text{(mv)}}  \Big) \boldsymbol{\Delta}^{\text{m}}_2 - 2 \boldsymbol{B} \boldsymbol{\Phi}^{\text{(mv)}}  \bigg) \lVert \boldsymbol{v_{\text{b}}} \rVert \boldsymbol{v_{\text{b}}} \\
    \nonumber
  & +\frac{1}{4} \rho S \bigg(\!\! \Big(\!\! \skewsym{\!\smallm{0 \\ a_y \\ 0}\!}\!\!\! \boldsymbol{\Phi}^{\text{(mv)}} \! + \! \boldsymbol{B} \boldsymbol{\Phi}^{\text{(m$\omega$)}} \Big) \boldsymbol{\Delta}^{\text{m}}_1 \\
    \nonumber
  & +  \Big(\!\! \skewsym{\!\smallm{0 \\ - a_y \\ 0}\!} \!\!\! \boldsymbol{\Phi}^{\text{(mv)}} \! + \! \boldsymbol{B} \boldsymbol{\Phi}^{\text{(m$\omega$)}}  \Big) \boldsymbol{\Delta}^{\text{m}}_2 - 2 \boldsymbol{B} \boldsymbol{\Phi}^{\text{(m$\omega$)}}\!  \bigg)\!  \boldsymbol{B}  \lVert \boldsymbol{v_{\text{b}}} \rVert  \boldsymbol{\omega}_{\text{b}} ,
% \end{split}
\end{align}
where $\boldsymbol{v}_{\text{b}} := \boldsymbol{R}^\top(\boldsymbol{q}) (\boldsymbol{v}-\boldsymbol{w})$ represents the air speed seen by the drone, represented in the body frame. In \cite{lustosa:hal-03035938}, the scalars $\lVert \boldsymbol{v_{\text{b}}} \rVert$ appearing in the expressions of $\boldsymbol{F}_{\text{b}}$ and $\boldsymbol{M}_{\text{b}}$ are replaced by the scalar $\eta = \sqrt{\lVert \boldsymbol{v_{\text{b}}} \rVert^{2} + \mu c^{2} \lVert \boldsymbol{\omega}_{\text{b}} \rVert^{2}}$, with $\mu \in \real$ being a parameter related to the model identification, but in the case of DarkO \cite{sansou:stage}, the identification provides $\mu = 0$, therefore we present a simplified description here. The constant aerodynamic coefficients' matrix 
$\boldsymbol{\Phi}:= \begin{bmatrix} \boldsymbol{\Phi}^{\text{(fv)}} & {\boldsymbol{\Phi}^{\text{(mv)}}}^\top \\ \boldsymbol{\Phi}^{\text{(mv)}} & \boldsymbol{\Phi}^{\text{(m$\omega$)}} \end{bmatrix} \in \real^{6 \times 6}$, is defined in \cite[eqs. (6)--(9)]{olszaneckibarth:hal-02542982} as $ \boldsymbol{\Phi}^{\text{(fv)}} \!:=\! \diag(C_{\text{d}},C_{y}, C_{\ell})$ and
\begin{align*}
&\left[ \begin{array}{c|c}
    \boldsymbol{\Phi}^{\text{(mv)}}  &  \boldsymbol{\Phi}^{\text{(m$\omega$)}} 
\end{array}\right] :=\\ 
&\left[ \begin{array}{ccc|ccc}
    0 & 0 & 0    &                                          0.1396 & 0 & 0.0573 \\
    0 & 0 & \!\!\!\!\! -\frac{\Delta_{\text{r}}}{c}C_{\ell} &    0 &  0.6358  & 0 \\
    0 & 0 & 0 &     0.0405 & 0 & 0.0019 
\end{array}\right],
\end{align*}
the numerical values of the constants being reported in Table~\ref{tab:pars} (these numerical values were not reported in \cite{lustosa:hal-03035938} and \cite{olszaneckibarth:hal-02542982} and are given here to allow reproducing our simulation results). The numerical values in Table \ref{tab:pars} have been obtained by a model identification campaign \cite{sansou:stage}. In particuler, coefficient $k_{\text{f}}$ was identified from the equation \eqref{eq:thrust}, which links the motor rotation speed $\omega_{i}$ with the generated traction, the minimum and maximum rotation speed and the time constant of the motor actuation chain. The diagonal elemnts of the inertia $\boldsymbol{J}$ were measured using a bifilar pendulum system. This method is widely used in the drone field \cite{Jardin2007OptimizedMO}, and is based on the period of oscillation about each one of the three axes ($x_{\text{b}}$,$y_{\text{b}}$,$z_{\text{b}}$) of the drone suspended by two wires, which forms a torsion pendulum as shown in Fig. \ref{fig:BifilarPend}.
It is interesting to note that the surface area blown by the propellers represents 67 percent of the drone's total surface area.


\begin{figure}[ht!]
    \centerline{
    \includegraphics[trim=0cm 0cm 0cm 0cm,clip,width=1\columnwidth]{figures/ident_motor March 27 2024 1651.png}}
    \caption{Input-output response of an Esc-Motor-Propeller assembly.}
    \label{IOmot}
\end{figure}

\begin{figure}[ht!]
    \centerline{
    \includegraphics[trim=20cm 15cm 23cm 0cm,clip,width=0.6\columnwidth]{figures/IMG_20230609_085023.jpg}}
    \caption{Bifilar pendulum mounting for the identication of $\boldsymbol{J}$.}
    \label{fig:BifilarPend}
\end{figure}
\chapter{Objectifs de commande}
\minitoc
\label{chap:objectif}

\section{section1}

\todo{rejet de perturbation, model based control}
\chapter{Modélisation d'un drone convertible : DarkO}
\minitoc
\label{chap:model}

\section{Modèle du drone DarkO}
\label{sec:model}
DarkO est un drone conçu et développé à l'École Nationale de l'Aviation Civile (ENAC) de Toulouse (France), est un exemple clair de drone convertible avec une architecture dite \textit{tailsitter}.
DarkO est assemblé à partir de plusieurs pièces d'Onyx imprimées en 3D (un matériau très robuste composé de fibres de carbone omnidirectionnelles). Toutes les pièces sont emboîtées sur un seul axe, de sorte que le drone puisse facilement être démonté pour remplacer des pièces ou accéder à l'électronique embarquée. 

L'autopilote embarqué est une carte Apogee~\footnote{\url{https://wiki.paparazziuav.org/wiki/Apogee/v1.00}} fabriquée à l'ENAC, voir Fig. \ref{fig:apogee}. 


\begin{figure}[ht]
    \centering
        \includegraphics[width=0.8\columnwidth]{figures/800px-Apogee_v100_top_1E.jpeg}
        \caption{Vue de dessus d'un autopilote Apogee v1.00.}
        \label{fig:apogee}
\end{figure}

L'autopilote offre la possibilité d'enregistrer les données de bord sur une carte mémoire SD, à la fréquence de contrôle de 500 Hz, ce qui permet un post-traitement efficace des données acquises. Le protocole de communication utilisé entre l'autopilote et les contrôleurs électroniques de vitesse (ESC) est le Dshot 600. Les ESC sont des AIKON AK32 35A \todo{trouver un synonyme} flasher avec un firmware AM32. La communication sol-bord est réalisée via un canal bidirectionnel basé sur des modules XBee-PRO S1.

\nomenclature[]{\(ESC\)}{Contrôleurs électroniques de vitesse (\textit{Electronic Speed Controller})}

\begin{figure}[ht]
    \centering
    \includegraphics[width=1\columnwidth]{figures/darko.pdf}
    \caption{Repère de référence de DarkO avec une représentation schématique des actionneurs.}
    \label{fig:darko2}
\end{figure}

Les actionneurs de DarkO peuvent être décomposé en deux catégories. La première est composée de deux hélices (T-Motor T5147) placées symétriquement à l'avant de l'aile (illustrées en \textbf{noir} dans la Fig. \ref{fig:darko2}) alimentées par deux moteurs électriques (T-Motor F30 2300kv) générant une traction selon l'axe $x_{b}$. La seconde catégorie est relative aux actionneurs aérodynamiques ainsi le drone possède deux élevons, placés à l'arrière de l'aile (illustrés en \textcolor{cyan}{bleu} dans la Fig. \ref{fig:darko2}), agissant en tant que surfaces de contrôle. Les élevons génèrent des forces et des moments en modifiant leurs incidences relativement au flux d'air dans lequel ils sont placé. Ce flux d'air peut être généré par le vent relatif (liée à la vitesse du drone), le vent extérieur, mais aussi par le souffle des hélices. Les élevons sont commandés par deux servomoteurs MKS DS65K.

La figure \ref{fig:darko2} montre le modèle de DarkO, ainsi qu'un repère de référence inertiel NED (ou repère terrestre) ``$\text{i}$'' lié à la surface de la Terre, et un repère corps `$\text{b}$'' attaché au drone, avec $x_{\text{b}}$ correspondant à l'axe de roulis (l'axe des hélices dans le plan $z_{\text{b}} =0$), $y_{\text{b}}$ l'axe de tangage (la direction des ailes), $z_{\text{b}}$ l'axe de lacet. En utilisant la même notation que dans \cite{lustosa:hal-03035938}, le couple hélice/élévateur gauche et droit sont désignés par les indices $i=1$ (gauche) et $i=2$ (droite). La convention de signe sera définie comme positive pour les positions des élevons $\delta_{1}$, $\delta_{2}$ lorsqu'ils créent un moment à cabrer avec les hélices tournant dans des directions opposées avec des vitesses angulaires $\omega_{1} > 0$ et $\omega_{2} < 0$, respectivement.

\begin{table}[ht]
    \centering
      \begin{tabular}{|l|c|c|}
        \hline
        \multicolumn{1}{|c|}{Paramètres et coefficients} & Valeurs & Unités \\
        \hline
        $m$ (Masse du drone)  & 0.519 & \SI{}{\kilogram} \\
        \hline
        $b$ (Envergure)  & 0.542 & \SI{}{\meter} \\
        \hline
        $c$ (Corde aérodynamique)  & 0.13 & \SI{}{\meter} \\
        \hline
        $\boldsymbol{B}=\diag(b,c,b)$ & $\!\!\diag(0.542, 0.13, 0.542)$ \!\! & \SI{}{\meter}\\
        \hline
        $S$ (Surface de l'aile) & 0.026936 & \SI{}{\square\meter}\\
        \hline
        $S_{\text{wet}}$ (Surface soufflée) & 0.0180 & \SI{}{\square\meter}\\
        \hline
        $S_{\text{p}}$ (Surface de helice) & 0.0127 & \SI{}{\square\meter}\\
        \hline
        $\boldsymbol{J}=\diag(J_{x},J_{y},J_{z})$ & \!\! $\diag(0.0067,0.0012,0.0082)$\!\! & \SI{}{\kilogram\square\meter}\\
        \hline
        $k_{\text{f}}$ (Poussée des hélices) & 1.7800e-8 & \SI{}{\kilogram\meter}\\
        \hline
        $k_{\text{m}}$(Moment des hélices) & 2.1065e-10 & \SI{}{\kilogram\square\meter}\\
        \hline
        $p_{x}$ (Position en $x$ des hélices) & 0.065 & \SI{}{\meter}\\
        \hline
        $p_{y}$ (Position en $y$  des hélices) & 0.162 & \SI{}{\meter}\\
        \hline
        $a_{y}$ (Position en $y$ de la portance) & 0.1504 & \SI{}{\meter}\\
        \hline
        $\xi_{\text{f}}$ (Portance des élevons) & 0.2 & --\\
        \hline
        $\xi_{\text{m}}$ (Moment des élevons) & 1.4 & --\\
        \hline
        $\rho$ (Densité de l'air) & 1.225 & \SI{}{\kilogram\per\cubic\meter}\\
        \hline
        $C_{\text{d}}$ (Trainé) & 0.1644 & --\\
        \hline
        $C_{y}$ (Lateral) & 0 & --\\
        \hline
         $C_{\ell}$ (Portance) & 5.4001 & --\\
        \hline
        $\Delta_{\text{r}}$ (Centrage du drone) & -0.0145 & \SI{}{\meter}\\
        \hline
      \end{tabular}
      \caption{\label{tab:pars} Paramètres numériques identifiés du modèle DarkO.}
\end{table}

\subsection{Modèle non-linéaire complet}

En exploitant la modélisation présentée dans \cite{lustosa:hal-03035938} et \cite{olszaneckibarth:hal-02542982}, un modèle précis de la dynamique de DarkO décrit la position $\boldsymbol{p} \in \real^3$ du centre de gravité et sa vitesse $\boldsymbol{v} = \dot{\boldsymbol{p}} \in \real^3$, en plus de son orientation, bien représentée par un quaternion $\boldsymbol{q} \in {\mathbb S}^3:=\{ \boldsymbol{q} \in \real^4 : |\boldsymbol{q}| = 1\}$ et de sa vitesse angulaire $\boldsymbol{\omega}_{\text{b}}$ représentée dans le repère du corps, qui satisfait $boldsymbol{\dot q} = \frac{1}{2}\boldsymbol{q} \otimes \smallmat{0 \\ \boldsymbol{\omega}_{{\text{b}}}}$, où $\otimes$ représente le produit de Hamilton (voir \cite{lustosa:hal-03035938,olszaneckibarth:hal-02542982} ou le tutoriel \cite{hamel_minhduc} pour plus de détails). En choisissant l'état global comme $\boldsymbol{x}:=(\boldsymbol{p},~ \boldsymbol{v},~ \boldsymbol{q},~ \boldsymbol{\omega}_{\text{b}})$, le modèle mathématique dérivé dans \cite{lustosa:hal-03035938}, dépendent d'un ensemble de paramètres énumérés dans le tableau \ref{tab:pars}, où nous indiquons également la valeur obtenue à partir d'une identification du système \cite{sansou:stage}. Le modèle dynamique peut être écrit comme ci-dessous :

\begin{subequations}\label{eq:dyna_orig}
    \begin{empheq}[left=\empheqlbrace]{alignat=2}
          % \boldsymbol{\dot p} & &= & \boldsymbol{v} \label{eq:dyna1}\\
          m\boldsymbol{\dot v} &&=& - m\boldsymbol{g} +  \boldsymbol{R}(\boldsymbol{q})\boldsymbol{F}_{\text{b}},\\
          \label{eq:dyna_orig_b}
          \boldsymbol{J} \boldsymbol{\dot \omega_{\text{b}}} &&= &  - \skewsym{\boldsymbol{\omega}_{\text{b}}}\boldsymbol{J}\boldsymbol{\omega}_{\text{b}} + \boldsymbol{M}_{\text{b}},
    \end{empheq}
  \end{subequations}
  où $\boldsymbol{g}:=\smallmat{0 & 0& 9.81}^\top$ désigne le vecteur de gravité, $m\in \real$ est la masse, $\boldsymbol{J}\in \real^{3\times 3}$ est le moment d'inertie diagonal (voir Tableau~\ref{tab:pars}) et en partitionnant le quaternion $\boldsymbol{q} \in {\mathbb S}^3$ comme $\boldsymbol{q} := \left[ \eta ~ \boldsymbol{\epsilon}^\top \right]^\top$, la matrice de rotation correspondante est $\boldsymbol{R}(\boldsymbol{q}) \in SO(3): = \{\boldsymbol{R}\in \real^{3\times 3}: \; \boldsymbol{R}^\top \boldsymbol{R} = \mathbb{I}_{3}, \det (\boldsymbol{R})=1\}$ est défini comme (voir \cite{hamel_minhduc})
\begin{align}
    \label{eq:matrix_rot}
    \boldsymbol{R}(\boldsymbol{q}) := \mathbb{I}_{3} +2\eta \skewsym{\boldsymbol{\epsilon}} + 2\skewsym{\boldsymbol{\epsilon}}^{2}.
\end{align}


D'après \cite{lustosa:hal-03035938} le vecteur de force et de moment $\boldsymbol{F}_{\text{b}}$ et $\boldsymbol{M}_{\text{b}}$ dans \eqref{eq:dyna_orig} dépendent  (i) de l'état du système $\boldsymbol{x}$, (ii) de la perturbation $\boldsymbol{w} \in \real^3$, représentant la vitesse du vent dans le référentiel inertiel, et (iii) de la commande des actionneurs (voir Figure~\ref{fig:darko2}), comprenant la vitesse de rotation des deux hélices $\omega_1, \omega_2 \in \real$ et la déflexion des élevons $\delta_1, \delta_2\in \real$.
Considérons d'abord l'effet des commandes des actionneurs. Chaque hélice génère une poussée $\boldsymbol{T}_i$ orienté dans la direction $x$ du repère corps et un moment $\boldsymbol{N}_i$ selon le même axe :
\begin{align}
\label{eq:thrust}
\boldsymbol{T}_{i} \!:=\! \begin{bmatrix} \tau_{i} \\ 0 \\ 0 \end{bmatrix} \!:=\!
\begin{bmatrix} k_{\text{f}}\omega_{i}^{2} \\ 0 \\ 0 \end{bmatrix}\! , \;
\boldsymbol{N}_{i} \!:=\! (-1)^{i}  \frac{k_{\text{m}} }{k_{\text{f}}}\boldsymbol{T}_{i}, \quad i=1,2 .
\end{align} 

La position de chaque élevon $\delta_i \in \real$ est assignée par un servomoteur qui impose un niveau d'efficacité (en termes de déviation du courant d'air) quantifié par deux matrices antisymétriques :
\begin{align}
\label{eq:elevons_efficiency}
    \boldsymbol{\Delta}^{\text{f}}_{i} \!:=\! \begin{bmatrix} 0 & 0 & \xi_{\text{f}}\delta_{i} \\ 0 & 0 & 0 \\ -\xi_{\text{f}}\delta_{i} & 0 & 0 \end{bmatrix}\! ,\;
    \boldsymbol{\Delta}^{\text{m}}_{i} \!:=\! \begin{bmatrix} 0 & 0 & \xi_{\text{m}}\delta_{i} \\ 0 & 0 & 0 \\ -\xi_{\text{m}}\delta_{i} & 0 & 0 \end{bmatrix} \!, 
\end{align}
$i=1,2$. Les paramètres constants $k_{\text{f}}$, $k_{\text{m}}$, $\xi_{\text{f}}$, $\xi_{\text{m}}$ apparaissant dans \eqref{eq:thrust} et \eqref{eq:elevons_efficiency} sont listés dans la Table~\ref{tab:pars}.\\
Avec les quantités ci-dessus, nous pouvons réarranger la dynamique donnée dans le tableau suivant \cite[eqns (97),~(98)]{lustosa:hal-03035938} (voir aussi \cite{sansou:stage}) et exprimer $\boldsymbol{F}_{\text{b}}$ et $\boldsymbol{M}_{\text{b}}$ dans \eqref{eq:dyna_orig} comme
%
\begin{align}
\nonumber
    \boldsymbol{F}_{\text{b}} :={}&  \boldsymbol{T}_{1} + \boldsymbol{T}_{2} + \frac{S_{\text{wet}}}{4S_{\text{p}}} \boldsymbol{\Phi}^{\text{(fv)}} \Big( (\boldsymbol{\Delta}^{\text{f}}_1 - \mathbb{I}_{3} ) \boldsymbol{T}_{1} + ( \boldsymbol{\Delta}^{\text{f}}_2 - \mathbb{I}_{3}) \boldsymbol{T}_{2}\Big) \\ 
     \label{eq:Fb}
    &+ \frac{1}{4} \rho S  \boldsymbol{\Phi}^{\text{(fv)}} \Big(\boldsymbol{\Delta}^{\text{f}}_1+ \boldsymbol{\Delta}^{\text{f}}_2 - 2 \mathbb{I}_{3} \Big) \lVert \boldsymbol{v_{\text{b}}} \rVert \boldsymbol{v_{\text{b}}}\\
    \nonumber
    &+ \frac{1}{4} \rho S \boldsymbol{\Phi}^{\text{(mv)}} \Big(\boldsymbol{\Delta}^{\text{f}}_1 + \boldsymbol{\Delta}^{\text{f}}_2 - 2\mathbb{I}_{3}\Big) \boldsymbol{B} \lVert \boldsymbol{v_{\text{b}}} \rVert  \boldsymbol{\omega}_{\text{b}}, 
\end{align}
\begin{align} 
\label{eq:Mb}
 \boldsymbol{M}_{\text{b}} :&=\boldsymbol{N}_{1} + \boldsymbol{N}_{2} + \skewsym{\smallm{p_x\\ p_y\\ 0}} \boldsymbol{T}_{1} + \skewsym{\smallm{p_x\\ - p_y\\ 0}} \boldsymbol{T}_{2}\\
    \nonumber
  &- \frac{S_{\text{wet}}}{4S_{\text{p}}} \bigg( \boldsymbol{B} \boldsymbol{\Phi}^{\text{(mv)}} (\boldsymbol{\Delta}^{\text{m}}_1- \mathbb{I}_{3} ) + \skewsym{\smallm{0 \\ a_y \\ 0}} \boldsymbol{\Phi}^{\text{(fv)}} (\boldsymbol{\Delta}^{\text{m}}_1 +\mathbb{I}_{3} ) \bigg) \boldsymbol{T}_{1} \\
    \nonumber
  & - \frac{S_{\text{wet}}}{4S_{\text{p}}} \bigg( \boldsymbol{B} \boldsymbol{\Phi}^{\text{(mv)}} (\boldsymbol{\Delta}^{\text{m}}_2 - \mathbb{I}_{3} ) +  \skewsym{\smallm{0 \\ - a_y \\ 0}} \boldsymbol{\Phi}^{\text{(fv)}} (\boldsymbol{\Delta}^{\text{m}}_2 + \mathbb{I}_{3}) \bigg) \boldsymbol{T}_{2} \\
    \nonumber
  & + \frac{1}{4} \rho S  \bigg( \Big(\skewsym{\smallm{0 \\ a_y \\ 0}} \!\!\! \boldsymbol{\Phi}^{\text{(fv)}}  + \boldsymbol{B} \boldsymbol{\Phi}^{\text{(mv)}} \Big) \boldsymbol{\Delta}^{\text{m}}_1 \\
    \nonumber
  &  + \Big( \skewsym{\smallm{0 \\ - a_y \\ 0}} \!\!\! \boldsymbol{\Phi}^{\text{(fv)}} + \boldsymbol{B} \boldsymbol{\Phi}^{\text{(mv)}}  \Big) \boldsymbol{\Delta}^{\text{m}}_2 - 2 \boldsymbol{B} \boldsymbol{\Phi}^{\text{(mv)}}  \bigg) \lVert \boldsymbol{v_{\text{b}}} \rVert \boldsymbol{v_{\text{b}}} \\
    \nonumber
  & +\frac{1}{4} \rho S \bigg(\!\! \Big(\!\! \skewsym{\!\smallm{0 \\ a_y \\ 0}\!}\!\!\! \boldsymbol{\Phi}^{\text{(mv)}} \! + \! \boldsymbol{B} \boldsymbol{\Phi}^{\text{(m$\omega$)}} \Big) \boldsymbol{\Delta}^{\text{m}}_1 \\
    \nonumber
  & +  \Big(\!\! \skewsym{\!\smallm{0 \\ - a_y \\ 0}\!} \!\!\! \boldsymbol{\Phi}^{\text{(mv)}} \! + \! \boldsymbol{B} \boldsymbol{\Phi}^{\text{(m$\omega$)}}  \Big) \boldsymbol{\Delta}^{\text{m}}_2 - 2 \boldsymbol{B} \boldsymbol{\Phi}^{\text{(m$\omega$)}}\!  \bigg)\!  \boldsymbol{B}  \lVert \boldsymbol{v_{\text{b}}} \rVert  \boldsymbol{\omega}_{\text{b}} ,
\end{align}
où $\boldsymbol{v}_{\text{b}} := \boldsymbol{R}^\top(\boldsymbol{q}) (\boldsymbol{v}-\boldsymbol{w})$ représente la vitesse de l'air vu par le drone exprimé dans le repère du corps. Dans \cite{lustosa:hal-03035938}, la valeur $\lVert \boldsymbol{v_{\text{b}}} \rVert$ apparaissatn dans les expressions de  $\boldsymbol{F}_{\text{b}}$ et $\boldsymbol{M}_{\text{b}}$ est remplacé par la valeur $\eta = \sqrt{\lVert \boldsymbol{v_{\text{b}}} \rVert^{2} + \mu c^{2} \lVert \boldsymbol{\omega}_{\text{b}} \rVert^{2}}$, avec $\mu \in \real$ étant un paramètre lié à l'identification du modèle, mais dans le cas de DarkO \cite{sansou:stage}, l'identification fournit $\mu = 0$, c'est pourquoi nous présentons ici une description simplifiée. La matrice des coefficients aérodynamiques constants 
$\boldsymbol{\Phi}:= \begin{bmatrix} \boldsymbol{\Phi}^{\text{(fv)}} & {\boldsymbol{\Phi}^{\text{(mv)}}}^\top \\ \boldsymbol{\Phi}^{\text{(mv)}} & \boldsymbol{\Phi}^{\text{(m$\omega$)}} \end{bmatrix} \in \real^{6 \times 6}$, est défini dans \cite[eqs. (6)--(9)]{olszaneckibarth:hal-02542982} comme $ \boldsymbol{\Phi}^{\text{(fv)}} \!:=\! \diag(C_{\text{d}},C_{y}, C_{\ell})$ et
\begin{align*}
&\left[ \begin{array}{c|c}
    \boldsymbol{\Phi}^{\text{(mv)}}  &  \boldsymbol{\Phi}^{\text{(m$\omega$)}} 
\end{array}\right] :=\\ 
&\left[ \begin{array}{ccc|ccc}
    0 & 0 & 0    &                                          0.1396 & 0 & 0.0573 \\
    0 & 0 & \!\!\!\!\! -\frac{\Delta_{\text{r}}}{c}C_{\ell} &    0 &  0.6358  & 0 \\
    0 & 0 & 0 &     0.0405 & 0 & 0.0019 
\end{array}\right],
\end{align*}
les valeurs numériques des constantes figurant dans le tableau \ref{tab:pars} (ces valeurs numériques n'ont pas été indiquées dans \cite{lustosa:hal-03035938} et \cite{olszaneckibarth:hal-02542982} et sont données ici pour permettre de reproduire les résultats de nos simulations). 


\subsection{Modèle non linéaire simplifiée à basse vitesse}

Comme nous allons nous intéresser au maintien du drone en stationnaire, où la vitesse du drone est faible, nous pouvons simplifier la dynamique \eqref{eq:dyna_orig} en négligeant les effets aérodynamique quadratique dû à la vitesse $\boldsymbol{v_{\text{b}}}$ et à la vitesse angulaire $\boldsymbol{\omega}_{\text{b}}$ dans \eqref{eq:Fb} et \eqref{eq:Mb}. 
Nous définissons le vecteur de commande :
\begin{align}
\label{eq:vector_u}
    \boldsymbol{u} := \begin{bmatrix}\tau_{1}  \!&\! \tau_{2}  \!&\! \delta_{1} \!&\! \delta_{2} \end{bmatrix}^\top,
\end{align}
qui permet d'obtenir le modèle basse vitesse comportant majeur les effets non linéaires du vent  
\begin{subequations}\label{eq:dyna_simp}
    \begin{alignat}{3}
    &\boldsymbol{\dot p} = \boldsymbol{v}, \label{eq:dyna1}\\
       & m\boldsymbol{\dot v} =\! \shortminus m\boldsymbol{g} \!+ \! \boldsymbol{R} (\boldsymbol{q}) \!\!\left( \! \boldsymbol{M}_{\text{f}}(\boldsymbol{u}) \! + \! \boldsymbol{D}_{\text{f}}(\boldsymbol{u}) \| \boldsymbol{w} \| \boldsymbol{R}^\top \!(\boldsymbol{q}) (\boldsymbol{v} \! \shortminus \! \boldsymbol{w}) \! \right)\!,\!\! \label{eq:dyna2}\\
        &\boldsymbol{\dot q} = \frac{1}{2}\boldsymbol{q} \otimes \smallmat{0 \\ \boldsymbol{\omega}_{\text{b}}}, \label{eq:dyna3}\\
        &\boldsymbol{J} \boldsymbol{\dot \omega}_{\text{b}} = \shortminus \skewsym{\boldsymbol{\omega}_{\text{b}}}\boldsymbol{J}\boldsymbol{\omega}_{\text{b}}\! + \boldsymbol{M}_{\text{m}}(\boldsymbol{u})\! + \boldsymbol{D}_{\text{m}} (\boldsymbol{u}) \lVert \boldsymbol{w} \rVert \boldsymbol{R}^\top(\boldsymbol{q}) (\boldsymbol{v}-\boldsymbol{w}), \label{eq:dyna4}
    \end{alignat}
\end{subequations}
où les vecteurs $\boldsymbol{M}_{\text{f}}(\boldsymbol{u})$ et $ \boldsymbol{M}_{\text{m}}(\boldsymbol{u})$, et les matrices $\boldsymbol{D}_{\text{f}}(\boldsymbol{u})$ et $\boldsymbol{D}_{\text{m}} (\boldsymbol{u})$ proviennent de l'annulation des termes dépendant de la vitesse angulaire dans l'équation \eqref{eq:Fb} et \eqref{eq:Mb}. Ils peuvent etre développer en 
\begin{align}
\label{eq:Mf}
    &\boldsymbol{M}_{\text{f}}(\boldsymbol{u}) :=  \boldsymbol{T}_{1} \!+\! \boldsymbol{T}_{2} \!+\! \frac{S_{\text{wet}}}{4S_{\text{p}}} \boldsymbol{\Phi}^{\text{(fv)}} \Big( (\boldsymbol{\Delta}^{\text{f}}_1 \shortminus \mathbb{I}_{3} ) \boldsymbol{T}_{1} + ( \boldsymbol{\Delta}^{\text{f}}_2 \shortminus \mathbb{I}_{3}) \boldsymbol{T}_{2}\Big) \nonumber\\
& \hspace*{1.2cm} =\begin{bmatrix} \left(1-\frac{S_{\text{wet}}}{4S_{\text{p}}} C_{\text{d}}\right) (\tau_{1} + \tau_{2}) \\  0  \\ -\frac{S_{\text{wet}}}{4S_{\text{p}}}C_{\ell}\xi_{\text{f}} \left(\delta_{1}\tau_{1} + \delta_{2}\tau_{2}\right) \end{bmatrix}  \\
\nonumber
 & \boldsymbol{M}_{\text{m}}(\boldsymbol{u}) := \boldsymbol{N}_{1} + \boldsymbol{N}_{2} + \skewsym{\smallm{p_x\\ p_y\\ 0}} \boldsymbol{T}_{1} + \skewsym{\smallm{p_x\\ -p_y\\ 0}} \boldsymbol{T}_{2}\\
 \nonumber
   &\qquad \shortminus \frac{S_{\text{wet}}}{4S_{\text{p}}}\! \bigg( \boldsymbol{B} \boldsymbol{\Phi}^{\text{(mv)}} (\boldsymbol{\Delta}^{\text{m}}_1- \mathbb{I}_{3} ) \! + \! \skewsym{\smallm{0 \\ a_y \\ 0}} \!\! \boldsymbol{\Phi}^{\text{(fv)}} (\mathbb{I}_{3} + \boldsymbol{\Delta}^{\text{m}}_1 ) \bigg) \boldsymbol{T}_{1} \\
   \nonumber
   &\qquad \shortminus \frac{S_{\text{wet}}}{4S_{\text{p}}} \!\bigg( \boldsymbol{B} \boldsymbol{\Phi}^{\text{(mv)}} (\boldsymbol{\Delta}^{\text{m}}_2 - \mathbb{I}_{3} ) \! + \! \skewsym{\smallm{0 \\ \!\shortminus a_y \!\\ 0}} \!\! \boldsymbol{\Phi}^{\text{(fv)}} (\mathbb{I}_{3} + \boldsymbol{\Delta}^{\text{m}}_2 ) \bigg) \boldsymbol{T}_{2} \\
   \label{eq:Mm}
   & \hspace*{1.2cm} =\begin{bmatrix} \frac{k_{\text{m}} }{k_{\text{f}}}(\tau_{1} - \tau_{2}) + \frac{S_{\text{wet}}}{4S_{\text{p}}}a_{y}C_{\ell}\xi_{\text{f}}(\delta_{1}\tau_{1} - \delta_{2}\tau_{2}) \\
   \frac{S_{\text{wet}}}{4S_{\text{p}}} \Delta_{\text{r}}C_{\ell}\xi_{\text{m}}(\delta_{1}\tau_{1} + \delta_{2}\tau_{2}) \\
   \left(p_{y}+\frac{S_{\text{wet}}}{4S_{\text{p}}} a_{y} C_{\text{d}}\right)(\tau_{1} - \tau_{2})
   \end{bmatrix} \\
    \label{eq:df}
             &\boldsymbol{D}_{\text{f}}(\boldsymbol{u}) :=  \frac{1}{4} \rho S  \boldsymbol{\Phi}^{\text{(fv)}} \Big( \boldsymbol{\Delta}^{\text{f}}_1+  \boldsymbol{\Delta}^{\text{f}}_2 - 2 \mathbb{I}_{3} \Big)   \\
            &\hspace*{1.1cm} = \frac{1}{4}\rho S  \begin{bmatrix}
                -2C_{\text{d}} & 0 & C_{\text{d}}\xi_{\text{f}}(\delta_{1} +\delta_{2})\\
                0 & 0 & 0\\
                -C_{\ell}\xi_{\text{f}}(\delta_{1} +\delta_{2}) & 0 & -2C_{\ell}
            \end{bmatrix} \nonumber \\
      \label{eq:dm}      
      & \boldsymbol{D}_{\text{m}}(\boldsymbol{u}) := \frac{1}{4} \rho S  \bigg( \Big(\skewsym{\smallm{0 \\ a_y \\ 0}} \!\!\!  \boldsymbol{\Phi}^{\text{(fv)}}  +  \boldsymbol{B}  \boldsymbol{\Phi}^{\text{(mv)}} \Big)  \boldsymbol{\Delta}^{\text{m}}_1  \\
       &\hspace{0.8cm}  + \Big( \skewsym{\smallm{0 \\ -a_y \\ 0}} \!\!\!  \boldsymbol{\Phi}^{\text{(fv)}} +  \boldsymbol{B}  \boldsymbol{\Phi}^{\text{(mv)}}  \Big)  \boldsymbol{\Delta}^{\text{m}}_2 - 2  \boldsymbol{B}  \boldsymbol{\Phi}^{\text{(mv)}}  \bigg) \nonumber \\
       \nonumber
         & \hspace*{1.1cm} = \frac{1}{4}\rho S  \begin{bmatrix}
                \shortminus a_{y}C_{\text{d}}\xi_{\text{m}}(\delta_{1} \!-\! \delta_{2}) \!\! & 0 & 0\\
                \Delta_{\text{r}}C_{\ell}\xi_{\text{m}}(\delta_{1} +\delta_{2})\!\! & 0 & 2\Delta_{\text{r}}C_{\ell}\\
               0 & 0 & \!\! \shortminus a_{y}C_{\ell}\xi_{\text{m}}(\delta_{1} \!-\!\delta_{2})
            \end{bmatrix} 
    \end{align}
où l'on observe l'effet non linéaire d'un vent non nul, qui est non linéaire avec $\boldsymbol{q}$, $\lVert \boldsymbol{v}_{\text{b}} \rVert$ et $\boldsymbol{w}$. Comme dans \cite[eqn. (10)]{olszaneckibarth:hal-02542982} et selon la formule de Diederich, nous obtenons $C_{\ell} = C_{\text{d}} + \frac{\pi AR}{1+\sqrt{1+\left(\frac{AR}{2}\right)^{2}}}$ où $AR = \frac{b^{2}}{S}$ est l'allongement de l'aile.
Nous observons le couplage des actionneurs $\left(\delta_{1}\tau_{1} + \delta_{2}\tau_{2}\right)$  dans les expressions des matrices $\boldsymbol{M}_{\text{f}}(\boldsymbol{u})$ et $\boldsymbol{M}_{\text{m}}(\boldsymbol{u})$.

\section{Identification des paramètres du modèle}
    Les valeurs numériques du tableau \ref{tab:pars} ont été obtenues par une campagne d'identification du modèle \cite{sansou:stage}. En particulier, le coefficient $k_{\text{f}}$ a été identifié à partir de l'équation \eqref{eq:thrust}, qui relie la vitesse de rotation du moteur $\omega_{i}$ à la traction générée, à la vitesse de rotation minimale et maximale et à la constante de temps de la chaîne d'actionnement du moteur. Les éléments diagonaux de l'inertie $\boldsymbol{J}$ ont été mesurés à l'aide d'un système de pendule bifilaire. Cette méthode est largement utilisée dans le domaine des drones \cite{Jardin2007OptimizedMO}, et est basée sur la période d'oscillation autour de chacun des trois axes ($x_{{\text{b}}}$, $y_{text{b}}$, $z_{\text{b}}$) du drone suspendu par deux fils, ce qui forme un pendule de torsion comme le montre la Fig. \ref{fig:BifilarPend}.
    Il est intéressant de noter que la surface soufflée par les hélices représente 67 \% de la surface totale du drone.


    \begin{figure}[ht!]
        \centerline{
        \includegraphics[trim=0cm 0cm 0cm 0cm,clip,width=0.5\columnwidth]{figures/ident_motor March 27 2024 1651.png}}
        \caption{Input-output response of an Esc-Motor-Propeller assembly.}
        \label{IOmot}
    \end{figure}

    \begin{figure}[ht!]
        \centerline{
        \includegraphics[trim=20cm 15cm 23cm 0cm,clip,width=0.6\columnwidth]{figures/IMG_20230609_085023.jpg}}
        \caption{Bifilar pendulum mounting for the identication of $\boldsymbol{J}$.}
        \label{fig:BifilarPend}
    \end{figure}


    \begin{figure}[ht]
    \centerline{
    \includegraphics[trim=0cm 0cm 0cm 0cm,clip,width=0.6\columnwidth]{figures/ident_inertia.png}}
    \caption{Bifilar pendulum identication of $\boldsymbol{J}$.}
    \label{fig:BifilarPend_meas}
    \end{figure}
    \todo{ajouter explication sur l'indentification}

\subsection{Modélisation des actionneurs}
    Les actionneurs de DakO ont des dynamqies qui limite leurs actions en terme d'amplitude et de vitesse.

    Pour les moteurs électriques générant la traction par les hélices, il existe deux causes de saturation. Une saturation à haute vitesse liée à  la tension maximale du moteur et une saturation basse vitesse lié à la vitesse minimale de commutation de bobine du moteur pour maintenir la rotation. De plus, ces saturations permettent d'obtenir une modèle réaliste à énergie finie. Elle correspond à la contrainte suivante  $\omega_i \in [2500,~16000]~rpm = [262,~1675]~\SI{}{\radian\per\second}$, $i=1,2$. En termes de dynamique, nous avons représenté la chaîne d'actionnement du moteur (composée de l'ESC, du moteur et de l'hélice) par une fonction de transfert du premier ordre ayant une constante de temps égale à \SI{0,0125}{\second}, ce qui fournit un système d'actionnement assez agressif.

    Les saturations impactant les élevons proviennent des limites physiques des servomoteurs et du débattement limité par la forme de l'UAV, $\delta_i \in [-30~; 30]\text{\textdegree}$, $i=1,2$. La saturation la plus importante ici est peut-être la bande passante de l'actionneur (due à l'actionnement du servomoteur), qui est modélisée par une fonction de transfert du premier ordre avec une constante de temps \SI{0,05}{\second}. 

\section{Équilibres stationnaires}
    \subsection{Équilibre stationnaire sans vent}
    \label{sec:eq_nowind}

    Nous proposons une modification du vecteur de commande, dans le cas d'un équilibre sans vent $\boldsymbol{w}_{\mathrm{eq}} = 0$, basé sur le couplage des actionneurs. 
    \begin{align}
    \boldsymbol{u}_{\text{nowind}} := \begin{bmatrix}\tau_{1}  \!&\! \tau_{2}  \!&\! \delta_{1}\tau_{1} \!&\! \delta_{2}\tau_{2} \end{bmatrix}^\top
    \end{align}
    Nous obtenons un modèle linéaire vis-à-vis de sa commande, dérivé de \eqref{eq:dyna_simp} en imposant  $\boldsymbol{w} = 0$,
    \begin{subequations}\label{eq:withouwind}
    \begin{align}
            \boldsymbol{\dot p} &=  \boldsymbol{v}, \quad &
            m\boldsymbol{\dot v} &= - m\boldsymbol{g} +  \boldsymbol{R}(\boldsymbol{q})\boldsymbol{F}\boldsymbol{u}_{\text{nowind}},\\
            \boldsymbol{\dot q} &= \frac{1}{2}\boldsymbol{q} \otimes \smallmat{0 \\ \boldsymbol{\omega}_{\text{b}}} \quad & \boldsymbol{J} \boldsymbol{\dot \omega}_{\text{b}} &= \! \shortminus \skewsym{\boldsymbol{\omega}_{\text{b}}}\!J\boldsymbol{\omega}_{\text{b}} \! + \! \boldsymbol{M}\boldsymbol{u}_{\text{nowind}},
    \end{align}
    \end{subequations}
    avec les matrices
    \begin{align*}
        \left[ \begin{array}{c|c}\!\boldsymbol{F}\!&\!\boldsymbol{M}\!\end{array} \right] := \left[ \begin{array}{cccc | cccc} a_{\text{f}} & a_{\text{f}} & 0 & 0 & a_{\text{m}} & -a_{\text{m}} & b_{\text{m}} & -b_{\text{m}} \\  0 & 0 & 0 & 0 & 0 & 0 & c_{\text{m}} & c_{\text{m}} \\ 0 & 0 & b_{\text{f}} & b_{\text{f}} & d_{\text{m}} & -d_{\text{m}} & 0 & 0 \end{array} \right]
    \end{align*}
    et les scalaires
        \begin{align*}
            \left[\!\! \begin{array}{c|c} 
            a_{\text{f}} & b_{\text{f}} \\ \hline
            a_{\text{m}} & b_{\text{m}} \\ \hline
            c_{\text{m}} & d_{\text{m}}
            \end{array} \!\!\right] \!=\!
            \left[\begin{array}{c|c}
            1-\frac{S_{\text{wet}}}{4S_{\text{p}}} C_{\text{d}}  & -\frac{S_{\text{wet}}}{4S_{\text{p}}}C_{\ell}\xi_{\text{f}} \\ \hline
            \frac{k_{\text{m}} }{k_{\text{f}}}  &   \! \frac{S_{\text{wet}}}{4S_{\text{p}}}a_{y}C_{\ell}\xi_{\text{f}} \!\\ \hline
            \!\! \frac{S_{\text{wet}}}{4S_{\text{p}}} \Delta_{\text{r}}C_{\ell}\xi_{\text{m}} \!\! & 
            p_{y}+\frac{S_{\text{wet}}}{4S_{\text{p}}} a_{y} C_{\text{d}}
            \end{array}\right].
        \end{align*}


    Tous les couples d'équilibre $(\boldsymbol{u}_{\text{nowind}}, \boldsymbol{x}) = (\boldsymbol{u}_{\text{nowind},\text{eq}}, \boldsymbol{x_{\text{eq}}})$ sont paramétré par une rotation arbitraire autour de l'axe $z_{[\text{i}]}$ définit par $\beta \in \left[-\sqrt{\frac{1}{2}},\sqrt{\frac{1}{2}}\right]$. Le point d'équilibre a pour expression
    \begin{subequations}
        \label{eq:equilibria}
        \begin{align}
            \label{eq:bar_u}
            \boldsymbol{u}_{\text{nowind},\text{eq}} = \frac{mg}{( 1-\frac{S_{\text{wet}}}{4S_{\text{p}}} C_{\text{d}})} [1~1~0~0]^\top\\
            \boldsymbol{q}_{\text{eq}} = [\eta_{\text{eq}} ~\boldsymbol{\epsilon}_{\text{eq}}^\top]^\top = \smallmat{\sqrt{\frac{1}{2}-\beta} & \beta & \frac{2\beta^{2}-1}{2\sqrt{\frac{1}{2}-\beta}} & \beta}^\top.
        \end{align}
    \end{subequations}
    En présence d'un vent nul, le degré de liberté $\beta$ permet d'orienter le drone dans n'importe quelle direction horizontale.

    \subsection{Équilibre stationnaire en présence de vent}
    À partir des modèles \eqref{eq:dyna_orig} et \eqref{eq:dyna_simp}, nous caractérisons un équilibre stationnaire en présence d'un vent constant $\boldsymbol{w}_{\mathrm{eq}} =\smallmat{w_x \\ w_y \\w_z} \in \real^3$ exprimé dans le repère inertiel, tel que $\smallmat{w_x \\ w_y} \neq 0$, c'est-à-dire qu'il existe toujours un vent horizontal non nul.
    Ainsi, pour chaque position de référence $\boldsymbol{p}_{\text{eq}} \in \real^3$, 
    un ensemble de couple état/commande possible est $(\boldsymbol{u}_{\text{eq}}, \boldsymbol{x}_{\text{eq}}) = (\boldsymbol{u}_{\text{eq}}, \boldsymbol{p}_{\text{eq}}, \boldsymbol{v}_{\text{eq}}, \boldsymbol{q}_{\text{eq}}, \boldsymbol{\omega}_{\text{b},\text{eq}})$
    obtenu à l'aide de
    \begin{subequations}
    \label{eq:equilibrium}
    \begin{align}
    \label{eq:ueq}
            &\boldsymbol{u}_{\text{eq}} = \begin{bmatrix} \tau & \tau & \delta & \delta \end{bmatrix}^\top\\
            & \boldsymbol{q}_{\text{eq}} = \boldsymbol{q}_{\mathrm{eq}\psi} \otimes  \boldsymbol{q}_{\mathrm{eq}\theta} \label{eq:qeq}\\
            &\boldsymbol{\omega}_{\text{b},\text{eq}} = 0 , \quad \boldsymbol{v}_{\text{eq}} = 0, 
    \end{align}
    \end{subequations}
    où
    \begin{align}
    \label{eq:qtheta}
        \boldsymbol{q}_{\mathrm{eq}\theta} &:= \begin{bmatrix} \cos(\frac{\theta}{2}) & 0 & \sin(\frac{\theta}{2}) & 0 \end{bmatrix}^\top
    \end{align}
    \todo{traduire et amelioré l'explication}
    we define the quaternion $\boldsymbol{q}_{\mathrm{eq}\psi}$ associated with a horizontal rotation $\psi = \arctan(w_{x}, w_{y})$ of the inertial reference frame towards the (nonzero) horizontal wind direction:

    \begin{align}
    \label{eq:qpsi}
        \boldsymbol{q}_{\mathrm{eq}\psi} &:= \begin{bmatrix} \cos(\frac{\psi}{2}) & 0 & 0 & \sin(\frac{\psi}{2}) \end{bmatrix}^\top.
    \end{align}
    et l'angle d'inclinaison $\theta$, la poussée des hélices $\tau$, et la déflexion des élevons $\delta$ peuvent être obtenu à partir de l'algorithme~\ref{alg:eq}. 

    \begin{algorithm}
    \caption{Obtention des paramètres d'équilibre en \eqref{eq:equilibrium}.}
    \label{alg:eq}
    \hspace*{.1cm} \textbf{Entrée}: Vecteur vent $\boldsymbol{w}_{\text{eq}} =\smallmat{w_x & w_y & w_z}^\top$ \\
    \hspace*{.1cm} \textbf{Sortie}: Paramètres $\psi$, $\theta$, $\tau$, $\delta$ dans \eqref{eq:equilibrium}
    \begin{algorithmic}[1]
        %\Require {\bf Input} values: $\boldsymbol{w} =\smallmat{w_x & w_y & w_z}^\top$ 
        %\Ensure  $\psi$, $\theta$, $\tau$, $\delta$
        \State Détermine l'angle $\psi = \text{atan2}(w_x, w_y)$ de manière à obtenir $\boldsymbol{q}_{\mathrm{eq}\psi}$ dans \eqref{eq:qpsi}  
        \State Détermine la perturbation tournée $\boldsymbol{w}_{\text{r}}$ avec la composante $y$ nulle, en utilisant $\boldsymbol{R}_{\psi}:= \smallmat{ \cos \psi & \sin \psi & 0 \\ -\sin \psi & \cos \psi & 0 \\ 0 & 0 & 1 }$, selon
        \begin{align}
        \label{eq:wh}
        \boldsymbol{w}_{\mathrm{r,eq}} := \smallmat{w_{\text{r}x} \\ 0 \\w_{\text{r}z}} :=  \boldsymbol{R}^\top(\boldsymbol{q}_{\mathrm{eq}\psi}) \boldsymbol{w}_{\mathrm{eq}} = \boldsymbol{R}^\top_{\psi} \boldsymbol{w}_{\mathrm{eq}}
        \end{align}

        \State Détermine l'angle d'inclinaison $\theta$ de manière à obtenir $\boldsymbol{q}_{\text{eq}\theta}$ dans \eqref{eq:qeq}:  
        \begin{align}
        \label{eq:theta_alg}
            \theta = -\tan^{-1}\left(\frac{w_{\text{r}z}}{w_{\text{r}x}} + \frac{2mg}{\rho S \lVert \boldsymbol{w}_{\mathrm{eq}} \rVert C_{\ell}  (1-\frac{\xi_{\text{f}}}{\xi_{\text{m}}}) w_{\text{r}x} } \right)
        \end{align}
    \State Pour des raisons de commodité, nous définissons les scalaires 
        $$ 
        \left[\begin{array}{c|c} 
        \!\!a\!\!&\!\!b\!\! \\ \hline \!\!c\!\!&\!\! d\!\!\end{array}  \right] \!:=\! 
        \left[\begin{array}{c|c}
        2 S_{\text{wet}} C_{\ell} mg \sin{\theta} \xi_{\text{f}} &
        \! 2 S_{\text{wet}} C_{\text{d}} C_{\ell} \rho  \lVert \boldsymbol{w}_{\mathrm{eq}} \rVert  w_{x}^{\text{b}}\!\! \\ \hline
        \!\!-4 S S_{\text{p}} C_{\ell} \rho  \lVert \boldsymbol{w}_{\mathrm{eq}} \rVert  w_{x}^{\text{b}} \xi_{\text{f}}\!\! & \frac{b \xi_{\text{f}}}{2}
        \end{array}\right]
        $$ 
        et grâce à ces scalaires $(a,b,c,d)$, déterminons la traction des hélices $\tau$ dans \eqref{eq:ueq} comme
        \begin{align}
            \nonumber
            \tau &= \frac{S_{\text{p}}}{2 S_{\text{wet}} C_{\ell} \xi_{\text{f}} (4S_{\text{p}} -  S_{\text{wet}} C_{\text{d}} )} \Bigg( a+b+c+d + \Bigg[ (a+b+c \shortminus d)^2 \shortminus 4 (d^2+ac \shortminus bd)  \\ 
            & \quad
            \shortminus \frac{4 {w_{z}^{\text{b}}}^2 d}{ {w_{x}^{\text{b}}}^2 } (d+c) + \frac{4 w_{z}^{\text{b}}ad\cos{\theta}}{w_{x}^{\text{b}} C_{\ell} \sin{\theta} } \left(C_{\text{d}} - \frac{4 S_{\text{p}}}{S_{\text{wet}}}\right) \Bigg] ^{\frac{1}{2}} \Bigg),\label{eq:tau_alg}
        \end{align}
        où
        $$
        \bigmat{w_{x}^{\text{b}} \\ w_{z}^{\text{b}}} = \bigmat{   w_{\text{r}x} \cos{\theta} - w_{\text{r}z} \sin{\theta}\\
                    w_{\text{r}x} \sin{\theta} +  w_{\text{r}z} \cos{\theta} }.
        $$
        
        \State Déterminons la déflexion des élevons $\delta$ comme
        \begin{align}
        \label{eq:delta_alg}
            \delta = \frac{2mg\sin{\theta}}{\rho S \lVert \boldsymbol{w}_{\mathrm{eq}} \rVert C_{\text{d}}\xi_{\text{f}} w_{z}^{\text{b}}} + \frac{w_{x}^{\text{b}}}{\xi_{\text{f}}w_{z}^{\text{b}}} -  \frac{(4-\frac{S_{\text{wet}}}{S_{\text{p}}} C_{\text{d}})}{\rho S \lVert \boldsymbol{w}_{\mathrm{eq}} \rVert C_{\text{d}}\xi_{\text{f}} w_{z}^{\text{b}}} \tau.
        \end{align}

    \end{algorithmic}
    \hspace*{.1cm} \textbf{Retourne}:  $\psi$, $\theta$, $\tau$, $\delta$
    \end{algorithm}


    \begin{theorem}\label{thm:eqs}
    Pour tout vent constant, $\boldsymbol{w} =\smallmat{w_x & w_y & w_z}^\top \in \real^3$ ayant une composante horizontale non nulle $\smallmat{w_x \\ w_y}$,
    les equations \eqref{eq:qpsi}--\eqref{eq:qtheta} avec $\theta$, $\tau$ et $\delta$ sélectionné selon l'Algorithme~\ref{alg:eq} caractérisent un couple d'équilibre $(\boldsymbol{u}_{\text{eq}}, \boldsymbol{x}_{\text{eq}})$ pour la dynamique non linéaire \eqref{eq:dyna_orig} et \eqref{eq:dyna_simp}.
    
    \end{theorem}
    
    \begin{proof}
        Dans un premier temps, notons qu'avec l'expression de $\boldsymbol{R}$ \eqref{eq:matrix_rot} et l'expression de  $\psi$ dans l'étape 1 de l'Algorithme~\ref{alg:eq}, on peut définir la perturbation à l'équilibre tournée $\boldsymbol{w}_{\mathrm{r,eq}} := \boldsymbol{R}^\top_{\psi} \boldsymbol{w}_{\mathrm{eq}} :=    \boldsymbol{R}^\top(\boldsymbol{q}_{\mathrm{eq}\psi})\boldsymbol{w}_{\mathrm{eq}}$ (voir \eqref{eq:wh} dans l'Algorithme~\ref{alg:eq}),
        qui correpond à la rotation necessaire pour aligner l'axe  $x_{[\text{b}]}$ du repère corps avec la direction du vent. Une fois que le drone est face au vent, il subit un vent avec une composante latérale $y$ nulle et il peut ajuster son angle d'inclinaison $\theta$ afin de générer la poussée et la portance nécessaires pour compenser les effets du vent dans les directions longitudinale et verticale (l'effet latéral est nul en raison de l'orientation spécifique de l'appareil $\psi$). Avec cette rotation $\psi$, il est possible d'exprimer le vent dans le repere corps comme étant
            \begin{align}
            \label{eq:wb}
                \boldsymbol{w}^{\text{b}}_{\mathrm{eq}} &:= 
                \begin{bmatrix}
                    w_{x}^{\text{b}} \\ 0 \\ w_{z}^{\text{b}}
                \end{bmatrix} \!=\! 
                \boldsymbol{R}^\top(\boldsymbol{q}_{\text{eq}\theta}) \boldsymbol{w}_{\mathrm{r,eq}}  \\
                &=\!\! \begin{bmatrix}
                    \cos{\theta} & 0 & -\sin{\theta}\\
                        0 & 1 & 0\\
                    \sin{\theta} & 0 & \cos{\theta}
                \end{bmatrix}^\top \!\! \begin{bmatrix}
                    w_{\text{r}x}\\
                    0\\
                    w_{\text{r}z}
                \end{bmatrix}
                \!\!=\!\!\begin{bmatrix}
                    w_{\text{r}x} \cos{\theta} - w_{\text{r}z} \sin{\theta}\\
                    0\\
                    w_{\text{r}x} \sin{\theta} +  w_{\text{r}z} \cos{\theta}
                \end{bmatrix}
            \nonumber
            \end{align}
        Nous insistons sur le fait que $w_{x}^{\text{b}}$ est toujours négatif et différent de zéro, car le drone est orienté dans la direction du vent grâce à la rotation engendré par $ \boldsymbol{q}_{\mathrm{eq}\psi}$, et suite à l'hypothèse $\smallmat{w_x \\ w_y} \neq 0$.
    
        L'équation \eqref{eq:dyna1} montre qu'il est nécessaire d'avoir $\boldsymbol{v}_{\text{eq}} = 0$ pour maintenir l'équilibre stationnaire. En multipliant \eqref{eq:dyna2} par $\boldsymbol{R}(\boldsymbol{q}_{\text{eq}})$ donné dans  \eqref{eq:wb}, nous l'exprimons dans le repère corps.
        Comme nous appliquons la même commande $\tau_{1} = \tau_{2} = \tau $ aux deux moteurs et la même commande au deux élevons $\delta_{1} = \delta_{2} = \delta$, nous obtenons pour les deux modèles \eqref{eq:dyna_orig} et \eqref{eq:dyna_simp}, l'équilibre des forces selon l'axe $x_{[\text{b}]}$ donné par
        \begin{align}
            & (2-\frac{S_{\text{wet}}}{2S_{\text{p}}} C_{\text{d}})\tau - \frac{1}{2}\rho S \lVert \boldsymbol{w}_{\mathrm{eq}} \rVert C_{\text{d}} \left(w_{x}^{\text{b}} - \xi_{\text{f}} \delta w_{z}^{\text{b}} \right) - mg \sin(\theta) = 0 \label{eq:forcex}
        \end{align}
        et l'équilibre des forces selon l'axe $z_{[\text{b}]}$ donné par
        \begin{align}\label{eq:forcez}
            - \frac{S_{\text{wet}}}{2S_{\text{p}}}\xi_{\text{f}} C_{\ell} \tau \delta - \frac{1}{2}\rho S \lVert \boldsymbol{w}_{\mathrm{eq}} \rVert C_{\ell} \left(w_{z}^{\text{b}} + \xi_{\text{f}} \delta w_{x}^{\text{b}} \right) + mg \cos(\theta) = 0
        \end{align}
        De manière similaire, à partir de \eqref{eq:dyna_orig_b} et \eqref{eq:dyna4}, l'équilibre des moments autour de l'axe $y_{[\text{b}]}$ permet d'obtenir
        \begin{align}\label{eq:momenty}
            \frac{S_{\text{wet}}}{2S_{\text{p}}}  \Delta_{\text{r}} \xi_{\text{m}} C_{\ell} \tau \delta + \frac{1}{2}\rho S \Delta_{\text{r}} \lVert \boldsymbol{w}_{\mathrm{eq}} \rVert C_{\ell} \left(w_{z}^{\text{b}} + \xi_{\text{m}} \delta w_{x}^{\text{b}} \right) = 0.
        \end{align}
        
        Pour calculer la solution du triplet ($\theta$,$\tau$,$\delta$) des trois équations d'équilibre \eqref{eq:forcex}--\eqref{eq:momenty}, ajoutons \eqref{eq:forcez} multiplié par $\Delta_{\text{r}} \xi_{\text{m}}$, à \eqref{eq:momenty} multiplié par $\xi_{\text{f}}$, de manière à annuler le premier terme et à obtenir
        \begin{multline*}
            \Delta_{\text{r}} \xi_{\text{m}} \left( - \frac{1}{2}\rho S \lVert \boldsymbol{w}_{\mathrm{eq}} \rVert C_{\ell} (w_{z}^{\text{b}} + \xi_{\text{f}} \delta w_{x}^{\text{b}}) + mg \cos(\theta) \right) \\+ \xi_{\text{f}} \left(\frac{1}{2}\rho S  \Delta_{\text{r}} \lVert \boldsymbol{w}_{\mathrm{eq}} \rVert C_{\ell} (w_{z}^{\text{b}} + \xi_{\text{m}} \delta w_{x}^{\text{b}})  \right) = 0,
        \end{multline*}
        qui est équivalent à
        \begin{align*}
            \frac{1}{2}\rho S  \Delta_{\text{r}} \lVert \boldsymbol{w}_{\mathrm{eq}} \rVert C_{\ell}  (\xi_{\text{f}} - \xi_{\text{m}}) w_{z}^{\text{b}} +   \Delta_{\text{r}} \xi_{\text{m}} mg \cos(\theta)  = 0,
        \end{align*}
        où ($w_{x}^{\text{b}}$,$w_{z}^{\text{b}}$) sont les première et troisième composantes de $\boldsymbol{w}^{\text{b}}$ dans \eqref{eq:wb}. Ensuite, en utilisant \eqref{eq:wb} et en réarrangeant, nous obtenons
        \begin{multline*}
                -\frac{1}{2}\rho S  \Delta_{\text{r}} \lVert \boldsymbol{w}_{\mathrm{eq}} \rVert C_{\ell}  (\xi_{\text{f}} - \xi_{\text{m}}) w_{\text{r}x} \sin{\theta}  +\bigg( -\frac{1}{2}\rho S  \Delta_{\text{r}} \\ \lVert \boldsymbol{w}_{\mathrm{eq}} \rVert C_{\ell}  (\xi_{\text{f}} - \xi_{\text{m}})w_{\text{r}z} +   \Delta_{\text{r}} \xi_{\text{m}} mg \bigg) \cos{\theta} = 0,
        \end{multline*}
        qui est satisfaite par
            \begin{align} \label{eq:theta}
                \theta &=  -\tan^{-1}\left(\frac{\rho S \lVert \boldsymbol{w}_{\mathrm{eq}} \rVert C_{\ell}  (\xi_{\text{f}} - \xi_{\text{m}})w_{\text{r}z} - 2 \xi_{\text{m}} mg }{\rho S\lVert \boldsymbol{w}_{\mathrm{eq}} \rVert C_{\ell}  (\xi_{\text{f}} - \xi_{\text{m}}) w_{\text{r}x}}\right).
            \end{align}
            Cette dernière expression coïncide avec la sélection \eqref{eq:theta_alg} dans l'Algorithme~\ref{alg:eq} après quelques manipulations.
        À partir de \eqref{eq:theta_alg}, nous pouvons calculer les commandes à l'équilibre en substituant \eqref{eq:forcex} dans \eqref{eq:forcez}. Après quelques simplifications, la force nécessaire de traction des hélices  $\tau$ pour maintenir la position d'équilibre corresponds à l'expression  \eqref{eq:tau_alg}. Finalement, avec la valeur de $\tau$ dans \eqref{eq:tau_alg}, nous pouvons obtenir la déflexion des élevons nécessaire $\delta$ à partir de l'équation \eqref{eq:forcez}, ce qui nous donne la valeur obtenue dans \eqref{eq:delta_alg}.
    \end{proof}
    Il est intéressant de noter que pour chaque couple de vent($w_{\text{r}z}$, $w_{\text{r}x}$) correspond une orientation d'équilibre \eqref{eq:qeq}, \eqref{eq:theta_alg} est indépendante de l'entrée $\boldsymbol{u}_{\text{eq}}$. En outre, il convient de souligner que pour toutes les valeurs de vent raisonnables, l'équation \eqref{eq:tau_alg} correspond à la racine positive d'un polynôme du second ordre, l'autre racine étant toujours négative, ce qui conduit à une condition de poussée négative physiquement impossible.


    À partir de l'expression analytique \eqref{eq:equilibrium} de l'équilibre du drone pour différentes conditions de vent $\boldsymbol{w}$, nous reportons, sur la Fig. \ref{fig:saturation}, les valeurs correspondantes de $\theta$, $\delta$, $\tau$ pour des valeurs de vitesse de vent horizontale allant de 0 à \SI{-20}{\meter\per\second} et pour des valeurs de vitesse de vent verticale allant de \SI{-6}{} à \SI{6}{\meter\per\second}. L'angle d'incidence $\theta$ diminue de \SI{90}{\degree} à \SI{-4.65}{\degree}. $\theta = \SI{90}{\degree}$ correspond à un vol stationnaire sans vent. La traction $\tau$ atteint son minimum à $w_{rx} = \SI{-12.8}{\meter\per\second}$, ce qui correspond à une condition de vol qui minimise la consommation d'énergie, car les moteurs sont la principale source de consommation électrique.
        
    \todo{traduction caption}
    \begin{figure}[ht!]
        \centering
        \includegraphics[trim=0cm 0cm 0cm 0cm,clip,width=0.5\columnwidth]{figures/equilibrium_wx_wz.png}
        \caption{Parameters ($\theta$, $\delta$, $\tau$) of the equilibrium point (surface) established in Theorem~\ref{thm:eqs} and Algorithm \ref{alg:eq} for constant horizontal and vertical wind ($w_{\text{r}x}$,$w_{\text{r}z}$), and actuators saturation levels (red).}
        \label{fig:saturation}
    \end{figure}
    Il est possible de faire une coupe des surfaces présentée dans \eqref{fig:saturation} pour une vitesse verticale nulle $\boldsymbol{w}_{\text{rx}} = 0$, ce qui nous donne le résultat de la Figure \ref{fig:saturation_wz0}
    \begin{figure}[ht!]
        \centering
        \includegraphics[trim=0cm 0cm 0cm 0cm,clip,width=0.5\columnwidth]{figures/equilibrium.png}
        \caption{Parameters ($\theta$, $\delta$, $\tau$) of the equilibrium point (blue) established in Theorem~\ref{thm:eqs} and Algorithm \ref{alg:eq} for a constant horizontal wind $w_{\text{r}x}$, and actuators saturation levels (red).}
        \label{fig:saturation_wz0}
    \end{figure}

\section{Dynamiques linéarisés}
\subsection{Dynamique linéarisé sans vent}
Considérons le cas sans vent discuté dans la section \ref{sec:eq_nowind} pour lequel nous utilisons le vecteur de commande $\boldsymbol{u}_{\text{nowind}}$  et le vecteur de commande à l'équilibre $\boldsymbol{u}_{\text{nowind},\text{eq}}$ défini dans l'équation \eqref{eq:bar_u} et rapellons la transformation du vecteur de comamande suivante $ \boldsymbol{u}_{\text{nowind}} := \begin{bmatrix}\tau_{1}  \!&\! \tau_{2}  \!&\! \delta_{1}\tau_{1} \!&\! \delta_{2}\tau_{2} \end{bmatrix}^\top$, la dynamqiue linearisé dans le cas sans vent est
\begin{align}
    \label{eq:linearized}
     \boldsymbol{\dot{\tilde{x}}} = \boldsymbol{A}_{0} \tilde{\boldsymbol{x}} + \boldsymbol{G}_{0} (\boldsymbol{u}_{\text{nowind}}-\boldsymbol{u}_{\text{nowind},\text{eq}}),
\end{align}
where the expression of $\boldsymbol{A}_{0}$ is 
\begin{align}
    \label{matrice_A}
        \boldsymbol{A}_{0} = \boldsymbol{A}_{w} \Big|_{\boldsymbol{w}=0} =\begin{bmatrix}
        \mathbb{0}_{3} & \mathbb{I}_{3} & \mathbb{0}_{3} & \mathbb{0}_{3} \\
        \mathbb{0}_{3} & \mathbb{0}_{3} &  \boldsymbol{A}_{v\epsilon} & \mathbb{0}_{3} \\
        \mathbb{0}_{3} & \mathbb{0}_{3} & \mathbb{0}_{3} & \boldsymbol{A}_{\epsilon\omega} \\
        \mathbb{0}_{3} & \mathbb{0}_{3} & \mathbb{0}_{3} & \mathbb{0}_{3}
        \end{bmatrix},
\end{align}
with the following selections
\begin{align*}
    \boldsymbol{A}_{\epsilon\omega} = \frac{\sqrt{2}}{4}\begin{bmatrix} 
        1 & 0 & -1 \\ 
        0 & 1 & 0  \\
       1 & 0 & 1
    \end{bmatrix} \text{ and } \boldsymbol{A}_{ v\epsilon} = \sqrt{2}\begin{bmatrix} 
        0 & -2g & 0\\
        g & 0 & g  \\ 
         0 & -2g & 0 \end{bmatrix},
\end{align*}
while the expression of $\boldsymbol{G}_{0}$ is 
\begin{align*}
    \boldsymbol{G}_{0}  := \begin{bmatrix}
    \mathbb{0}_{3\times 1} & \mathbb{0}_{3\times 1} & \mathbb{0}_{3\times 1} & \mathbb{0}_{3\times 1}\\
    0 & 0 & a_{\text{g}} & a_{\text{g}}\\
    0 & 0 & 0 & 0\\
    b_{\text{g}} & b_{\text{g}}  & 0 & 0\\
    \mathbb{0}_{3\times 1} & \mathbb{0}_{3\times 1} & \mathbb{0}_{3\times 1} & \mathbb{0}_{3\times 1}\\
    c_{\text{g}} & -c_{\text{g}} & d_{\text{g}} & -d_{\text{g}}\\
    0 & 0 & e_{\text{g}} & e_{\text{g}}\\
    f_{\text{g}} & -f_{\text{g}} & 0 & 0\\
    \end{bmatrix} , 
\end{align*}
with
\begin{align*}
    \left[\!\! \begin{array}{c|c} 
    a_{\text{g}} & b_{\text{g}} \\ \hline
    c_{\text{g}} & d_{\text{g}} \\ \hline
    e_{\text{g}} & f_{\text{g}}
    \end{array} \!\!\right] \!=\!
    \left[\begin{array}{c|c}
   \shortminus \frac{S_{\text{wet}}}{4mS_{\text{p}}}C_{\ell}\xi_{\text{f}}  & \frac{1}{m}( 1-\frac{S_{\text{wet}}}{2S_{\text{p}}} C_{\text{d}}) \\ \hline
    \frac{k_{\text{m}} }{J_{x}k_{\text{f}}}  &   \! \frac{ S_{\text{wet}} a_{y} }{4 J_{x} S_{\text{p}} } C_{\ell}\xi_{\text{f}} \!\\ \hline
    \!\! \frac{S_{\text{wet}}\Delta_{\text{r}}}{4J_{y}S_{\text{p}}} C_{\ell}\xi_{\text{m}} \!\! & 
    \frac{1}{J_{z}}(p_{y}\!+\!\frac{S_{\text{wet}}}{4S_{\text{p}}} a_{y} C_{\text{d}})\!\!
    \end{array}\right].
\end{align*}
We emphasize that these equations coincide with the expressions given in our preliminary work \cite[eqn. (22)]{sansou:ECC}.



\subsection{Dynamique linéarisé en présence de vent}

For each one of the equilibria characterized in Theorem~\ref{thm:eqs}, we derive here some linearized equations of motion with respect to the simplified nonlinear low-speed model \eqref{eq:dyna_simp}. A direct approach would lead to linearized equations that depend on the $\psi$ angle characterized in step 1 of Algorithm~\ref{alg:eq}.
Instead, we define here the incremental coordinates in a suitably rotated inertial reference frame, so that the linearized dynamics is independent of the $\psi$ angle .
More specifically, for each equilibrium wind condition $\boldsymbol{w}_{\text{eq}}$ and the ensuing equilibrium
$(\boldsymbol{u}_{\text{eq}}, \boldsymbol{p}_{\text{eq}},
\boldsymbol{q}_{\text{eq}})$ characterized in \eqref{eq:qpsi}--\eqref{eq:qtheta}, 
denoting the scalar and vector components of the quaternion in \eqref{eq:qeq} as
$\boldsymbol{q}_{\text{eq}} = (\eta_{\text{eq}}, \boldsymbol{\epsilon}_{\text{eq}})$, and
based on the rotation matrix
$\boldsymbol{R}_{\psi} :=    \boldsymbol{R}(\boldsymbol{q}_{\mathrm{eq}\psi})$ introduced at the beginning of the proof of Theorem~\ref{thm:eqs}, we study here the approximate linear dynamics of the rotated incremental input-state vector:  
\begin{align}
\label{eq:xtilde}
     \nonumber &\tilde{\boldsymbol{x}} := (\tilde{\boldsymbol{p}},
     \tilde{\boldsymbol{v}},
     \tilde{\boldsymbol{\epsilon}},
     \tilde{\boldsymbol{\omega}}_{\text{b}}) = \left(\boldsymbol{R}^\top_{\psi} (\boldsymbol{p} \! \shortminus \! \boldsymbol{p}_{\text{eq}}), \boldsymbol{R}^\top_{\psi} \boldsymbol{v}, \boldsymbol{R}^\top_{\psi} (\boldsymbol{\epsilon} \! \shortminus \! \boldsymbol{\epsilon}_{\text{eq}}), \boldsymbol{\omega}_{\text{b}} \right), \\ &\tilde{\boldsymbol{u}} := \boldsymbol{u}-\boldsymbol{u_{\text{eq}}}, \quad \tilde{\boldsymbol{w}} := \boldsymbol{R}^\top_{\psi} (\boldsymbol{w}-\boldsymbol{w}_{\mathrm{eq}}).
\end{align}
Note that the rotation in \eqref{eq:xtilde} enjoys the useful property that $\boldsymbol{R}^\top_{\psi} \boldsymbol{\epsilon}_{\text{eq}} = \smallmat{0 & \sin(\frac{\theta}{2}) & 0}^\top$, a fact that greatly simplifies the linearized motion.


% the constants $\boldsymbol{p}_{\text{eq}} \in \real^3$, $\boldsymbol{u}_{\text{eq}} \in \real^4$ and $\boldsymbol{\epsilon}_{\text{eq}} \in \real^3$ come from any of the equilibrium pairs characterized in Theorem~\ref{thm:eqs} (see equation \eqref{eq:equilibrium})) and $\boldsymbol{u}$ in equation \eqref{eq:vector_u}).


Exploiting the fact that the translational and rotational speeds ($\boldsymbol{v}_{\text{eq}}$, $\boldsymbol{\omega}_{\text{b,eq}})$ must be zero at the equilibrium (see \eqref{eq:equilibrium}), we prove below that the approximate linearized dynamics for 
the state in \eqref{eq:xtilde} is given by
% about $\boldsymbol{x}_{\text{eq}}$ is given by
%
\begin{align}
\label{eq:lpv_linearisation}
 \boldsymbol{\dot{\tilde{x}}} &= \boldsymbol{A}_{w} \tilde{\boldsymbol{x}} + \boldsymbol{G}_{w} \tilde{\boldsymbol{u}} + \boldsymbol{E}_{w}  \tilde{\boldsymbol{w}} \\
 &= \smallmat{
     \mathbb{0}_{3} & \mathbb{I}_{3} & \mathbb{0}_{3} &\mathbb{0}_{3} \\
    \mathbb{0}_{3} & \boldsymbol{A}_{vv}  & \boldsymbol{A}_{v\epsilon}  & \mathbb{0}_{3}\\
    \mathbb{0}_{3} & \mathbb{0}_{3} & \mathbb{0}_{3} & \boldsymbol{A}_{\epsilon \omega} \\    
    \mathbb{0}_{3} & \mathbb{0}_{3} &  \boldsymbol{A}_{\omega \epsilon} & \mathbb{0}_{3}
    } \tilde{\boldsymbol{x}} \!+\!
    \smallmat{ \mathbb{0}_{3 \times 4} \\
     \boldsymbol{G}_{v}\\
     \mathbb{0}_{3 \times 4}\\
     \boldsymbol{G}_{\omega}
    } \tilde{\boldsymbol{u}}
    \!+\! \smallmat{
     \mathbb{0}_{3 \times 3} \\
     \boldsymbol{E}_{v} \\
     \mathbb{0}_{3 \times 3} \\
     \boldsymbol{E}_{\omega} 
     } \!\tilde{\boldsymbol{w}},
     \nonumber
\end{align}
%
with matrices $\boldsymbol{A}_{vv} $, $\boldsymbol{A}_{v\epsilon}$, $\boldsymbol{A}_{\epsilon \omega_{\text{b}}}$, $\boldsymbol{A}_{\omega \epsilon}$, $ \boldsymbol{G}_{v}$, $\boldsymbol{G}_{\omega}$ $\boldsymbol{E}_{v}$, $\boldsymbol{E}_{\omega}$ constructed by following Algorithm~\ref{alg:linea}.

\begin{theorem} \label{th:lin}
For any constant wind, $\boldsymbol{w} =\smallmat{w_x & w_y & w_z}^\top \in \real^3$ having a nonzero horizontal component $\smallmat{w_x \\ w_y}$, and the ensuing equilibrium pair $(\boldsymbol{u}_{\text{eq}}, \boldsymbol{x}_{\text{eq}})$ of dynamics \eqref{eq:dyna_simp}, as
characterized in \eqref{eq:qpsi}-\eqref{eq:qtheta}, the linearized dynamics of 
the incremental input-state vector \eqref{eq:xtilde}
% \eqref{eq:dyna_simp_rot}) about $\boldsymbol{x_{\text{eq}}}$
is given by \eqref{eq:lpv_linearisation} with the matrices constructed as in Algorithm~\ref{alg:linea}.
\end{theorem}
%
\begin{proof}
To begin with, by exploiting the rotation matrix 
$\boldsymbol{R}_{\psi} :=    \boldsymbol{R}(\boldsymbol{q}_{\mathrm{eq}\psi})$ used in \eqref{eq:xtilde}, we transform the nonlinear dynamics \eqref{eq:dyna_simp} into rotated coordinates
\begin{align}
\label{eq:rotated_coord}
(\boldsymbol{p}_{\text{r}} ,
\boldsymbol{v}_{\text{r}} ,
\boldsymbol{q}_{\text{r}}
)
:=
\left(\boldsymbol{R}_{\psi}^\top
\boldsymbol{p},
\boldsymbol{R}_{\psi}^\top \boldsymbol{v},
\boldsymbol{q}_{\mathrm{eq}\psi}^{-1} \otimes
\boldsymbol{q}
\right), 
\; \boldsymbol{w}_{\text{r}}:=\boldsymbol{R}_{\psi}^\top\boldsymbol{w}
\end{align}
while $\boldsymbol{\omega}_{\text{b}}$ remains unchanged because it is expressed in the body frame. 
A few observations allow simplifying
the transformed dynamics  \eqref{eq:dyna_simp}: (i)
first, we have $\boldsymbol{R}_{\psi}^\top
m\boldsymbol{g} = m\boldsymbol{g}$ because
the rotation of $\psi$ is about the $z$-axis; (ii) secondly, 
since $\boldsymbol{q}_{\text{r}} = \boldsymbol{q}_{\mathrm{eq}\psi}^{-1} \otimes
\boldsymbol{q}$, then $\boldsymbol{R}_{\psi}^\top \boldsymbol{R}(\boldsymbol{q}) = \boldsymbol{R}(\boldsymbol{q}_{\text{r}})$; 
(iii) since $\boldsymbol{v}_{\text{b}} := \boldsymbol{R}^\top(\boldsymbol{q}) (\boldsymbol{v}-\boldsymbol{w})$ (as defined after equation \eqref{eq:Mb}, then $\| \boldsymbol{v}_{\text{b}} \| = 
\| \boldsymbol{v} -  \boldsymbol{w}  \| - \| \boldsymbol{v}_{\text{r}} -  \boldsymbol{w}_{\text{r}}  \|$
(iv) finally, $\boldsymbol{R}^\top(\boldsymbol{q}) \boldsymbol{w}
= \boldsymbol{R}^\top\! (\boldsymbol{q}_{\text{r}}) \boldsymbol{R}_{\psi}^\top\! \boldsymbol{R}_{\psi} \boldsymbol{w}_{\text{r}}= \boldsymbol{R}^\top \!(\boldsymbol{q}_{\text{r}}) \boldsymbol{w}_{\text{r}}$.
Based on the observations above, we can derive the rotated version of equations \eqref{eq:dyna_simp} as
\begin{subequations}\label{eq:dyna_simp_rot}
    \begin{alignat}{3}
        \boldsymbol{\dot p}_{\text{r}} &=  \boldsymbol{v}_{\text{r}}, \label{eq:p_r} \\
        m \boldsymbol{\dot v}_{\mathrm{r}} &= \!\shortminus m\boldsymbol{g} \!+ \! \boldsymbol{R}(\boldsymbol{q}_{\mathrm{r}})\left(\boldsymbol{M}_{\text{f}}(\boldsymbol{u}) \! + \! \boldsymbol{D}_{\text{f}}(\boldsymbol{u}) \lVert \boldsymbol{w}_{\text{r}} \rVert \boldsymbol{R}^\top \!(\boldsymbol{q}_{\mathrm{r}}) (\boldsymbol{v}_{\mathrm{r}} \! \shortminus \! \boldsymbol{w}_{\text{r}}) \right),  \label{eq:v_r} \\
       \boldsymbol{\dot{q}}_{\text{r}} &=  \left( \frac{1}{2}\boldsymbol{q}_{\text{r}} \otimes \smallmat{0 \\ \boldsymbol{\omega}_{\text{b}}} \right),  \label{eq:q_r}\\
        \boldsymbol{J} \boldsymbol{\dot \omega}_{\text{b}} &=   \shortminus \skewsym{\boldsymbol{\omega}_{\text{b}}}\boldsymbol{J}\boldsymbol{\omega}_{\text{b}}\! + \boldsymbol{M}_{\text{m}}(\boldsymbol{u}) \! + \! \boldsymbol{D}_{\text{m}} (\boldsymbol{u}) \lVert  \boldsymbol{w}_{\text{r}} \rVert \boldsymbol{R}^\top \!(\boldsymbol{q}_{\mathrm{r}}) (\boldsymbol{v}_{\mathrm{r}} \! \shortminus \! \boldsymbol{w}_{\text{r}})  
        \label{eq:w_r}
    \end{alignat}
\end{subequations}
With these new coordinates, the incremental input-state vectors in \eqref{eq:xtilde} can be expressed as
\begin{align}
\label{eq:xtilde_rot}
     \nonumber &\tilde{\boldsymbol{x}} = \left(
     \boldsymbol{p}_{\text{r}} \! \shortminus \!  \boldsymbol{R}^\top_{\psi}\boldsymbol{p}_{\text{eq}}, \boldsymbol{v}_{\text{r}},  
     \boldsymbol{\epsilon}_{\text{r}} \! \shortminus \! \boldsymbol{R}^\top_{\psi}\boldsymbol{\epsilon}_{\text{eq}}, \boldsymbol{\omega}_{\text{b}} \right), \\ &\tilde{\boldsymbol{u}} := \boldsymbol{u}-\boldsymbol{u_{\text{eq}}}, \quad \tilde{\boldsymbol{w}} :=  \boldsymbol{w}_{\mathrm{r}}-\boldsymbol{w}_{\mathrm{r,eq}}
\end{align}
where $\boldsymbol{w}_{\mathrm{r,eq}} = \boldsymbol{R}^\top_{\psi}\boldsymbol{w}_{\mathrm{eq}} = \smallmat{w_{\text{r}x} \\ 0 \\w_{\text{r}z}}$, already defined in
 \eqref{eq:wh}, and
  $\boldsymbol{R}^\top_{\psi} \boldsymbol{\epsilon}_{\text{eq}} = \smallmat{0 & \sin(\frac{\theta}{2}) & 0}^\top$
 have both a convenient sparse structure.

% The result is $ \boldsymbol{w}_{\text{r}} := \smallmat{w_{\text{r}x} \\ 0 \\w_{\text{r}z}} :=  \boldsymbol{R}^\top_{\psi} \boldsymbol{w}$  as in \eqref{eq:wh}. Consider a constant selection $\boldsymbol{w}_{\mathrm{eq}} = \boldsymbol{R}^\top(\boldsymbol{q}) \boldsymbol{w} $ in the nonlinear dynamics \eqref{eq:dyna_simp_rot}) and the corresponding induced equilibrium $\boldsymbol{x}_{\text{eq}}$ with the constant input $\boldsymbol{u}_{\text{eq}}$, both characterized in \eqref{eq:equilibrium}. 


By focusing on the rotated dynamics \eqref{eq:dyna_simp_rot} and the expression \eqref{eq:xtilde_rot} of the incremental variables, the proof of the theorem amounts to showing that the linearization of \eqref{eq:dyna_simp_rot} about
the rotated equilibrium
\begin{align}
\label{eq:rotated_eq}
\boldsymbol{x}_{\text{r,eq}} &= \left( \boldsymbol{p}_{\text{r,eq}}, \boldsymbol{v}_{\text{r,eq}},
\boldsymbol{\epsilon}_{\text{r,eq}},
\boldsymbol{\omega}_{\text{br,eq}} \right) \\
\nonumber
&= \left(\boldsymbol{R}^\top_{\psi} \boldsymbol{p}_{\text{eq}},  
\smallmat{0 \\ 0 \\ 0},   \smallmat{0 \\ \sin(\frac{\theta}{2}) \\ 0}, 
\smallmat{0 \\ 0 \\ 0} \right),\;
\boldsymbol{w}_{\mathrm{r,eq}}  = \smallmat{w_{\text{r}x} \\ 0 \\w_{\text{r}z}}
\end{align}
coincides with equation \eqref{eq:lpv_linearisation} and the expressions in Algorithm~\ref{alg:linea}.
%
To this end, inspired by \cite[Proof of Lemma 1]{tregouet:hal-01760720}, to linearize the dynamics of the quaternion $\boldsymbol{q}_{\text{r}} = \left [ \eta_{\text{r}} ~ \boldsymbol{\epsilon}_{\text{r}}^\top \right]^\top$ evolving in ${\mathbb S}^3$, we replace  $\eta_{\text{r}}$ by its positive value induced by the unit norm of the quaternion. Thus, $\eta_{\text{r}} = (1- \boldsymbol{\epsilon}_{\text{r}}^\top \boldsymbol{\epsilon}_{\text{r}})^\frac{1}{2}$. 

Let us now first focus on matrix $\boldsymbol{A}_w$ in \eqref{eq:lpv_linearisation}. Its first block row is evidently $\smallmat{\mathbb{0}_{3} & \mathbb{I}_{3} & \mathbb{0}_{3} &\mathbb{0}_{3}}$, due to linearity of equation \eqref{eq:p_r}. 
For its second block row we focus on equation \eqref{eq:v_r}
and start by characterizing $\boldsymbol{R}(\boldsymbol{q}_{\mathrm{r,eq}})$, whose structure is relatively simple due to the sparsity of $\boldsymbol{\epsilon}_{\text{r,eq}}$. In particular,
we recall from \eqref{eq:wb} that, using the expression of $\boldsymbol{R}$ in \eqref{eq:matrix_rot}, we may write 
\begin{align*}
    \boldsymbol{R}(\boldsymbol{q}_{\mathrm{r,eq}\textbf{}})= \boldsymbol{R}_\theta :=
     \begin{bmatrix}1-2\overline \epsilon_{2}^{2} & 0 & 2\overline\epsilon_{2} \overline{\eta} \\ 0 & 1 & 0 \\ -2\overline\epsilon_{2} \overline{\eta} & 0 & 1-2\overline\epsilon_{2}^{2} \end{bmatrix}
    = \smallmat{ \cos \theta & 0 & \sin \theta \\ 0 & 1 & 0 \\ -\sin \theta & 0 & \cos \theta },
\end{align*}
where $\overline \epsilon_{2} = \sin{\frac{\theta}{2}}$ denotes the second element of $\boldsymbol{\epsilon}_{\text{r,eq}}$ 
%(or the third element of $\boldsymbol{q}_{\text{eq}}$) 
as per \eqref{eq:rotated_eq} and $ \overline{\eta} = \sqrt{1-\overline \epsilon_{2}^{2}} = \cos{\frac{\theta}{2}}$.

With this expression of $ \boldsymbol{R}_\theta$, we may derive the following expression from \eqref{eq:v_r}, by using the shortcut notation 
$\left. \cdot \right|_{\text{eq}}$ to characterize 
the evaluation of a (matrix or vector) function at the equilibrium \eqref{eq:rotated_eq},
\begin{align}
\nonumber
\boldsymbol{A}_{vv} &\!=\! \frac{\partial }{\partial \boldsymbol{v}}  \left. \left( \frac{1}{m} \boldsymbol{R}(\boldsymbol{q}_{\text{r}}) \left( %\boldsymbol{M}_{\text{f}}(\boldsymbol{u}) +  
\boldsymbol{D}_{\text{f}}(\boldsymbol{u}) \lVert \boldsymbol{w}_{\text{r}} \rVert \boldsymbol{R}^\top \!(\boldsymbol{q}_{\mathrm{r}}) ( \boldsymbol{v}_{\mathrm{r}} \! \shortminus \! \boldsymbol{w}_{\text{r}})  \right) \right)\right|_{\text{eq}}\\
&\!=\!  \left. \frac{\partial }{\partial \boldsymbol{v}} \left( \frac{1}{m}  \boldsymbol{R}_\theta  \boldsymbol{D}_{\text{f}}(\boldsymbol{u_{\text{eq}}}) \lVert \boldsymbol{w}_{\mathrm{eq}} \rVert   \boldsymbol{R}_\theta^\top  \boldsymbol{v}_{\mathrm{r}} \right) \right|_{\text{eq}},
\label{eq:Avv_derivation}
\end{align} 
which, also considering the identity $\boldsymbol{D}_{\text{f,eq}} = \boldsymbol{D}_{\text{f}}(\boldsymbol{u_{\text{eq}}})$, is easily shown to coincide with matrix 
$\boldsymbol{A}_{vv}$ given in \eqref{eq:Avv_alg2} in Algorithm~\ref{alg:linea}.

We now focus on the entry 
$\boldsymbol{A}_{v\epsilon}$ of 
$\boldsymbol{A}_{w}$, which should be computed, starting from \eqref{eq:v_r} in parallel ways to 
\eqref{eq:Avv_derivation}, as
\begin{align}
\boldsymbol{A}_{v\epsilon} &\!=\! \frac{\partial }{\partial \boldsymbol{\epsilon}}  \left. \left( \frac{1}{m} \boldsymbol{R}(\boldsymbol{q}_{\text{r}}) \left( 
\boldsymbol{M}_{\text{f}}(\boldsymbol{u}) +  
\boldsymbol{D}_{\text{f}}(\boldsymbol{u}) \lVert \boldsymbol{w}_{\text{r}} \rVert \boldsymbol{R}^\top \!(\boldsymbol{q}_{\mathrm{r}}) \boldsymbol{w}_{\text{r}}  \right) \right)\right|_{\text{eq}}.
\label{eq:Aveps_derivation}
\end{align} 
For evaluating the right-hand side of \eqref{eq:Aveps_derivation}, starting from the expression of 
$\boldsymbol{R}(\boldsymbol{q}) = \boldsymbol{R}\left( \smallmat{\eta \\ \epsilon}\right)$ in \eqref{eq:matrix_rot}, after substituting 
$\eta = \sqrt{1-\boldsymbol{\epsilon}^\top \boldsymbol{\epsilon}} \neq 0$
(we recall that for all of the characterized equilibria we have $\eta \neq 0$), we may compute the
generic derivative
\begin{align}
\label{eq:diffR_eps}
&\partial \boldsymbol{R}_{\boldsymbol{ \epsilon}} (\boldsymbol{ \epsilon},\mathfrak{v}) := \frac{\partial }{\partial \boldsymbol{\epsilon}}
\boldsymbol{R}\left(
\smallmat{\sqrt{1-\boldsymbol{\epsilon}^\top \boldsymbol{\epsilon}} \\ \boldsymbol{\epsilon}}
\right) \mathfrak{v}  \\
\nonumber
&\; = 2 \eta \skewsym{\mathfrak{v}} \left( \frac{\boldsymbol{\epsilon}\boldsymbol{\epsilon}^\top}{1-\boldsymbol{\epsilon}^\top \boldsymbol{\epsilon}} - \mathbb{I}_{3}  \right) \shortminus 4 \mathfrak{v} \boldsymbol{\epsilon}^\top \! + \!2\boldsymbol{\epsilon} \mathfrak{v}^\top \!+ \! 2\boldsymbol{\epsilon}^\top \mathfrak{v} \mathbb{I}_{3},
\end{align}
which also implies
\begin{align}
\label{eq:diffRtop}
    \frac{\partial }{\partial \boldsymbol{\epsilon}}
\boldsymbol{R}^\top\left(
\smallmat{\eta \\ \boldsymbol{\epsilon}}
\right) \mathfrak{v} =
    \frac{\partial }{\partial \boldsymbol{\epsilon}}
\boldsymbol{R}\left(
\smallmat{\sqrt{1-\boldsymbol{\epsilon}^\top \boldsymbol{\epsilon}} \\ -\boldsymbol{\epsilon}}
\right) \mathfrak{v} = \partial \boldsymbol{R}_{\boldsymbol{ \epsilon}} (\boldsymbol{ -\epsilon},\mathfrak{v}).
\end{align}
For evaluating \eqref{eq:Aveps_derivation}, it will be useful to derive the following simplified form
\begin{align}
\label{eq:diffRsparse}
    &\partial \boldsymbol{R}_{\boldsymbol{ \epsilon}} \left(
    \smallmat{0 \\ \epsilon_2 \\ 0}
    ,\smallmat{\mathfrak{v}_1 \\ 0 \\ \mathfrak{v}_3}\right) \nonumber \\ 
    & = 2  \smallmat{0 & \left( \overline{\eta} - \frac{\overline \epsilon_{2}^{2}}{\overline{\eta}} \right) \mathfrak{v}_3 & 0\\
    -\overline{\eta} \mathfrak{v}_3 & 0 & \overline{\eta} \mathfrak{v}_1 \\
    0 & \left( \frac{\overline \epsilon_{2}^{2}}{\overline{\eta}} - \overline{\eta} \right) \mathfrak{v}_1 & 0} + 2\overline \epsilon_{2}\smallmat{ 0 & -2\mathfrak{v}_1 & 0\\
    \mathfrak{v}_1 & 0 & \mathfrak{v}_3 \\
    0 & -2\mathfrak{v}_3 & 0}.
\end{align}

We can define two forces $(f_{\text{d}} , f_{\ell})$ acting on the drone at the equilibrium, expressed in the body frame, which depend on the wind $\boldsymbol{w}$ and on the two identical elevon inputs $\delta$. These two forces are the drag and lift generated by the airflow over the wing. They are the result of the development of the expression $ \boldsymbol{D}_{\text{f}}(\boldsymbol{u}) \lVert \boldsymbol{v}_{\text{b}} \rVert \boldsymbol{v}_{\text{b}}$ from \eqref{eq:v_r} with $\boldsymbol{D}_{\text{f}}(\boldsymbol{u})$ as in \eqref{eq:df}:
\begin{align}
\label{eq:draglift}
    \smallmat{f_{\text{d}} \\ 0 \\ f_{\ell}} = - \boldsymbol{D}_{\text{f}}(\boldsymbol{u_{\text{eq}}}) \lVert \boldsymbol{w}_{\mathrm{eq}} \rVert  \boldsymbol{R}_\theta^\top \boldsymbol{w}_{\mathrm{r,eq}},
\end{align}
which, after some calculations, can be shown to coincide with the selections in \eqref{eq:draglift_ALG}, given in Algorithm~\ref{alg:linea}.

From the two forces $(f_{\text{d}} , f_{\ell})$ in \eqref{eq:draglift}, it is possible to determine their partial derivatives with respect to the quaternion component $\overline \epsilon_2$ representing the drone's pitch. Using \eqref{eq:diffRtop}, we obtain
\begin{align}
    \smallmat{\frac{\partial  f_{\text{d}}  }{\partial \epsilon_{2}} \\ 0 \\ \frac{\partial  f_{\ell}  }{\partial \epsilon_{2}}} = - \boldsymbol{D}_{\text{f}}(\boldsymbol{u_{\text{eq}}})  \lVert \boldsymbol{w}_{\mathrm{eq}} \rVert \partial \boldsymbol{R}_{\boldsymbol{ \epsilon}}( -\boldsymbol{ \epsilon}, \boldsymbol{w}_{\mathrm{r,eq}}) \smallmat{0\\ 1\\ 0},
\end{align}
which, after some calculations, 
also considering the identity $\boldsymbol{D}_{\text{f,eq}} = \boldsymbol{D}_{\text{f}}(\boldsymbol{u_{\text{eq}}})$,
can be shown to coincide with the selections in \eqref{eq:draglift_ALG}, given in Algorithm~\ref{alg:linea}.

Following parallel derivations, the force
$f_{\text{m}}$ generated by the motors, linked to the propeller traction and the drag generated by the airflow over the wing, and the force $f_{\text{e}}$ generated by the elevons, linked to the airflow created by the propellers are obtained from \eqref{eq:Mf} as
\begin{align}
\label{eq:motor_el}
    \smallmat{ f_{\text{m}}  \\ 0 \\ f_{\text{e}} } &= \boldsymbol{M}_{\text{f}}(\boldsymbol{u_{\text{eq}}}) ,
\end{align}
which, after some calculations, can be shown to coincide with the selections in \eqref{eq:motor_elevon_forces}, given in Algorithm~\ref{alg:linea}.

Using the definitions \eqref{eq:diffR_eps}, \eqref{eq:diffRtop}, together with the sparse expressions \eqref{eq:diffRsparse}, \eqref{eq:draglift}, \eqref{eq:motor_el}, and their equivalent forms reported in \eqref{eq:draglift_ALG}, \eqref{eq:motor_elevon_forces} given in Algorithm~\ref{alg:linea}, we finally may compute from \eqref{eq:Aveps_derivation}
\begin{multline*}
    \!\boldsymbol{A}_{v\epsilon} \!=\! \frac{1}{m} \big( \partial \boldsymbol{R}_{\boldsymbol{ \epsilon}} (\boldsymbol{ \epsilon} , \boldsymbol{M}_{\text{f}}(\boldsymbol{u_{\text{eq}}}))  \shortminus \partial \boldsymbol{R}_{\boldsymbol{ \epsilon}} ( \boldsymbol{ \epsilon} ,\boldsymbol{D}_{\text{f}}(\boldsymbol{u_{\text{eq}}})  \lVert \boldsymbol{w}_{\mathrm{eq}} \rVert  \boldsymbol{R}_\theta^\top \! \boldsymbol{w}_{\mathrm{eq}})  \\    \left. \shortminus  \boldsymbol{R}_\theta \boldsymbol{D}_{\text{f}}(\boldsymbol{u})  \lVert \boldsymbol{w}_{\mathrm{r,eq}} \rVert \partial \boldsymbol{R}_{\boldsymbol{ \epsilon}}( -\boldsymbol{ \epsilon}, \boldsymbol{w}_{\mathrm{r,eq}})\smallmat{0\\ 1\\ 0} \big)\right|_{\mathrm{eq}}.
\end{multline*}
which provides expression \eqref{eq:Aveps_derivation_ALG} in Algorithm~\ref{alg:linea}
after some straightforward calculations also exploiting  $\boldsymbol{D}_{\text{f,eq}} = \boldsymbol{D}_{\text{f}}(\boldsymbol{u_{\text{eq}}})$.

% \\&= \frac{1}{m} \left ( \partial \boldsymbol{R}_{\boldsymbol{ \epsilon}} (\boldsymbol{ \epsilon} , \boldsymbol{M}_{\text{f}}(\boldsymbol{u_{\text{eq}}})) - \partial \boldsymbol{R}_{\boldsymbol{ \epsilon}} ( \boldsymbol{ \epsilon} ,\boldsymbol{D}_{\text{f}}(\boldsymbol{u_{\text{eq}}})  \lVert \boldsymbol{w}_{\mathrm{eq}} \rVert  \boldsymbol{R}_\theta^\top  \boldsymbol{w}_{\mathrm{eq}})  -  \boldsymbol{R}_\theta \boldsymbol{D}_{\text{f}}(\boldsymbol{u})  \lVert \boldsymbol{w}_{\mathrm{r,eq}} \rVert \partial \boldsymbol{R}_{\boldsymbol{ \epsilon}}( -\boldsymbol{ \epsilon}, \boldsymbol{w}_{\mathrm{r,eq}}) \right) \nonumber

We now focus on the entry $\boldsymbol{A}_{\epsilon \omega}$ of 
$\boldsymbol{A}_{w}$, and we recall that,
due to the properties of the quaternion product (see, e.g., \cite{hamel_minhduc}), 
$\smallmat{\eta \\ \boldsymbol{\epsilon}} \otimes \smallmat{0 \\ 
\boldsymbol{\omega}_{\mathrm{b}}} = 
\smallmat{ - \boldsymbol{\epsilon}^\top  \\ 
\eta \mathbb{I}_{3}   + [ \boldsymbol{\epsilon}]_{\times} 
} \boldsymbol{\omega}_{\mathrm{b}}$. 
From the two lower terms of the matrix at the right-hand side of this last equation, when developing \eqref{eq:q_r}
and computing $\boldsymbol{A}_{\epsilon \omega} = \left. \frac{\partial }{\partial \boldsymbol{\omega}_{\text{b}}} \left( \frac{1}{2} \boldsymbol{{q}}_{\text{r}}  \otimes \smallmat{0 \\ \boldsymbol{\omega}_{\text{b}}} \right) \right|_{\mathrm{eq}}$,
we obtain the two terms in expression \eqref{eq:Aveps_derivation_ALG} given in Algorithm~\ref{alg:linea}.

We now focus on the entry 
$\boldsymbol{A}_{\omega \epsilon}$ of 
$\boldsymbol{A}_{w}$, which should be computed starting from \eqref{eq:w_r}. Since only the last term of the right hand-side depends on $\boldsymbol{\epsilon}$ (through $\boldsymbol{q}_{\mathrm r}$), we obtain
\begin{align}
\label{eq:Aomega_eps_first}
\boldsymbol{A}_{ \omega \epsilon} =  \boldsymbol{J}^{-1}
\boldsymbol{D}_{\mathrm{m}} (\boldsymbol{u}_{\text{eq}}) \lVert  \boldsymbol{w}_{\mathrm{r}} \rVert 
\left. 
\frac{\partial }{\partial \boldsymbol{\epsilon}} \left( \boldsymbol{R}^\top \!(\boldsymbol{q}_{\mathrm{r}}) (\boldsymbol{v}_{\mathrm{r}} \! \shortminus \! \boldsymbol{w}_{\text{r}}) \right) \right|_{\mathrm{eq}}.
\end{align}
To compute the explicit expression of \eqref{eq:Aomega_eps_first}, we exploit again \eqref{eq:diffRtop} and \eqref{eq:diffRsparse}, and use the expression of $\boldsymbol{D}_{\mathrm{m}}$ in \eqref{eq:dm}, together with the identities $\overline \eta^2 - \overline \epsilon_2^2 = \cos \theta$ and $2\overline \eta \overline \epsilon_2^2 = \sin \theta$, which provide, after some simplifications, the expression \eqref{eq:ddotomega_deps}, given in Algoirithm~\ref{alg:linea}.

Let us now move on to deriving the entries of matrix $\boldsymbol{G}_{w}$ in \eqref{eq:lpv_linearisation}, whose components can be derived from  \eqref{eq:v_r} and \eqref{eq:w_r}. Recalling from \eqref{eq:vector_u} the four entries of $\boldsymbol{u}$, and also based on the structure of $\boldsymbol{M}_{\text{f}}$,
$\boldsymbol{D}_{\text{f}}$,  in \eqref{eq:Mf}, \eqref{eq:df}, an explicit form for
\begin{align}
    \boldsymbol{G}_{v} \! &= \! 
    % \begin{bmatrix}
    %      \boldsymbol{G}_{v \tau} \!\! & \!\! \boldsymbol{G}_{v\delta}
    % \end{bmatrix} = 
    \frac{1}{m} \boldsymbol{R}_\theta \! \left.\frac{\partial}{\partial \boldsymbol{u}} \! 
     \left( \boldsymbol{M}_{\text{f}}(\boldsymbol{u}) \! - \! \boldsymbol{D}_{\text{f}}(\boldsymbol{u}) \lVert \boldsymbol{w}_{\text{r}} \rVert \boldsymbol{w}^{\text{b}}_{\mathrm{eq}}   \right)\right|_{\mathrm{eq}} ,
\end{align}
can be computed as in \eqref{eq:Gv_ALG_luca}, after some straightforward factorizations. 

Similarly, based on the matrices  $\boldsymbol{M}_{\text{m}}$, $\boldsymbol{D}_{\text{m}}$ in   in \eqref{eq:Mm}, \eqref{eq:dm}, 
we may compute
\begin{align}
     \boldsymbol{G}_{\omega} \! &= \!  \boldsymbol{J}^{-1} \! \left.\frac{\partial}{\partial \boldsymbol{u}} \! 
     \left( \boldsymbol{M}_{\text{m}}(\boldsymbol{u}) \! - \! \boldsymbol{D}_{\text{m}}(\boldsymbol{u}) \lVert \boldsymbol{w}_{\text{r}} \rVert \boldsymbol{w}^{\text{b}}_{\mathrm{eq}}   \right)\right|_{\mathrm{eq}}
\end{align}
as in \eqref{eq:Gomega_ALG_Luca}, after some straightforward factorizations. 



% All the following expressions are relative to the matrix $\boldsymbol{G}_{w}$ \eqref{eq:lpv_linearisation} linked to \eqref{eq:v_r} and \eqref{eq:w_r}
% \begin{align}
%     \boldsymbol{G}_{v} \! = \! \frac{\partial \dot{\boldsymbol{v}} }{\partial \boldsymbol{u}} \! = \! \begin{bmatrix}
%          \boldsymbol{G}_{vT} \!\! & \!\! \boldsymbol{G}_{v\delta}
%     \end{bmatrix}, \quad    \boldsymbol{G}_{\omega} \! = \! \frac{\partial \boldsymbol{\dot \omega}_{\text{b}} }{\partial \boldsymbol{u}} \! = \! \begin{bmatrix}\boldsymbol{G}_{\omega T} \!\! & \!\! \boldsymbol{G}_{\omega \delta} 
%     \end{bmatrix}
% \end{align}
% where the details of the matrices are shown in \eqref{eq:Gv_ALG} and \eqref{eq:Gomega_ALG} in Algorithm~\ref{alg:linea}.

Let us finally determine the expression of $\boldsymbol{E}_{v}$
in \eqref{eq:lpv_linearisation}
as follows. First note that we may write
$\|\boldsymbol{w}_\text{r}\| \boldsymbol{w}_\text{r} = 
 \boldsymbol{w}_\text{r} \sqrt{\boldsymbol{w}_\text{r}^\top \boldsymbol{w}_\text{r}  }$,
so that
$$
\frac{\partial }{\partial \boldsymbol{w}_\text{r}}
\|\boldsymbol{w}_\text{r}\| \boldsymbol{w}_\text{r} = 
\|\boldsymbol{w}_\text{r}\| \mathbb{I}_3
+ \frac{\boldsymbol{w}_\text{r} \boldsymbol{w}_\text{r}^\top}{\|\boldsymbol{w}_\text{r}\|} = 
\|\boldsymbol{w}_\text{r}\|
\left( \mathbb{I}_3 + \frac{\boldsymbol{w}_\text{r} \boldsymbol{w}_\text{r}^\top}{\boldsymbol{w}_\text{r}^\top \boldsymbol{w}_\text{r}} \right).
$$
Then, starting from \eqref{eq:v_r} and \eqref{eq:w_r}
and following similar computations to the previous cases,
also using the expression of $\boldsymbol{w}_\text{r}$
in \eqref{eq:rotated_coord}, we obtain expression
\eqref{eq:Ev_alg2} (reported in Algorithm~\ref{alg:linea}), for $\boldsymbol{E}_{v} = -  \frac{1}{m} \boldsymbol{R}_\theta 
 \left. \frac{\partial}{\partial \boldsymbol{w}_\text{r}} 
 \left(  \! \boldsymbol{D}_{\text{f}}(\boldsymbol{u}) \lVert \boldsymbol{w}_{\text{r}} \rVert \boldsymbol{w}_{\text{r}}  \right)\right|_{\mathrm{eq}} $ and  
 $\boldsymbol{E}_{w} = - \boldsymbol{J}^{-1} \left.
 \frac{\partial}{\partial \boldsymbol{w}_\text{r}}
     \left(  \boldsymbol{D}_{\text{m}}(\boldsymbol{u}) \lVert \boldsymbol{w}_{\text{r}} \rVert \boldsymbol{w}_{\text{r}}\right)\right|_{\mathrm{eq}}$, where we recall that $\boldsymbol{D}_{\text{m,eq}} = \boldsymbol{D}_{\text{m}}(\boldsymbol{u_{\text{eq}}})$.
% \begin{align*}
%     \left[ \begin{array}{c}
%     \boldsymbol{E}_{vx} \\ \hline
%     \boldsymbol{E}_{vz}
%     \end{array} \right] = \frac{\rho S \lVert \boldsymbol{w}_{\mathrm{eq}} \rVert }{2m} \smallmat{ \boldsymbol{R}_\theta ~\diag(C_{\text{d}},0,C_{\ell}) \\  \boldsymbol{R}_\theta ~\diag(-C_{\text{d}},0,C_{\ell})}  \boldsymbol{R}_\theta^\top \smallmat{1 \\ 0\\ \xi_{\text{f}} \delta}
% \end{align*}
%
% \begin{align*}
%     \left[
%          \boldsymbol{E}_{\omega x}
%          ~ \vline~
%           \boldsymbol{E}_{\omega z}
%     \right] = \frac{\rho S \Delta_{\text{r}} C_{\ell} \lVert \boldsymbol{w} \rVert}{2J_{y}} \left[ \begin{smallmatrix}
%         0 & 0 \\
%         1-2\epsilon_{2}^{2} & 2\epsilon_{2}^{2} \\
%         0 & 0
%         \end{smallmatrix}
%     \vline \begin{smallmatrix}
%         0 & 0 \\
%         -2\epsilon_{2}^{2} & 1-2\epsilon_{2}^{2}  \\
%         0 & 0
%     \end{smallmatrix} \right] \smallmat{\xi_{\text{m}} \delta\\ 1 }
% \end{align*}
%
\end{proof}

\begin{algorithm}
     \caption{Design of the linearization matrices in \eqref{eq:lpv_linearisation}}\label{alg:linea}
      \hspace*{.1cm} \textbf{Input}: Wind vector $\boldsymbol{w}_{\text{eq}} =\smallmat{w_x & w_y & w_z}^\top$ and\\
      \hspace*{1.2cm} equilibrium $(\boldsymbol{u}_{\text{eq}}, \boldsymbol{x}_{\text{eq}})$ from \eqref{eq:equilibrium} and Algorithm~\ref{alg:eq}.\\
 \hspace*{.1cm} \textbf{Output}: 
 Matrices $\boldsymbol{A}_{w}$, $\boldsymbol{G}_{w}$, $\boldsymbol{E}_{w}$ in \eqref{eq:lpv_linearisation}

 \begin{algorithmic}[1]

\State Select parameters $\psi$, $\theta$, $\tau$, $\delta$ in \eqref{eq:equilibrium} from Algorithm~\ref{alg:eq} and  $\overline \epsilon_{2} = \sin{\frac{\theta}{2}}$, $\overline \eta =  \cos{\frac{\theta}{2}}$.

        
\State With the quantities in \eqref{eq:wb}, \eqref{eq:df}, \eqref{eq:dm}, define:
\begin{align*}
    & \boldsymbol{R}_\psi := \smallmat{ \cos \psi & \sin \psi & 0 \\ -\sin \psi & \cos \psi & 0 \\ 0 & 0 & 1 },    \quad 
      \boldsymbol{R}_\theta := \smallmat{ \cos \theta & 0 & \sin \theta \\ 0 & 1 & 0 \\ -\sin \theta & 0 & \cos \theta },         \\
     &\smallmat{w_{\text{r}x} \\ 0 \\ w_{\text{r}z}} :=  \boldsymbol{R}_\psi^\top \boldsymbol{w}_{\text{eq}}, \quad
    \smallmat{w^{\text{b}}_{x} \\ w^{\text{b}}_{z}} := \smallmat{w_{\text{r}x} \cos \theta -   w_{\text{r}z} \sin\theta \\ w_{\text{r}z} \cos \theta + w_{\text{r}x} \sin \theta }
     \\
     &\left[ \! \! \begin{array}{c|c} 
     \boldsymbol{D}_{\text{f,eq}} \! \! &  \! \! \boldsymbol{D}_{\text{m,eq}}
     \end{array} \! \!\right] \! \! := \! \! \frac{\rho S}{2} \left[\begin{array}{c|c} \! \! \begin{smallmatrix}
                \shortminus C_{\text{d}} & 0 & C_{\text{d}}\xi_{\text{f}} \delta\\
                0 & 0 & 0\\
                \shortminus C_{\ell}\xi_{\text{f}} \delta & 0 & \shortminus C_{\ell}
            \end{smallmatrix} &  \! \! \begin{smallmatrix}
                0 & 0 & 0\\
               \Delta_{\text{r}}C_{\ell}\xi_{\text{m}}\delta\ & 0 & 2\Delta_{\text{r}}C_{\ell}\\
                0 & 0 & 0
            \end{smallmatrix} \end{array} \! \!\right]
\end{align*}

\State Define the drag and lift forces, and their derivatives with respect to $\epsilon_2$ (defined in Step 1), as
\begin{equation}
\label{eq:draglift_ALG}
\!\! \smallmat{\!f_{\text{d}} & \frac{\partial  f_{\text{d}}  }{\partial \epsilon_{2}} \!\! \\ 0 & 0  \\ \! f_{\ell} & \frac{\partial  f_{\ell} \!\! }{\partial \epsilon_{2}}} 
 \!\! := \! 
    \shortminus \| \boldsymbol{w}_{\mathrm{eq}} \|  \boldsymbol{D}_{\text{f,eq}}\! 
    \smallmat{\! w^{\text{b}}_{x} &  \left(\! 4 \overline{\eta}  \shortminus \frac{2\overline \epsilon_{2}^{2}}{\overline{\eta} } \! \right) w_{\text{r}z} \shortminus 8 \overline \epsilon_{2}  w_{\text{r}x} \!\! \\  0 & 0 \\  
   \! w^{\text{b}}_{z} &    \left(\! 4 \overline{\eta}  \shortminus \frac{2\overline \epsilon_{2}^{2}}{\overline{\eta} } \! \right) w_{\text{r}x} -8 \overline \epsilon_{2} w_{\text{r}z} \!\!
    }\!\!, 
\end{equation}

\State Define the motor and elevon forces as
\begin{align}
\label{eq:motor_elevon_forces}
    \smallmat{ f_{\text{m}}  \\f_{\text{e}} }\! := \!\smallmat{\left(\frac{S_{\text{wet}}C_{\text{d}}}{2S_{\text{p}}}-2\right)\tau \\  - \frac{S_{\text{wet}}\tau \delta \xi_{\text{f}} C_{\ell}}{2S_{\text{p}}}}
\end{align}

\State Select the entries of matrix $\boldsymbol{A}_w$ in \eqref{eq:lpv_linearisation} as:
\begin{align}
\label{eq:Avv_alg2}
& \boldsymbol{A}_{vv} =
\frac{\| \boldsymbol{w}_{\mathrm{eq}} \| }{m} 
 \boldsymbol{R}_\theta \boldsymbol{D}_{\text{f,eq}}  \boldsymbol{R}_\theta^\top \\
\nonumber
&\smallmat{\boldsymbol{A}_{v\epsilon}^{1,2} \\ 
 \boldsymbol{A}_{v\epsilon}^{2,1} \\ \boldsymbol{A}_{v\epsilon}^{2,3} \\ \boldsymbol{A}_{v\epsilon}^{3,2}}
      := \smallmat{ 2 \overline{\eta}  - \frac{\overline \epsilon_{2}^{2}}{\overline{\eta} }  & 4 \overline \epsilon_{2} &  2\overline \epsilon_{2}^{2} -1 & 2\overline \epsilon_{2} \overline{\eta} \\
     -2 \overline{\eta}  & -2 \overline \epsilon_{2} & 0 & 0 \\ 
      2 \overline \epsilon_{2} & -2 \overline{\eta}  & 0 & 0 \\
      -4 \overline \epsilon_{2} & 2 \overline{\eta}  - \frac{\overline \epsilon_{2}^{2}}{\overline{\eta} } & -2\overline \epsilon_{2} \overline{\eta}  & 1- 2\overline \epsilon_{2}^{2}   }\smallmat{ f_{\text{e}}  + f_{\ell}  \\ f_{\text{m}}  + f_{\text{d}}  \\ \frac{\partial  f_{\text{d}}  }{\partial \overline \epsilon_{2}} \\ \frac{\partial  f_{\ell}  }{\partial \overline \epsilon_{2}}}\\
\label{eq:Aveps_derivation_ALG}
&  \boldsymbol{A}_{v\epsilon} = \frac{1}{m}\smallmat{ 
        0 & \boldsymbol{A}_{v\epsilon}^{1,2} & 0 \\ 
        \boldsymbol{A}_{v\epsilon}^{2,1} & 0 & \boldsymbol{A}_{v\epsilon}^{2,3}  \\
        0 & \boldsymbol{A}_{v\epsilon}^{3,2} & 0
   }, 
    %\label{eq:Aeomega_ALG}
   \boldsymbol{A}_{\epsilon \omega} = \frac{\overline{\eta}  }{2} \mathbb{I}_{3} + \frac{\overline \epsilon_{2}}{2}\smallmat{
        0 & 0 &  1 \\ 
        0 & 0 & 0  \\
        \shortminus 1 & 0 & 0
    }\\
\label{eq:ddotomega_deps}
&   \boldsymbol{A}_{ \omega \epsilon} = \tfrac{\rho S C_{\ell} \Delta_{\text{r}}  \lVert \boldsymbol{w}_{\mathrm{eq}} \rVert (w^{\text{b}}_{x} -  \xi_{\text{m}} \delta w^{\text{b}}_{z})}{J_{y} \overline{\eta}  } \smallmat{ 
        0 & 0 & 0 \\ 
        0 & 1  & 0  \\
       0 & 0 & 0
    }
\end{align}

\State Select the entries of matrix $\boldsymbol{G}_w$ in \eqref{eq:lpv_linearisation} as: 
\begin{align}
\nonumber
& \boldsymbol{G}_{v} = \frac{1}{m}    \boldsymbol{R}_\theta \!\! \left[ \!\! \begin{array}{c|c} 
     \boldsymbol{G}_{v\tau}\!\! & \! \! \boldsymbol{G}_{v\delta} 
    \end{array} \!\! \right ], \; \boldsymbol{G}_{v\tau} :=      \smallmat{
   1 \shortminus \frac{S_{\text{wet}} C_{\text{d}}}{4S_{\text{p}}}\\ 0 \\ \shortminus\frac{S_{\text{wet}} C_{\ell} \xi_{\text{f}}\delta}{2S_{\text{p}}} } \smallmat{ 1 \\ 1}^\top                     \\
   \label{eq:Gv_ALG_luca}
   &\quad  \boldsymbol{G}_{v\delta} := 
   \smallmat{
        -   \frac{1}{4}\rho S C_{\text{d}} \xi_{\text{f}} \lVert \boldsymbol{w}_{\text{eq}} \rVert w^{\text{b}}_{z}\\0 \\\shortminus \frac{S_{\text{wet}}  C_{\ell} \xi_{\text{f}}\tau}{2S_{\text{p}}} + \frac{1}{4}\rho S C_{\ell}  \xi_{\text{f}} \lVert \boldsymbol{w}_{\text{eq}} \rVert w^{\text{b}}_{x}
    } \smallmat{ 1 \\ 1}^\top 
\end{align}
\begin{align}
\nonumber
     \boldsymbol{G}_{\omega} \! &= \!  \boldsymbol{J}^{-1} \begin{bmatrix}\boldsymbol{G}_{\omega \tau} \!\! & \!\! \boldsymbol{G}_{\omega \delta} 
    \end{bmatrix}, 
    \boldsymbol{G}_{\omega \delta} :=
      \tfrac{S_{\text{wet}}C_{\ell}\tau}{4S_{\text{p}}} 
      \smallmat{ a_{y} \xi_{\text{f}} & - a_{y} \xi_{\text{f}}\\
                 \Delta_{\text{r}}\xi_{\text{m}} & \Delta_{\text{r}}\xi_{\text{m}}\\
                 0 & 0} + 
    \\
\label{eq:Gomega_ALG_Luca}
    &\qquad \qquad +  \tfrac{\rho S \lVert \boldsymbol{w}_{\mathrm{eq}} \rVert\xi_{\text{m}}}{4}
    \smallmat{ a_{y} C_{\text{d}} w^{\text{b}}_{x} & -  a_{y} C_{\text{d}} w^{\text{b}}_{x} \\
    \Delta_{\text{r}} C_{\ell} w^{\text{b}}_{x} & \Delta_{\text{r}} C_{\ell} w^{\text{b}}_{x} \\
    a_{y} C_{\ell} w^{\text{b}}_{z} & - a_{y} C_{\ell} w^{\text{b}}_{z} 
    }  \\
\nonumber
   & \boldsymbol{G}_{\omega \tau} := \smallmat{ \frac{k_{\text{m}}}{k_{\text{f}}} + \frac{S_{\text{wet}}}{4S_{\text{p}}}  a_{y} \xi_{\text{f}} C_{\ell} \delta \\
    0 \\ p_{y}+ \frac{S_{\text{wet}}}{4S_{\text{p}}}  a_{y} C_{\text{d}} }
    \smallmat{ 1 \\ \shortminus  1}^\top 
     \!\! + \!\! 
     \smallmat{ 0 \\ \frac{S_{\text{wet}}}{4S_{\text{p}}} 
     \Delta_{\text{r}} \xi_{\text{m}} C_{\ell}\delta
     \\ 0} \smallmat{ 1 \\ 1}^\top 
\end{align}




\State Select the entries of matrix $\boldsymbol{E}_w$ in \eqref{eq:lpv_linearisation} as:
\begin{align}
\label{eq:Ev_alg2}
\left[  \begin{smallmatrix}
   \! \boldsymbol{E}_{v} \rule[-0.1cm]{0cm}{0.35cm} \! \\ \hline \! \boldsymbol{E}_{\omega} \rule{0cm}{0.25cm} \!
     \end{smallmatrix} \right] \!=\! \shortminus
      \left[ \begin{smallmatrix}
     \boldsymbol{A}_{vv} \rule[-0.1cm]{0cm}{0.35cm}
     \\ \hline 
     \rule{0cm}{0.35cm}
     \!\! \boldsymbol{J} \| \boldsymbol{w}_{\mathrm{eq}} \| \boldsymbol{D}_{\text{m,eq}} \boldsymbol{R}_\theta^\top \!\! 
     \end{smallmatrix} \right]
         \left( \mathbb{I}_3 \!+\! \tfrac{
        \boldsymbol{R}_\psi^\top \boldsymbol{w}_\text{eq} \boldsymbol{w}_\text{eq}^\top \boldsymbol{R}_\psi}{\boldsymbol{w}_\text{eq}^\top \boldsymbol{w}_\text{eq}} \right)
\end{align}

% \begin{align}
% \label{eq:Ev_alg2}
% \boldsymbol{E}_{v} = - \boldsymbol{A}_{vv} \left( \mathbb{I}_3 + \frac{
% \boldsymbol{R}_\psi^\top
% \boldsymbol{w}_\text{eq} \boldsymbol{w}_\text{eq}^\top \boldsymbol{R}_\psi}{\boldsymbol{w}_\text{eq}^\top \boldsymbol{w}_\text{eq}} \right) .
% \end{align}


% \State \begin{align}
% \label{eq:Eomega_alg2}
% \boldsymbol{E}_{\omega}  = - \boldsymbol{J} \| \boldsymbol{w}_{\mathrm{eq}} \| \boldsymbol{D}_{\text{m}}(\boldsymbol{u_{\text{eq}}})  \boldsymbol{R}_\theta^\top \left( \mathbb{I}_3 + \frac{
% \boldsymbol{R}_\psi^\top
% \boldsymbol{w}_\text{eq} \boldsymbol{w}_\text{eq}^\top \boldsymbol{R}_\psi}{\boldsymbol{w}_\text{eq}^\top \boldsymbol{w}_\text{eq}} \right) .
% \end{align}
     \end{algorithmic} 
\hspace*{.1cm} \textbf{Return}:  $\boldsymbol{A}_{w}$, $\boldsymbol{G}_{w}$, $\boldsymbol{E}_{w}$ 
 \end{algorithm}

\section{Conclusion du Chapitre \ref{chap:model}}
 
\todo{Conclusion sur la modification du vecteur de commande }
\chapter{Commande hybride}
\minitoc

\section{Motivation}

\section{Schéma de commande hybride}

\section{Simulations}







\chapter{Commande proportionnelle intégrale d'un drone convertible à 6 degrés de liberté}
\minitoc

\section{Schéma de commande linéaire proportionnel intégral : 6 Dof}

\section{Maquette expérimentale : 6 Dof}

\section{Résultats}







\chapter{Commande proportionnelle intégrale de DarkO}
\minitoc
\label{chap:6DOF}

\section{Motivation}
\label{sec:motivation6DOF}
Nous avons observé expérimentalement dans le chapitre \ref{chap:3DOF} qu'un drone \textit{tailsitter} pouvait se stabiliser en stationnaire face à du vent, vers les points d'équilibre décrits dans la section \ref{sec:eq_vent}, grâce à une architecture de commande linaire. Cette dernière est basée sur un retour de sortie, proportionnel intégral, libérant deux degrés de liberté sur l'orientation du drone.


\section{Commande linéaire proportionnelle intégrale}
\label{sec:6dofcmd}
\subsection{Description du schéma de contrôle}
\label{sec:ctl_sche}

Une inspection minutieuse des matrices de commande et d'entrée des perturbations $\boldsymbol{G}_{w}$ et $\boldsymbol{E}_{w}$ du modèle \eqref{eq:lpv_linearisation} (voir la sortie de l'Algorithme~\ref{alg:linea}) suggère une architecture de contrôle efficace pour rejeter une perturbation de vent constante $\boldsymbol{w}$. En effet, les ailerons et les hélices peuvent être utilisés symétriquement pour générer respectivement un moment autour de l'axe $y_{[\text{b}]}$, vérifiant l'équation \eqref{eq:momenty}, et une force le long de l'axe $x_{[\text{b}]}$, vérifiant l'équation \eqref{eq:forcex}, compensant ainsi l'effet de la perturbation. Néanmoins, il reste une force le long de l'axe $z_{[\text{b}]}$ à compenser en vérifiant l'équation \eqref{eq:forcez}. Une action intégrale peut converger asymptotiquement vers la force désirée, même avec une perturbation du vent non mesurée ($\boldsymbol{w}$). Nous pouvons donc stabiliser le drone à l'équilibre en vol stationnaire tel que caractérisé dans le Théorème~\ref{thm:eqs}. Comme nous ne mesurons pas le vent $\boldsymbol{w}$, les valeurs de $\psi$ et $\theta$ dans l'Algorithme~\ref{alg:eq} sont inconnues. Le contrôleur proposé, illustré à la Fig.~\ref{fig:commande_int6DOF}, utilise l'action intégrale pour faire converger ces angles vers leur valeur d'équilibre. Le bouclage utilise les variables d'erreur suivantes : 
\begin{align}
    \boldsymbol{e}_{p}= \boldsymbol{r}_{p} - \boldsymbol{p}, \; \boldsymbol{e}_{v \epsilon \omega} = -  
       \left[ \begin{smallmatrix} \mathbb{I}_{3}  & \mathbb{0}_{3\times 1} & \mathbb{0}_{3\times 2} & \mathbb{0}_{3}\\
       \mathbb{0}_{1\times 3}  & 1 & \mathbb{0}_{1\times 2} & \mathbb{0}_{1 \times 3} \\
           \mathbb{0}_{3}  & \mathbb{0}_{3\times 1} & \mathbb{0}_{3\times 2} &   \mathbb{I}_{3}
           \end{smallmatrix} \right]
    \smallmat{
           \tilde{\boldsymbol{v}} \\
           \tilde{\boldsymbol{\epsilon}} \\
           \tilde{\boldsymbol{\omega}}_{\text{b}} 
    }.
  \label{eq:error_var}
\end{align} 

Ces variables d'erreur doivent converger vers zéro dans le cas d'un vol stationnaire et pour n'importe quelle position de référence constante $\boldsymbol{r}_{p} \in \mathbb{R}^{3}$. Notons que $\boldsymbol{r}_{p}$ est l'entrée de référence du schéma de contrôle.

Les variables d'erreur dans \eqref{eq:error_var} peuvent être représentées comme dans le schéma  de la Fig.~\ref{fig:commande_int6DOF}, c'est-à-dire, en définissant la sortie $\boldsymbol{y}\in \real^{10}$ de la dynamique du système linéarisé \eqref{eq:lpv_linearisation}, ayant le vecteur d'état incrémental $\boldsymbol{\tilde{x}} \in \real^{10 \times 1}$ défini ci-dessous : 
\begin{align}
    \label{eq:output_lin}
    \boldsymbol{y} = \boldsymbol{C} \boldsymbol{\tilde{x}} + \smallmat{\boldsymbol{p}_{\mathrm{eq}} \\ \mathbb{0}_{7 \times 1}}, \quad
 \boldsymbol{C} := \smallmat{\mathbb{I}_{6} & \mathbb{0}_{6\times 1} & \mathbb{0}_{6\times 2} &\mathbb{0}_{6\times 3}\\
    \mathbb{0}_{1\times 6} & 1 & \mathbb{0}_{1\times 2} &\mathbb{0}_{1\times 3}\\
    \mathbb{0}_{3\times 6} & \mathbb{0}_{3\times 1} & \mathbb{0}_{3\times 2} &  \mathbb{I}_{3}
},
\end{align}
où la matrice de sortie $\boldsymbol{C} \in \real^{10\times12}$ enlève la composante $\tilde{\epsilon}_{2}$ et $\tilde{\epsilon}_{3}$ du vecteur d'état $\tilde{\boldsymbol{x}}$. 

\begin{figure}[t!]
    \centering
    \includegraphics[width=0.8\columnwidth]{figures/commande_integrale.png}
    \caption{Schéma de commande intégrale avec la perturbation de vent $\boldsymbol{w}$, la perturbation du système à l'entrée $\boldsymbol{d}$ et à la sortie $\nu$.}
    \label{fig:commande_int6DOF}
\end{figure}

Comme le montre la Figure~\ref{fig:commande_int6DOF}, les équations de la dynamique du contrôleur sont basées sur l'erreur $\boldsymbol{e}$ suivante :

\begin{subequations}
    \label{eq:contoller}
    \begin{align}
        \boldsymbol{e} = \smallmat{
        \boldsymbol{e}_{p}^\top & \boldsymbol{e}_{v\epsilon\omega}^\top}^\top, \quad \dot{\boldsymbol{x}}_{c} = \boldsymbol{H} \boldsymbol{e}, \quad    \boldsymbol{u} = \boldsymbol{\Sigma} \boldsymbol{x}_{c} + \boldsymbol{u}_{K},
        \\
        \boldsymbol{\Sigma} := \begin{bmatrix} \! 1 \!&\! 1\! & \!0\! &\! 0\!\\ \!0\! & \!0\! & \!1 \!& \!1\!\end{bmatrix}^\top, \quad
        \boldsymbol{u}_{K} = \frac{n_1 s+n_0}{d_2 s^{2}+d_1 s + d_0}\boldsymbol{K} \boldsymbol{e} ,
    \end{align}
\end{subequations}
où $\boldsymbol{x}_{c} \in \mathbb{R}^{2}$ est l'état intégral, $\boldsymbol{\Sigma}$ est une matrice d'allocation des entrées qui permet d'affecter la première composante de l'état de l'intégrateur à l'action des hélices et la deuxième composante à l'action des élevons. Les scalaires $n_1$, $n_0$,  $d_2$,  $d_1$,  $d_0$ sont respectivement les coefficients du numérateur et du dénominateur d'un filtre utilisé pour éviter une transmission directe entrée-sortie qui amplifierait le bruit de mesure à haute fréquence. Ce filtre induit un contrôleur strictement propre, pour une robustesse accrue aux incertitudes additives. 

Nous définissons le contrôleur $\boldsymbol{F}$ ayant pour dimensions 4\texttimes10 avec la matrice de transfert $\boldsymbol{F}(s) = T_{\boldsymbol{e} \rightarrow \boldsymbol{u}}(s)$ telle que décrite dans \eqref{eq:contoller} et l'interconnexion détaillée dans la Figure~\ref{fig:commande_int6DOF}. Le système $\boldsymbol{P}$ de dimensions 10\texttimes4 représente la dynamique linéarisée de DarkO. La sortie du système $\boldsymbol{y} \in \real^{10\times1}$ est utilisée comme entrée du contrôleur $\boldsymbol{F}$.

Compte tenu des symétries des actionneurs du drone, nous avons contraint la structure de la matrice $\boldsymbol{K}$ dans \eqref{eq:contoller}, associée à l'action proportionnelle du contrôleur, afin d'utiliser les actionneurs selon leur action physique. Ainsi, $\boldsymbol{K}$ prend la forme : 
\begin{align}
\label{eq:k_struct}
    \!\!\!\boldsymbol{K}_{\text{struct}} \!=\!  \smallmat{
             k_{1}& \shortminus k_{2}& k_{3}&  k_{4}& \shortminus k_{5}&  k_{6}& \shortminus k_{7}&  k_{8}&  k_{9}& \shortminus k_{10}\\
             k_{1}&  k_{2}& k_{3}&  k_{4}&  k_{5}&  k_{6}&   k_{7}& \shortminus k_{8}& \shortminus k_{9}&  k_{10}\\
            \shortminus k_{11}& \shortminus k_{12}& k_{13}& \shortminus k_{14}& \shortminus k_{15}& \shortminus k_{16}&   k_{17}& \shortminus k_{18}&  k_{19}& \shortminus k_{20}\\
            \shortminus k_{11}&  k_{12}& k_{13}& \shortminus k_{14}&  k_{15}&  k_{16}&   \shortminus k_{17}&  k_{18}&  k_{19}&  k_{20} 
         }.
\end{align}

La structure précédente s'explique par le comportement physique du drone. Une erreur de position sur l'axe $z_{[\text{i}]}$ du repère inertiel NED (voir Fig.~\ref{fig:darko2}) entraîne une utilisation symétrique des deux hélices, ce qui génère une force le long de l'axe $x_{[\text{b}]}$ du drone. L'utilisation symétrique des deux moteurs se traduit par le même signe dans les coefficients $k_{3}$ et $k_{6}$ des colonnes 3 et 6 de $\boldsymbol{K}$, qui correspondent respectivement aux erreurs de position et de vitesse sur l'axe $z_{[\text{i}]}$. De même, une erreur de position ou de vitesse le long de l'axe latéral du drone $y_{[\text{b}]}$ sera compensée par une utilisation antisymétrique des moteurs, comme le montrent les coefficients $k_{2}$ et $k_{5}$ et leur signe opposé dans les colonnes 2 et 5 de $\boldsymbol{K}$. Une erreur de vitesse angulaire autour de l'axe $x_{[\text{b}]}$ doit être compensée par une utilisation antisymétrique des élevons, comme le montre le coefficient $k_{18}$ de signe opposé dans la colonne 8 de $\boldsymbol{K}$. Des arguments parallèles expliquent les coefficients restants de la matrice $\boldsymbol{K}$ dans \eqref{eq:k_struct}. Toutes ces explications ne sont valables que dans un voisinage de l'équilibre, où le drone se trouve être à la verticale. On comprend aisément que dans d'autres configurations, ces contraintes d'actionnement ne sont plus valides. Un avantage de la structure de \eqref{eq:k_struct} est la réduction du nombre de variables à optimiser, de 40 à 20 gains scalaires, cela se traduisant par une diminution du temps nécessaire à l'optimisation.

La boucle fermée illustrée à la Figure~\ref{fig:commande_int6DOF} est un retour de sortie à dix éléments, comprenant les trois positions, les trois vitesses linéaires, l'un des trois angles d'attitude ($\epsilon_{1}$) et les trois vitesses angulaires. Cette structure peut être considérée comme un bouclage proportionnel-intégral MIMO. Les paramètres à régler dans le contrôleur $\boldsymbol{F}$ \eqref{eq:contoller} sont le gain proportionnel $\boldsymbol{K} \in \real^{4\times10}$ dans \eqref{eq:k_struct}, le gain intégral $\boldsymbol{H} \in \real^{2\times10}$ et les paramètres du filtre $n_1$, $n_0$, $d_2$, $d_1$, $d_0$, comme repérés en jaune sur la Fig.~\ref{fig:commande_int6DOF}. Une méthode de réglage appropriée doit garantir une réjection adéquate des perturbations et une robustesse satisfaisante aux dynamiques non modélisées. Ces deux objectifs conduisent à un compromis car le rejet des perturbations nécessite un réglage agressif tandis que les propriétés de robustesse sont assurées par une stratégie d'atténuation des hautes fréquences.

Nous examinons ensuite deux méthodes de réglage basées sur l'optimisation. La première est issue des idées proposées dans \ref{sec:3dofcmd}, méthode qui ne nécessitait pas la dynamique linéarisée des Théorèmes~\ref{thm:eqs} et~\ref{th:lin}, et est résumée dans la section~\ref{sec:zerowind}. Il s'agit d'une synthèse multiobjectif avec des contraintes $H_{\infty}$, basée sur le modèle de vent nul, discutée dans la section~\ref{sec:eq_nowind} et~\ref{sec:nowind_lin} et détaillée dans \ref{sec:3dofcmd}. Nous montrerons que cette première méthode ne permet pas de stabiliser le drone dans certaines plages de vent, en raison de la méconnaissance de la dynamique caractérisée par les Théorèmes~\ref{thm:eqs} et~\ref{th:lin}. La deuxième méthode de réglage, présentée à la section \ref{sec:h_inf6DOF_multi}, est une synthèse itérative multiobjectif avec des contraintes $H_{\infty}$, basée sur un ensemble de modèles associés à différentes conditions de vent, lesquels sont obtenus des Théorèmes~\ref{thm:eqs} et~\ref{th:lin}, par le biais des Algorithmes~\ref{alg:eq} et~\ref{alg:linea}.


Dans notre validation numérique, présentée dans les sections~\ref{sec:zerowind} et~\ref{sec:h_inf6DOF_multi} (voir en particulier les Fig.~\ref{fig:SimSytuneStruct_zero} et Fig.~\ref{fig:SimSytuneStruct_lpv}), un bruit de mesure est ajouté à la sortie pour produire des résultats numériques semblables aux expérimentaux. Les écarts types des niveaux de bruit adoptés sont indiqués dans la Table~\ref{tab:noise}.
\begin{table}[ht!]
    \centering
    \begin{tabular}{|c|c|c|} 
        \hline
        Grandeurs & Valeurs & Unités\\
        \hline
        $\boldsymbol{p}$ & \SI{2.5e-4}{} & \SI{}{\meter}  \\ 
        \hline
        $\tilde{\boldsymbol{v}}$  & \SI{1.2e-3}{} &  \SI{}{\meter\per\second}  \\ 
        \hline
        $\tilde{\boldsymbol{\epsilon}}$ & \SI{4.7e-4}{} &  \\
        \hline
        $\tilde{\boldsymbol{\omega}}_{\text{b}}$ & \SI{2.7e-3}{} &\SI{}{\radian\per\second}\\
        \hline
    \end{tabular}
    \caption{ Écart-type du bruit pour la modélisation des capteurs en simulation.}
    \label{tab:noise}
\end{table}

En plus de présenter les résultats de la simulation du bouclage linéaire de la Fig. ~\ref{fig:commande_int6DOF} avec le modèle linéarisé \eqref{eq:lpv_linearisation}, dans les sections~\ref{sec:zerowind} et~\ref{sec:h_inf6DOF_multi}, nous simulons également la boucle fermée en remplaçant le modèle linéarisé $\boldsymbol{P}$ par le modèle non linéaire \eqref{eq:dyna_orig}, comprenant la dynamique réelle du drone.

Lorsque l'on remplace le modèle linéarisé par la dynamique non linéaire \eqref{eq:dyna_orig}, dont l'état est $\boldsymbol{x} = (\boldsymbol{p}, \boldsymbol{v}, \boldsymbol{q},\boldsymbol{\omega}_{\text{b}}) \in \real^{13} $, nous remplaçons la sortie linéaire $\boldsymbol{y}$ avec la version non linéaire suivante :
\begin{align}
\label{eq:output}
    \boldsymbol{y}_{\text{NL}} \!=\! \smallmat{\boldsymbol{p}\\
     \boldsymbol{v}\\
     \epsilon_{1}\\
     \boldsymbol{\omega}_{\text{b}}} \!=\! \left[ \begin{smallmatrix} \mathbb{I}_{6} & \mathbb{0}_{6\times 1} & \mathbb{0}_{6\times 1} & \mathbb{0}_{6\times 2} & \mathbb{0}_{3}\\
     \mathbb{0}_{1\times 3} & 0 & 1 & \mathbb{0}_{1\times 2} & \mathbb{0}_{1 \times 3} \\
         \mathbb{0}_{3} & \mathbb{0}_{3\times 1} & \mathbb{0}_{3\times 1} & \mathbb{0}_{3\times 2} &   \mathbb{I}_{3}
         \end{smallmatrix} \right]
         \smallmat{\boldsymbol{R}^\top_{\psi}\boldsymbol{p} \\ \boldsymbol{R}^\top_{\psi}\boldsymbol{v} \\
\boldsymbol{q}_{\mathrm{eq}\psi}^{-1} \otimes \boldsymbol{q} \\
         \boldsymbol{\omega}_{\text{b}}  }.
\end{align}


Dans les sections suivantes, nous notons la marge de module d'une matrice de transfert $s \mapsto T_{v \rightarrow z}$ as $\Delta_m(T_{v \rightarrow z}) = \min\limits_{\omega\in R} \sigma_{\min}(T_{v \rightarrow z}(j\omega))$.



\subsection{Contrôleur optimisé sous contraintes $H_{\infty}$, cas sans vent}
\label{sec:zerowind}

Pour régler le contrôleur dans le cas sans vent, nous utilisons le modèle linéaire du système décrit dans la section~\ref{sec:nowind_lin}, $\boldsymbol{P}(s) = T_{\boldsymbol{u} \rightarrow \boldsymbol{y}}(s)$, obtenu à partir des équations \eqref{eq:linearized} et \eqref{eq:output_lin} tel que :

\begin{align*}
    \boldsymbol{P}(s) = \boldsymbol{C} (s \mathbb{I}_{12} - \boldsymbol{A}_{0})^{-1} \boldsymbol{G}_{0}.
\end{align*} 

En lien avec la Figure~\ref{fig:commande_int6DOF}, nous introduisons des matrices de transfert qui correspondent aux objectifs de robustesse : la fonction de sensibilité en sortie définie par $T_{\nu \rightarrow e}=(\mathbb{I}_{10}+\boldsymbol{P}\boldsymbol{F})^{-1}$ de dimensions 10\texttimes10, telle que $\lVert T_{\nu \rightarrow e} \rVert _{\infty}=\Delta_m(T_{\nu \rightarrow e})^{-1} $ et la fonction de sensibilité en entrée $T_{d \rightarrow u}=(\mathbb{I}_{4}+\boldsymbol{F}\boldsymbol{P})^{-1}$ de dimensions 4\texttimes4, définie par $\lVert T_{d \rightarrow u} \rVert _{\infty}=\Delta_m(T_{d \rightarrow u})^{-1}$.
Par conséquent, la minimisation de la norme $H_{\infty}$ de $T_{\nu \rightarrow e}$ ou de $T_{d \rightarrow u}$ correspond à l'augmentation des marges de module en entrée et en sortie. Étant donné que le système $\boldsymbol{P}$ est MIMO, nous accordons de l'importance aux fonctions de sensibilité en entrée et en sortie qui ne coïncident pas, car $\boldsymbol{P}$ et $\boldsymbol{F}$ ne commutent pas.

Nous définissons aussi la matrice de transfert $T_{\nu \rightarrow u}$ de dimensions 4\texttimes10 liée à l'impact du bruit de mesure $\nu$ sur la commande $\boldsymbol{u}$, e $T_{d \rightarrow y}$ de dimensions 10\texttimes4 représentant l'impact de la perturbation en entrée $\boldsymbol{d}$ sur la sortie du système $\boldsymbol{y}$. 

Nous résolvons le même problème que dans notre travail précédent \ref{eq:pb_optim} en utilisant le logiciel {\tt Systune} \cite{1576856}, mais nous utilisons le diagramme de contrôle présenté dans la section~\ref{sec:ctl_sche} qui comprend un filtre sur l'action proportionnelle et un nombre différent de sorties. Nous incluons également dans le système $\boldsymbol{P}$ la dynamique des actionneurs linéaires discutée dans la section~\ref{sec:saturation}.
\begin{figure}[ht!]
    \centering
    \includegraphics[trim=0cm 0cm 0cm 0cm,clip,width=0.6\columnwidth]{figures/sim_systune_zero_wind.png}
    \caption{Simulation du modèle non linéaire \eqref{eq:dyna_orig} (ligne continue) et du modèle linéarisé \eqref{eq:lpv_linearisation} (ligne en pointillé) avec des incréments de vent constants croissants et le contrôleur basé sur l'optimisation sans vent de la section~\ref{sec:zerowind}.}
    \label{fig:SimSytuneStruct_zero}
\end{figure}

Une augmentation successive de l'intensité horizontale et verticale du vent (allant de 0 à \SI{-6}{\meter\per\second}) est appliquée, comme le montre le tracé inférieur de la Fig.~\ref{fig:SimSytuneStruct_zero}. Les couples de vent sélectionnés $(w_{x}, w_{z})$ sont représentés par des points rouges sur les surfaces de la figure~\ref{fig:saturation}, où l'on peut voir que l'équilibre $(\boldsymbol{u}_{\text{eq}}, \boldsymbol{x}_{\text{eq}})$ est atteint sans que les actionneurs ne soient saturés. Nous ne nous intéressons qu'à la partie négative de la vitesse verticale du vent car elle est la plus limitante. En effet, le drone est soulevé par le vent vertical ascendant (dont le signe est négatif dans le cadre de la NED), nécessitant ainsi moins de traction sur les hélices pour compenser la gravité. Les moteurs génèrent moins de flux d'air sur les élevons, ce qui réduit leur efficacité, conduit à la saturation et déstabilise le drone.
L'objectif du système de contrôle est de maintenir le drone en position de vol stationnaire (définie comme $\boldsymbol{r}_{p} = [0,0,0]^\top$), malgré l'augmentation du vent horizontal et vertical $w_{x}$ et $w_{z}$. 


La Figure~\ref{fig:SimSytuneStruct_zero} présente à la fois des simulations linéaires avec la dynamique linéarisée \eqref{eq:lpv_linearisation} (en pointillé) et des simulations non linéaires avec le modèle non-linéaire \eqref{eq:dyna_orig} (en traits pleins). Les simulations linéaires et non linéaires montrent systématiquement que le contrôleur fonctionne bien à faible vitesse de vent (effectivement, le réglage est effectué sur la base du modèle de vent nul). Cependant, lorsque la vitesse du vent $w_{x}$ et $w_{z}$ dépasse \SI{-5}{\meter\per\second}, la position de vol stationnaire devient instable et le drone oscille et diverge. Les angles d'inclinaison $\theta$ sont utilisés pour représenter l'attitude afin de donner un meilleur aperçu du comportement du véhicule, mais la simulation de la dynamique non linéaire \eqref{eq:dyna_orig} est effectuée avec un quaternion unitaire. L'instabilité observée dans les résultats de la simulation de la figure~\ref{fig:SimSytuneStruct_zero} confirme les instabilités expérimentales rapportées dans la section \ref{sec:exp3DOF}, où nous avons utilisé cette même méthode de réglage. Est également confirmée l'importance des Théorèmes~\ref{thm:eqs} et~\ref{th:lin} dans la Section~\ref{sec:model}, pour un réglage approprié des gains du contrôleur, comme cela est effectué dans la section suivante.


\subsection{Contrôleur optimisé sous contrainte $H_{\infty}$, cas multimodèle}
\label{sec:h_inf6DOF_multi}

Les résultats de simulation obtenus avec la méthode de réglage sans vent (voir la Figure~\ref{fig:SimSytuneStruct_zero}), ainsi que les instabilités expérimentales observées dans la section \ref{sec:exp3DOF} confirment la nécessité d'une procédure de réglage du gain du contrôleur exploitant les linéarisations paramétrées en fonction du vent des Théorèmes~\ref{thm:eqs} et~\ref{th:lin}. En nous concentrant à nouveau sur le schéma de contrôle de la Figure~\ref{fig:commande_int6DOF}, nous considérons maintenant explicitement l'effet du vent (linéarisé) sur le système \eqref{eq:lpv_linearisation} avec la sortie \eqref{eq:output_lin} et avec les sélections de l'algorithme~\ref{alg:linea} comme définies ci-dessous :
\begin{align*}
\label{eq:Pw_synthesis}
\numberthis
    \boldsymbol{P}_w(s) &= \begin{bmatrix}
        \boldsymbol{P}_{u}(s;w) &  \boldsymbol{P}_{w}(s;w)
    \end{bmatrix}\\&:= \boldsymbol{C} (s \mathbb{I}_{12} - \boldsymbol{A}_{w})^{-1} \begin{bmatrix}\boldsymbol{G}_{w} &   \boldsymbol{E}_{w}\end{bmatrix},
\end{align*}

où l'entrée est la concaténation de l'entrée de commande $\boldsymbol{u}$ et de l'entrée de perturbation de vent $\boldsymbol{w}$. Comme le modèle dépend de la vitesse du vent $\boldsymbol{w}$, nous introduisons une nouvelle matrice de transfert $T_{w \rightarrow y}$ ayant des dimensions de 10\texttimes3, laquelle correspond à la matrice de transfert entre l'entrée du vent $\boldsymbol{w}$ et la sortie du système $\boldsymbol{y}$, quantifiant l'effet de la perturbation du vent sur la boucle de commande du drone. 

Avec l'ensemble des matrices de transfert définies dans la section~\ref{sec:zerowind} et la nouvelle matrice de transfert $T_{w \rightarrow y}$, nous utilisons l'approche proposée dans \cite{1576856,ApkarianMulti} qui utilise des techniques d'optimisation non lisses pour traiter les problèmes de bouclage non convexes, appelée ''{\tt Systune}''. Ainsi nous pouvons régler notre architecture de contrôle structurée, pour laquelle nous optimisons les matrices de gain $\boldsymbol{K}$, $\boldsymbol{H}$ et les paramètres de filtrage $n_1$, $n_0$,  $d_2$,  $d_1$,  $d_0$  (en jaune sur la Figure~\ref{fig:commande_int6DOF}). Comme indiqué dans \cite[eq. (2)]{ApkarianMulti}, nous résolvons le problème d'optimisation multiobjectifs, en exploitant l'implémentation Matlab bien expliquée dans \cite[\S 3]{ApkarianMulti}.

En particulier, sur la base d'un ensemble ${\mathcal W}$ comprenant une collection finie de couples $(w_x, w_z)$, avec $w_{x} \in [0,~8]~\SI{}{\meter\per\second}$ et $ w_{z} \in [\shortminus 4,~4]~\SI{}{\meter\per\second}$, nous considérons l'ensemble des systèmes linéarisés qui résulte de \eqref{eq:Pw_synthesis} et nous résolvons l'optimisation convexe suivante, où les scalaires $W_{1}$, $W_{2}$, $W_{3}$, $W_{4}$ et $W_{5}$ sont des facteurs de pondération à ajuster pour obtenir un compromis satisfaisant entre la robustesse (associée à $W_2$, $W_3$ et $W_4$) et la performance (associée à $W_1$ et $W_5$) :

\begin{align*} \label{eq:pb_optim_lpv}
\numberthis
\gamma^\star &= \min_{\boldsymbol{F}} \max_{w \in {\mathcal W}} 
\begin{vmatrix}
    \| W_{1} T_{\nu \rightarrow e}(\boldsymbol{P}_w,\boldsymbol{F})\|_{\infty} \\
    \|W_{2} T_{d \rightarrow u}(\boldsymbol{P}_w,\boldsymbol{F})\|_{\infty}\\
    \|W_{3} T_{\nu \rightarrow u}(\boldsymbol{P}_w,\boldsymbol{F})\|_{\infty}\\
    \|W_{4} T_{d \rightarrow y}(\boldsymbol{P}_w,\boldsymbol{F})\|_{\infty}\\
    \|W_{5} T_{w \rightarrow y}(\boldsymbol{P}_w,\boldsymbol{F})\|_{\infty}
    \end{vmatrix}_{\infty}, \text{ sous condition que } \\ 
    &\qquad \boldsymbol{F}
    \text{ stabilise en interne } {\mathcal F}_\ell (\boldsymbol{P}_w,\boldsymbol{F}), \forall w \in {\mathcal W},
\end{align*}
où ${\mathcal F}_\ell(\boldsymbol{P}_w,\boldsymbol{F})$ désigne l'interconnexion de bouclage linéaire de la Figure~\ref{fig:commande_int6DOF} pour une valeur spécifique de $w$ (ceci est cohérent avec la notation classique de contrôle robuste \cite{1576856,ApkarianMulti}). Nous notons que, par rapport à \cite[eq. (2)]{ApkarianMulti}, nous ne spécifions que des contraintes \textit{soft} et non des contraintes \textit{hard}.


\begin{algorithm}
  \caption{Réglage itératif et multimodèle des gains du contrôleur.}
  \label{alg:iterativeOptimisation}
  \hspace*{.1cm} \textbf{Entrées} : $\boldsymbol{A}_{w}$, $\boldsymbol{G}_{w}$, $\boldsymbol{E}_{w}$  les matrices de sortie de l'algorithme~\ref{alg:linea} et les scalaires de pondération positifs $W_1$--$W_5$\\
  \hspace*{.1cm} \textbf{Sorties} : $\boldsymbol{K}$, $\boldsymbol{H}$ et les gains du filtre
  \begin{algorithmic}[1]
   
    \State (Initialisation) Initialiser ${\mathcal W}$ comme un maillage comprenant toutes les paires $ w_{x} \in \{0,~-4,~-8\}$ et $ w_{z} \in \{\shortminus 4,~0,~4\}$
    \State \label{step:synthesis} (Synthèse) Résoudre l'optimisation \eqref{eq:pb_optim_lpv} avec le logiciel {\tt Systune}

    \State \label{step:analysis} (Analyse) Définir un maillage de validation ${\mathcal W}_{\text{v}}$ en discrétisant l'intervalle $(w_x,w_y) \in [0,8]\times[-4,4]$ avec un pas de discrétisation de $1$. En utilisant le contrôleur $\boldsymbol{F}$ obtenu à l'étape précédente et pour chaque $w_{\text{v}}\in {\mathcal W}_{\text{v}}$, nous calculons :
    \begin{align}
    \label{eq:validation_step}
    \gamma_{\text{v}} = \begin{vmatrix}
    \| W_{1} T_{\nu \rightarrow e}(\boldsymbol{P}_{w_{\text{v}}},\boldsymbol{F})\|_{\infty} \\
    \|W_{2} T_{d \rightarrow u}(\boldsymbol{P}_{w_{\text{v}}},\boldsymbol{F})\|_{\infty}\\
    \|W_{3} T_{\nu \rightarrow u}(\boldsymbol{P}_{w_{\text{v}}},\boldsymbol{F})\|_{\infty}\\
    \|W_{4} T_{d \rightarrow y}(\boldsymbol{P}_{w_{\text{v}}},\boldsymbol{F})\|_{\infty}\\
    \|W_{5} T_{w \rightarrow y}(\boldsymbol{P}_{w_{\text{v}}},\boldsymbol{F})\|_{\infty}
    \end{vmatrix}_{\infty}.
    \end{align}
    Augmenter  ${\mathcal W}$ avec le point correspondant si $\gamma_{\text{v}} > 1$ ou si $\gamma_{\text{v}}$ est indéfini (à savoir si $\boldsymbol{F}$ n'est pas stabilisant).


    \State (Conclusion) Si ${\mathcal W}$ n'a pas été augmenté à l'étape précédente, passer à l'étape~\ref{step:final}, sinon passer à l'étape~\ref{step:synthesis}.
    
    \State \label{step:final} 
    \textbf{Retourne} : $\boldsymbol{K}$, $\boldsymbol{H}$ et les paramètres du filtre $n_1$, $n_0$,  $d_2$,  $d_1$,  $d_0$

  \end{algorithmic}
\end{algorithm}

Le problème d'optimisation \eqref{eq:pb_optim_lpv} devient de plus en plus lourd d'un point de vue informatique, à mesure que nous augmentons la cardinalité de l'ensemble des conditions de vent considérées dans ${\mathcal W}$. En fait, une approche de force brute incluant un maillage fin de points dans ${\mathcal W}$ conduit à une optimisation difficile à calculer. Au lieu de cela, nous suivons ici la procédure itérative décrite dans l'algorithme~\ref{alg:iterativeOptimisation}, où ${\mathcal W}$ est initialement sélectionné comme un maillage grossier comprenant $3 \times 3 = 9$ points (étape 1) ; une étape de synthèse (étape 2) est ensuite suivie de manière répétée par une étape d'analyse (simple du point de vue du calcul) (étape 3) où le contrôleur $\boldsymbol{F}$ est fixé.
L'étape 3 identifie les points de violation en utilisant un maillage de validation plus fin ${\mathcal W}_{\text{v}}$ et les ajoute à l'ensemble d'optimisation ${\mathcal W}_{\text{v}}$. L'algorithme se termine après quelques itérations, lorsque aucun point du maillage de validation ne viole les contraintes.

\begin{table}[ht]
    \centering
    \begin{tabular}{|l|c|c|c|c|c|} 
    \hline
    Pondération & $W_1$ & $W_2$ & $W_3$ & $W_4$ & $W_5$ \\ \hline
    Values &18 & 16 & 11 & 26 & 5 \\ \hline
    \end{tabular}
    \caption{\label{tab:W1W5} Valeurs des scalaires de pondération positifs $W_1$--$W_5$ utilisés dans l'exécution de l'Algorithme~\ref{alg:iterativeOptimisation}.}
\end{table}
L'exécution de l'algorithme~\ref{alg:iterativeOptimisation}, pour les modèles de DarkO issus des théorèmes~\ref{thm:eqs} et~\ref{th:lin} avec la sélection des scalaires de pondération positifs $W_1$--$W_5$ indiqués dans la Table \ref{tab:W1W5}, a donné la sélection suivante après 2 itérations :
\begin{align*}
 \left[\!\! \begin{array}{c|c} 
 \boldsymbol{K}^\top \!\!&  \boldsymbol{H}^\top \!\!
       \end{array} \right] \!&=\!
\left[\!\! \begin{array}{c|c} 
\begin{smallmatrix}
    -3.86&-3.86&0.79&0.79\\ 
1.43&-1.43&1.71&-1.71\\ 
4.06&4.06&-2.07&-2.07\\ 
-6.86&-6.86&-11.60&-11.60\\ 
-10.75&10.75&-1.89&1.89\\ 
27.20&27.20&-4.29&4.29\\ 
-12.32&12.32&-3.46&3.46\\  
-5.84&5.84&-2.29&2.29\\ 
-5.19&5.19&5.79&5.79\\ 
-6.52&6.52&0.08&-0.08\\ 
\end{smallmatrix}&
\begin{smallmatrix}
    0.02&0.48\\ 
    -0.47&-1.63\\ 
    -0.45&0.52\\ 
    -0.14&1.40\\ 
    3.35&5.69\\ 
    -1.84&3.79\\ 
    3.72&6.81\\ 
    1.58&3.13\\ 
    2.86&-1.54\\ 
    0.08&2.82\\ 
\end{smallmatrix}
\end{array} \right],\\
\left[\!\! \begin{array}{c|c} 
        n_1 &  n_0\\ \hline
        d_2 &  d_1\\\hline
        d_0 &  
       \end{array} \right] \!&=\!
       \left[\begin{array}{c|c} 
        \smallm{-429} & \smallm{-389}\\ \hline
        \smallm{1} &  \smallm{6475}\\ \hline
        \smallm{4905} &  
       \end{array}\right].
       \numberthis
       \label{eq:gain_selection}
\end{align*}

Après la première itération de l'algorithme~\ref{alg:iterativeOptimisation} et qu'un contrôleur candidat $\boldsymbol{F}$ ait été évalué à l'étape~\ref{step:synthesis}, nous pouvons tracer la Figure~\ref{fig:transferts_tcst}. Elle montre en bleu les diagrammes de bode des valeurs singulières maximales des matrices de transferts $T_{\nu \rightarrow e}$, $T_{d \rightarrow u}$, $T_{\nu \rightarrow u}$, $T_{d \rightarrow y}$, et $T_{w \rightarrow y}$ (associées à la valeur de $\gamma_{\text{v}}$) reportées dans \eqref{eq:validation_step} à l'étape d'analyse~\ref{step:analysis}, à comparer à l'inverse des cinq poids $W_1$--$W_5$, représenté par les lignes horizontales vertes. 
Les diagrammes en rouge correspondent aux points qui violent les contraintes et qui sont ajoutés à l'ensemble ${\mathcal W}$ pour l'itération suivante. Les quelques diagrammes en magenta, en revanche, correspondent aux 9 points considérés dans ${\mathcal W}$ pour la première itération de l'étape de synthèse~\ref{step:synthesis}.
Les diagrammes rouges de la Fig.~\ref{fig:transferts_tcst} illustrent clairement que l'algorithme itératif parvient à détecter les valeurs critiques de la vitesse du vent $(w_x,w_z)$ que nous ajoutons à l'ensemble d'optimisation ${\mathcal W}$.

Les valeurs singulières de la fonction de sensibilité de la sortie et de l'entrée (respectivement $T_{r \rightarrow e}$ et $T_{d \rightarrow u}$) sont représentées sur la Fig.~\ref{fig:transferts_tcst} ligne supérieure. Le graphique de la troisième ligne représente la valeur singulière du transfert entre la perturbation du vent $\boldsymbol{w}$ et la sortie du drone $\boldsymbol{y}$. La valeur singulière qui tangente la contrainte est celle de la condition de vent la plus élevée du modèle de synthèse, $(w_x, w_z) = (-8,-4)~\SI{}{\meter\per\second}$.

\begin{figure}[ht!]
    \centering
    \includegraphics[trim=0cm 0cm 0cm 0cm,clip,width=0.6\columnwidth]{figures/transferts_tcst.png}
    \caption{Diagrammes des valeurs singulières des fonctions de transfert dans \eqref{eq:validation_step}, à la première itération de l'algorithme~\ref{alg:iterativeOptimisation}.}
    \label{fig:transferts_tcst}
\end{figure}


\begin{figure}[ht!]
    \centering
    \includegraphics[trim=0cm 0cm 0cm 0cm,clip,width=0.6\columnwidth]{figures/sim_systune_lpv_noise.png}
    \caption{Simulation du modèle non linéaire \eqref{eq:dyna_orig} (ligne continue) et du modèle linéarisé \eqref{eq:lpv_linearisation} (ligne en pointillé) avec des incréments de vent constants croissants, le contrôleur étant réglé à l'aide de l'optimisation multimodèle de l'algorithme~\ref{alg:iterativeOptimisation} dans la section~\ref{sec:h_inf6DOF_multi}.}
    \label{fig:SimSytuneStruct_lpv}
\end{figure}

Avec le réglage indiqué dans \eqref{eq:gain_selection}, tel qu'il est obtenu avec l'Algorithme~\ref{alg:iterativeOptimisation}, nous présentons dans la Figure~\ref{fig:SimSytuneStruct_lpv} des résultats de simulation similaires à ceux déjà présentés dans la Figure.~\ref{fig:SimSytuneStruct_zero} pour la méthode de réglage sans vent discutée dans la Section~\ref{sec:zerowind}. Une fois de plus, nous simulons à la fois le modèle non linéaire \eqref{eq:dyna_orig} (lignes pleines) et le modèle linéarisé \eqref{eq:lpv_linearisation} (ligne en pointillé). 

Par rapport à la Fig.~\ref{fig:SimSytuneStruct_zero}, les simulations de la Figure~\ref{fig:SimSytuneStruct_lpv} montrent que le réglage du contrôleur basé sur les Théorèmes~\ref{thm:eqs} et~\ref{th:lin} résout les problèmes d'instabilité et parvient à stabiliser le vol stationnaire dans tous les scénarios de vent considérés. Nous notons également que la Figure.~\ref{fig:SimSytuneStruct_lpv} montre une action plus agressive. En effet, l'entrée de contrôle $u$ (à la fois la poussée et les déflexions) est plus affectée par le bruit de mesure.

L'efficacité du schéma de contrôle réglé sur la base de l'Algorithme~\ref{alg:iterativeOptimisation} est également confirmée par les résultats expérimentaux présentés dans la section suivante.

\section{Rafale de vent}

Dans le chapitre \ref{sec:perturbation}, nous avons présenté des modèles de perturbation représentant une modélisation d'une rafale de vent. Nous allons les utiliser pour évaluer la réponse du drone face à cette perturbation. 


Nous avons appliqué un vent horizontal $w_x$ avec trois valeurs moyennes $w_{m} \in {0,3,6}\SI{}{\meter\per\second}$ et avec les caractéristiques suivantes pour la fonction "Chapeau mexicain" $A_g=\SI{-5}{\meter\per\second}$, $f_g = \frac{1}{T_g} = \SI{0.2}{\Hz}$ soit une période $T_g$ de 5 secondes. Les résultats sont visibles sur la figure \ref{fig:sim_mex0_2}.
\begin{figure}[ht!]
    \centering
    \resizebox{.99\textwidth}{!}{%
    \includegraphics[width=0.33\textwidth]{figures/sim_mex_hat5msT5.png}
    \quad
    \includegraphics[width=0.33\textwidth]{figures/sim_mex_hat8msT5.png}
    \quad
    \includegraphics[width=0.33\textwidth]{figures/sim_mex_hat11msT5.png}
    }
    \caption{Simulation du modèle non linéaire \eqref{eq:dyna_orig} face à une perturbation "Chapeau mexicain" \eqref{eq:mex} avec $f_g = \SI{0.2}{\Hz}$.}
    \label{fig:sim_mex0_2}
\end{figure}


Nous avons aussi réalisé une simulation similaire à la précédente avec une fréquence égale à $f_g = \SI{1.2}{\Hz}$ soit une période $T_g$ de 0.83 seconde. Cette valeur est utilisée dans \cite{Gillebaart2014} pour représenter une rafale. Les résultats sont visibles sur la figure \ref{fig:sim_mex1_2}.
\begin{figure}[h]
    \centering
    \resizebox{.99\textwidth}{!}{%
    \includegraphics[width=0.33\textwidth]{figures/sim_mex_hat5msF1.2.png}
    \quad
    \includegraphics[width=0.33\textwidth]{figures/sim_mex_hat8msF1.2.png}
    \quad
    \includegraphics[width=0.33\textwidth]{figures/sim_mex_hat11msF1.2.png}
    }
    \caption{Simulation du modèle non linéaire \eqref{eq:dyna_orig} face à une perturbation "Chapeau mexicain" \eqref{eq:mex} avec $f_g = \SI{1.2}{\Hz}$.}
    \label{fig:sim_mex1_2}
\end{figure}


Enfin nous avons réalisé une simulation avec un vent horizontal $w_x$ avec trois valeurs moyennes $w_{m} \in {0,3,6}\SI{}{\meter\per\second}$ et avec la caractéristique suivante pour la fonction "ondelettes de Morlet" $A_g=\SI{-5}{\meter\per\second}$. Les résultats sont visibles sur la figure \ref{fig:sim_morlet}.

\begin{figure}[h]
    \centering
    \resizebox{.99\textwidth}{!}{%
    \includegraphics[width=0.33\textwidth]{figures/sim_mor5ms.png}
    \quad
    \includegraphics[width=0.33\textwidth]{figures/sim_mor8ms.png}
    \quad
    \includegraphics[width=0.33\textwidth]{figures/sim_mor11ms.png}
    }
    \caption{Simulation du modèle non linéaire \eqref{eq:dyna_orig} face à une perturbation "ondelettes de Morlet" \eqref{eq:morlet}.}
    \label{fig:sim_morlet}
\end{figure}
Nous observons que le drone et son contrôleur sont en mesure de rejeter les différentes perturbations.
Nous observons sur l'ensemble des simulations que le drone lutte contre le vent incident en modifiant son incidence, il s'aligne dans le flux du vent, ce qui à pour impact majeur de faire prendre de l'altitude au drone.


\section{Vol expérimental en soufflerie ouverte} 
\label{sec:exp6DOF}

Les vols expérimentaux de DarkO se sont déroulés dans un espace dédié (voir Fig.~\ref{fig:flight_windshape}) avec un système de localisation Optitrack basé sur une convention NED selon la Figure~\ref{fig:darko2}. Nous avons utilisé un générateur de vent à veine ouverte pour obtenir des incréments de vent que nous avons mesurés à l'aide d'une sonde à fil chaud (la barre verticale dans la Fig.~\ref{fig:flight_windshape}). 
\begin{figure}[ht!]
    \centering
    \includegraphics[trim=0cm 0cm 0cm 0cm,clip,width=0.6\columnwidth]{figures/img_flight_darko.png}
    \caption{Vol expérimental de DarkO devant la soufflerie ouverte.}
    \label{fig:flight_windshape}
\end{figure}
Bien que cette information sur le vent soit enregistrée à bord du drone pour synchroniser les données, nous n'utilisons pas cette mesure dans la loi de commande. La fréquence de mesure de cette sonde de vent n'est que de 0,5 Hz, de sorte que nous n'avons qu'une mesure toutes les deux secondes. 
L'estimation de l'état est effectuée à l'aide d'un système de navigation inertielle pour fusionner les données de l'unité de mesure inertielle (IMU) et du système de localisation Optitrack afin d'obtenir une estimation précise de la sortie $\boldsymbol{y}$ dans la Fig.~\ref{fig:commande_int6DOF}. Cependant, la vitesse angulaire du drone $\boldsymbol{\omega}_{text{b}}$ est mesurée sur la base du gyromètre de l'IMU, qui fournit des mesures bruitées. Nous avons donc ajouté un filtre passe-bas \textit{Butterworth} de second ordre avec une fréquence de coupure de 20 Hz pour lisser la sortie $\boldsymbol{\omega}_{text{b}}$. Le filtre de \textit{Butterworth} est pris en compte dans la dynamique linéarisée lors de l'optimisation des gains du contrôleur en suivant l'Algorithme~\ref{alg:iterativeOptimisation}. 

\begin{figure}[ht!]
    \centering
    \includegraphics[trim=0cm 0cm 0cm 0cm,clip,width=0.6\columnwidth]{figures/exp_systune_struct.png}
    %'/home/florian/Log/DarkoLog/09_11_23/23_11_05__06_40_06_SD.data'
    \caption{Expérience du drone DarkO devant la soufflerie avec des incréments de vent constants croissants (graphique du bas).}
    \label{fig:ExpSytuneStruct}
\end{figure}

Nous avons également utilisé les ESC similaires à ceux utilisés lors de l'identification montrée dans la Figure~\ref{fig:IOmot} pour l'actionnement des hélices. Les deux ESC ont été flashés avec le code open-source AM32 (voir Annexe \ref{sec:AM32}). De cette manière, nous compensons les effets de décharge de la batterie et obtenons un suivi précis de la vitesse commandée. Avant cette modification, l'action intégrale de la rétroaction stabilisatrice de la Fig.~\ref{fig:commande_int6DOF} compensait la perte de vitesse du moteur causée par la réduction de la tension de la batterie pendant le vol. Cette compensation intégrale a été indirectement générée par la perte d'altitude du drone causée par la réduction de la traction. 

Nous avons réalisé une expérience de vol au cours de laquelle DarkO a été mis manuellement en mode de vol stationnaire stabilisé devant la soufflerie, puis nous avons activé la loi de contrôle de l'Algorithme~\ref{alg:iterativeOptimisation}. Comme le drone devait être stabilisé à au moins \SI{30}{\centi\meter} de la soufflerie, un pilotage manuel a été réalisé pour éviter tout dépassement qui pourrait endommager la soufflerie. Une fois que DarkO était suffisamment proche du point de consigne $\boldsymbol{r}_{p}$ de la Fig.~\ref{fig:commande_int6DOF}, nous avons activé le contrôleur proposé, obtenant les résultats de la Fig.~\ref{fig:ExpSytuneStruct}. Au cours de la phase d'expérimentation, comme le montre le graphique inférieur de la Fig.~\ref{fig:ExpSytuneStruct}, nous avons augmenté progressivement la vitesse du vent, en attendant 20 secondes entre chaque évolution de vent jusqu'à une vitesse finale de \SI{7}{\meter\per\second}.


\begin{figure}[ht!]
    \centering
    \includegraphics[trim=0cm 0cm 0cm 0cm,clip,width=0.6\columnwidth]{figures/boxplot.png}
    \caption{Visualisation statistique des performances de vol stationnaire.}
    \label{fig:statpos}
\end{figure}

Les figures~\ref{fig:ExpSytuneStruct} et~\ref{fig:statpos} montrent que le drone maintient sa position malgré l'augmentation de la vitesse du vent. Nous pouvons noter quelques points importants, en accord avec les simulations : la traction du moteur diminue lorsque la vitesse du vent augmente. Le schéma de contrôle profite de la portance générée par le vent pour soutenir le drone, de sorte que moins d'énergie soit nécessaire pour stabiliser la position en vol stationnaire. Le drone maintient son angle d'inclinaison à une valeur inconnue, a priori, pour la loi de commande et qui découle naturellement de l'action intégrale. Cette valeur est atteinte asymptotiquement et converge vers la valeur requise de $\theta$. Pour stabiliser la position, le drone utilise les élevons pour annuler le moment de tangage généré par la forme de l'aile, soumise à un vent horizontal, sans atteindre les limites de saturation.
On note également une légère asymétrie de l'efficacité des actionneurs, qui est efficacement compensée par l'action proportionnelle du schéma de contrôle.


\section{Conclusion du Chapitre \ref{chap:6DOF}}
Nous avons proposé un schéma de contrôle pour la stabilisation d'un \textit{tailsitter} lors d'un vol stationnaire en présence d'un vent constant inconnu. Notre bouclage contient une action intégrale et ne nécessite pas la mesure de la vitesse du vent. Les modèles paramétriques linéarisés se sont avérés être un instrument clé pour effectuer le réglage des paramètres du contrôleur.
Après avoir étudié les résultats de la simulation, nous avons mené une campagne de vols expérimentaux dans un environnement contrôlé pour valider notre solution de contrôle. 

La principale limitation de cette architecture réside dans l'impossibilité de mesurer le vent dans les phases de vol stationnaire ou de vitesse faible. Pour tenter de résoudre ce problème, nous avons proposé une nouvelle architecture.



% \chapter{Commande d'un drone à aile libre rotation libre}
\minitoc










% \chapter{Commande d'un drone à aile libre rotation libre}
\minitoc

\section{Inversion non linéaire incrémentale de la dynamique du drone}

\section{Commande Udwadia-Kalaba}

\section{Vols expérimentaux}







% LTeX: enabled=true
\chapter*{Conclusion}
\addstarredchapter{Conclusion} 

\renewcommand{\thefigure}{C.\arabic{figure}}
\setcounter{figure}{0} % Réinitialiser le compteur à 0

\markboth{Conclusion}{Conclusion}
Cette thèse est consacrée à la modélisation, à l'étude et au contrôle d'un drone convertible sujet à des forces aérodynamiques, des couplages entre les actionneurs et des dynamiques non-linéaires. Elle propose, au travers de l'utilisation d'un modèle unifié représentant les forces aérodynamiques sur l'ensemble du domaine de vol, d'analyser le comportement d'un \textit{tailsitter} et de proposer des méthodes de commande. Notre travail a débouché sur la proposition d'une nouvelle architecture de type \textit{freewing}.

La première partie du manuscrit propose un rapide aperçu des architectures de drones convertibles avant de se focaliser sur les \textit{tailsitters} et les \textit{freewings}. Nous avons développé les principales caractéristiques d'actionnement, le comportement des drones ainsi que les méthodes de modélisation. Ce chapitre introductif a permis de développer l'architecture conventionnelle de commande, ainsi qu'un tour d'horizon des méthodes de commande et d'optimisation des contrôleurs. Notre travail ayant une forte composante expérimentale, une vision de l'architecture nécessaire à ces expérimentations a été évoquée avant d'être développée plus en profondeur dans l'Annexe.  

Nous avons utilisé un modèle de la littérature, sans singularité sur l'ensemble du domaine de vol, ne faisant pas appel aux angles aérodynamiques $\alpha$ et $\beta$, appelé $\phi$-théorie. Ce modèle mathématique n'a d'utilité pratique que s'il est cohérent avec la réalité, ce qui a pu être démontré dans la littérature et qui est confirmé par nos travaux. De ce modèle non-linéaire, nous avons pu extraire des caractéristiques intéressantes pour les drones à décollage et atterrissage vertical. 
Nous avons caractérisé l'ensemble des points d'équilibre avec ou sans vent pour un \textit{tailsitter}. De ces équilibres, nous avons extrait la dynamique linéarisée, point de départ de la conception de toute loi de commande linéaire. Notre compréhension du comportement du drone a été augmentée par ces résultats qui nous informent sur la commandabilité du drone, sur les marges vis-à-vis des saturations et sur la capture du comportement du drone par une linéarisation autour des points d'équilibre. Nous avons donc pu valider la précision des linéarisations face aux nombreuses non-linéarités du modèle. Pour cela, nous avons effectué des simulations en boucle fermée, au vu du comportement instable du drone, du modèle linéaire et non-linéaire.

Un travail préliminaire a permis de proposer une architecture de commande hybride avec un mécanisme d'hystérésis, basée sur une variable discrète sélectionnant la loi de commande la plus appropriée en fonction de la phase de vol. Les deux lois proposées dans ce cas sont une loi non-linéaire basée sur une direction de zéro-moment et une loi linéaire LQR. Cette loi LQR est optimisée grâce au modèle obtenu précédemment.

De ce travail et à l'aide de la linéarisation, nous avons observé un comportement intéressant pour le rejet de perturbations de vent sur un \textit{tailsitter}. Ce comportement repose sur le changement de l'angle de tangage du drone pour compenser l'augmentation de la vitesse air qui engendre un déplacement du drone. Nous avons donc expérimenté, à l'aide d'une maquette à trois degrés de liberté, une loi de commande proportionnelle intégrale. Cette maquette, utilisant une architecture physique associée à un modèle de dynamique transactionnelle simulée, a permis de valider l'architecture de commande ainsi que son optimisation basée sur des contraintes $H_{\infty}$.
Bien que les résultats obtenus soient prometteurs, nous avons étudié une méthode différente d'obtention des gains du contrôleur PI. Cette méthode, plus conservative, est basée sur une résolution successive de LMI. Les résultats ont pu être évalués sur l'architecture complète du drone, par un vol expérimental en volière.

Cette expérimentation a permis d'identifier des problèmes de sensibilité de la boucle fermée aux dynamiques non modélisées et aux bruits. Nous avons donc proposé une extension du contrôleur PI pour augmenter sa robustesse. Une expérimentation face à un vent croissant par palier a validé notre travail.

Nos travaux nous ont amené à vouloir installer un capteur de vent sur le drone pour pouvoir utiliser la mesure pour la transition. Toutefois, le corps du \textit{tailsitter} étant en rotation lors de la transition, nous ne pouvions pas fixer le capteur de manière satisfaisante. Nous avons donc étudié et développé une architecture \textit{freewing} procurant un fuselage maintenu horizontal permettant d'installer n'importe quel capteur ou charge utile. L'aile étant en rotation libre autour du fuselage, nous conservons de nombreuses propriétés des \textit{tailsitters}. Dans cette démarche, nous avons modélisé le drone avec une dynamique multicorps, identifié les paramètres, fabriqué la maquette et réalisé des vols expérimentaux à l'aide de l'INDI.

\section*{Limite de l'étude}
Les travaux préliminaires, menés au chapitre \ref{chap:hybrid}, ne sont que des résultats de simulation. Il serait souhaitable de réaliser des expérimentations du contrôleur non-linéaire basé sur une direction de zéro-moment ainsi que de son utilisation dans l'architecture hybride avec une transition entre le contrôleur basé sur une direction de zéro-moment et le contrôleur PI étendu développé au chapitre \ref{chap:6DOF}.

Bien que nous souhaitions utiliser la mesure du vent, notre travail n'a pu aboutir, étant donné la richesse des questions que nous avons souhaité développer en amont et par le temps nécessaire au développement de l'architecture nécessaire.
De plus, de nombreuses architectures auraient pu répondre au besoin. Nous avons choisi de nous concentrer sur une architecture inspirée du \textit{tailsitter} DarkO car nous avions de l'expérience dans sa modélisation et sa dynamique.

Tous nos résultats ont été expérimentés dans une atmosphère contrôlée avec un générateur de perturbations. Il serait maintenant intéressant d'évaluer la précision de nos contrôleurs en extérieur. Ce travail possède une double complexité car le drone évoluerait dans un environnement plus turbulent, mais aussi avec une estimation d'état moins précise. Effectivement, en intérieur, nous avons accès à un système de positionnement millimétrique alors qu'en extérieur, les GPS ne peuvent nous fournir une information de position aussi précise.


\section*{Travaux futurs}

La modélisation de Udwadia-Kalaba permet d'obtenir un modèle d'un drone multicorps. Il serait intéressant d'utiliser ces travaux pour concevoir un contrôleur assurant la stabilisation de l'aile et du fuselage en prenant en compte leurs interactions.

De plus, ce contrôleur pourrait profiter de la mesure du vent réalisée par une sonde cinq trous installée sur le fuselage. Comme le fuselage est maintenu horizontal et avec une faible variation d'incidence, la mesure sera disponible dans toutes les phases de vol. Il est nécessaire de caractériser la sonde. Effectivement, la mesure étant réalisée par une différence de pression statique et dynamique, nous devons donc étudier la sensibilité des capteurs pour déterminer la plus petite variation de vent mesurable ainsi que le temps nécessaire à la mesure.

Enfin, demeure un sujet de recherche, non abordé à ce jour : si les drones convertibles possèdent deux modes de déplacement (stationnaire et vol d'avancement), il conviendra de s'interroger sur la stratégie de vol à adopter, en fonction des caractéristiques de la mission.

En effet, en fonction de la distance entre deux points de l'espace, il s'agit de choisir entre :
\begin{itemize}
    \item effectuer une transition du mode stationnaire vers le mode d'avancement (plus efficace énergétiquement)
    \item ou rester en mode stationnaire et se déplacer de proche en proche (très énergivore).
\end{itemize}

Si le mode d'avancement est énergétiquement intéressant, il implique deux transitions et la réalisation d'un cercle (nécessaire pour de courtes distances et pour avoir le temps de réaliser les transitions).

Ce problème est schématisé sur la Figure \ref{fig:pbhybride}.



\begin{figure}[ht!]
    \centerline{
    \includegraphics[trim=0cm 0cm 0cm 0cm,clip,width=0.8\columnwidth]{figures/DroneConvertibleGuidage.png}}
    \caption{Schéma de déplacement d'un drone convertible.}
    \label{fig:pbhybride}
\end{figure}





%%%%%%%% 6. APPENDIX %%%%%%%%
\appendix

\chapter{Annexe technique sur les drones}
\label{chap:annexe1}

\section{Système de drone : Paparazzi}

Un drone est composé de plusieurs pièces assemblées entre elles pour former la structure sur laquelle sont fixés des actionneurs, un autopilote et une charge utile (colis, caméra, capteur, etc.). L'élément central est l'autopilote qui assure la communication entre tous les éléments. Nous pouvons décomposer l'autopilote en deux parties : la partie matérielle et la partie logicielle.
\nomenclature[]{\(PCB\)}{Circuit imprimé  \textit{Printed Circuit Board}}
La partie matérielle est constituée d'un circuit imprimé (PCB) sur lequel des composants sont installés pour assurer les tâches relatives au vol. La partie logicielle se décompose en deux éléments : le segment sol et le logiciel embarqué \ref{sec:logiciel}.

Nous pouvons détailler les capteurs embarqués et le microcontrôleur avec l'ensemble de ses ports de communication \ref{sec:capteurs} et \ref{sec:micoctrl}. 

 \subsection{Les capteurs d'un autopilote}
 \label{sec:capteurs}
 Un autopilote comporte généralement un accéléromètre, un gyroscope, un magnétomètre et un baromètre.
 
 \paragraph*{}
 \textbf{L'accéléromètre} à trois axes permet de mesurer l'ensemble des forces appliquées sur le véhicule, à l'exception du poids. Il est possible d'obtenir la position du drone par double intégration de la mesure de l'accéléromètre. Toutefois, il convient de souligner que la position dérive rapidement en raison des bruits de mesure.

 \paragraph*{}
 \textbf{Le gyroscope} à trois axes permet de mesurer les vitesses de rotation du véhicule. Il est possible d'obtenir l'orientation du drone par intégration de la mesure du gyroscope. Toutefois, comme précédemment, l'orientation dérive rapidement en raison des bruits de mesure.

 \paragraph*{}
 \textbf{Le magnétomètre} à trois axes indique la direction du nord magnétique. Il permet de se diriger par rapport à une référence connue. Le principal inconvénient de ce capteur est sa perturbation par les masses magnétiques environnantes, ainsi que par les champs magnétiques parasites induits par la proximité des moteurs électriques par exemple. Il est donc difficile de les utiliser à l'intérieur d'un bâtiment. L'influence magnétique de l'engin porteur et les perturbations dues à d'éventuels moteurs électriques peuvent être éliminées en qualifiant, de manière statique, les erreurs dues aux masses métalliques du véhicule et aux moteurs électriques (en fonction des tensions et courants d'alimentation).

 \paragraph*{}
 \textbf{Le baromètre} est un capteur d'altitude basée sur la mesure de la pression atmosphérique.
 \todo{info sur le baro}


 \paragraph*{}
 \textbf{Le GPS} est monté en extérieur de l'autopilote. Ce système de géopositionnement par satellite (\textit{ Global Positioning System}) \nomenclature[]{\(GPS\)}{Géo-positionnement par satellite (\textit{Global Positioning System})} permet d'obtenir un positionnement absolu du drone. 


 \paragraph*{}
 Il est courant de retrouver plusieurs capteurs dans un même boitier, que l'on nomme centrale inertielle (Inertial Measurement Units, IMU), \nomenclature[]{\(IMU\)}{Centrales inertielles (\textit{Inertial Measurement Units})}. Ces dernières sont composées au minimum d'un accéléromètre 3-axes et d'un gyroscope 3-axes, mais il est courant de les trouver avec un magnétomètre 3-axes. 

 \subsection{Le microcontrôleur d'un autopilote}
 \label{sec:micoctrl}
 Le microcontrôleur (Microcontroller Unit, MCU) \nomenclature[]{\(MCU\)}{Microcontrôleurs (\textit{Microcontroller Unit})} est la pièce maitresse de l'autopilote en ce qu'il permet d'effectuer l'ensemble des traitements nécessaires à la conduite du vol.

 De plus il possède plusieurs ports de communication pour récupérer les données des capteurs ou envoyer des ordres aux actionneurs.


 Nous pouvons citer le \textit{Dshot} qui est un protocole de communication défini entre l'autopilote et l'ESC pour envoyer les commandes des moteurs. Les avancées sur ce protocole ont notamment permis la communication bidirectionnelle, permettant d'obtenir la vitesse des moteurs, leur consommation et d'autres informations.

 \todo{Autre protocole can, serial, I2c }

 \subsection{Évolutions}
 Les nombreux progrès dans les systèmes d'estimation état permettent de connaître précisément l'orientation et la position des drones pour assurer la stabilisation, le guidage et la navigation. Les progrès sont liés à l'amélioration continue des capteurs, notamment des centrales inertielles constituées d'un accéléromètre, d'un gyroscope et d'un magnétomètre.

La Table \ref{tab:autopilote_ev} montre l'évolution des vitesses des microcontrôleurs (Microcontroller Unit, MCU) \nomenclature[]{\(MCU\)}{Microcontrôleurs (\textit{Microcontroller Unit})} embarqués sur les autopilotes et de la réduction du bruit des capteurs inertiels.
\begin{table}[ht]
    \centering
    \begin{tabular}{|c|c|c|c|c|c|}
        \hline
        Type & Date & MCU & Vitesse & Capteur  & Bruit RMS \\
        \hline \hline
        \href{https://wiki.paparazziuav.org/wiki/Apogee/v1.00}{Apogee}  & 2013 & STM32F4 & 168 MHz & MPU-9150 & \begin{tabular}{ccc} Gyro : 0.06 dps \\
        Accel: 4 mg  \end{tabular}  \\
        \hline
        \href{https://wiki.paparazziuav.org/wiki/Chimera/v1.00}{Chimera} & 2016 & STM32F7 & 216 MHz &  MPU-9250 & \begin{tabular}{ccc} Gyro : 0.1  dps \\
        Accel: 8 mg  \end{tabular}\\
        \hline
        \href{https://wiki.paparazziuav.org/wiki/Tawaki/v1.10}{Tawaki 1} &2019 &  STM32F7 & 216 MHz  & ICM-20600 & \begin{tabular}{ccc} Gyro : 0.04 dps \\
        Accel: 1 mg  \end{tabular}\\
        \hline
        \href{https://wiki.paparazziuav.org/wiki/Tawaki/v2.01}{Tawaki 2} &2023 &  STM32H7 & 480 MHz & ICM-42688-P & \begin{tabular}{ccc} Gyro : 0.028 dps \\
        Accel: 0.70 mg  \end{tabular} \\
        \hline
    \end{tabular}
    \caption{Évolution des autopilotes Paparazzi sur dix ans.}
    \label{tab:autopilote_ev}
\end{table}

Sur une période de dix ans, nous pouvons observer que les microcontrôleurs ont doublé leur vitesse d'exécution, que les fabricants ont divisé par deux le bruit moyen sur les gyroscopes et par quatre le bruit moyen des accéléromètres.
Ces évolutions continues permettent une amélioration de l'estimation du drone utilisée pour la stabilisation. Il en résulte une stabilité accrue et de nouvelles possibilités pour la commande des drones.

 \subsection{Les logiciels d'un autopilote}
 \label{sec:logiciel}
 Tout le fonctionnement d'un drone repose sur le logiciel qui permet de le faire voler. Il se décompose en deux catégories : la partie sol et la partie embarquée.

 \subsection{Le segment sol}
\todo{remplir}
 \subsection{Le logiciel embarqué}
 Le logiciel embarqué 

 fusion de donnée 
 estimation d'état 

 \begin{figure}[ht!]
    \centerline{
    \includegraphics[trim=0cm 0cm 0cm 0cm,clip,width=1\columnwidth]{figures/PPRZ_Main_ap_loop.png}}
    \caption{Todo.}
    \label{fig:schedulingpaparazzi}
\end{figure}


\section{AM32}
\label{sec:AM32}

%%%%%%%% 7. BIBLIOGRAPHIE %%%%%%%%
\bibliographystyle{StyleThese}
% \bibliographystyle{plain}
\bibliography{biblio}


%%%%%%%% 8. LAST PAGE %%%%%%%%
\cleardoublepage
\pagestyle{empty} % remove headers on 2e page
\newgeometry{left=1.5in,right=1.3in,top=1.1in,bottom=1.1in} % remove header/footer space

% French
\begin{vcenterpage} % note: vcenterpage is not needed for long abstrat
\noindent\rule[2pt]{\textwidth}{0.5pt}

{\large\textbf{Résumé :}}
Les drones sont aujourd'hui devenus un outil dans de nombreux domaines tels que l'inspection, la surveillance ou la maintenance. Cependant, ils souffrent d'une autonomie limitée. Les \textit{tailsitters} apportent une solution grâce à leur grande enveloppe de vol et à leur efficacité énergétique. Toutefois, les \textit{tailsitters} sont grandement sujets aux perturbations aérologiques et notamment aux turbulences dans les phases stationnaires principalement. Cela est dû à la grande surface d'aile verticale, laquelle possède une grande prise au vent. De plus, leur corps tournant lors de la transition, il est donc compliqué de mesurer la vitesse de l'air.  Ainsi, en stationnaire ou à faible vitesse, le vent n'est pas connu. Ce type de drone est sous-actionné puisque l'on trouve deux moteurs sur l'aile et deux surfaces aérodynamiques sur le bord de fuite. Le flux d'air des hélices soufflant les élevons, nous avons un couplage entre les actionneurs.

Cette thèse cherche à étudier la commande de drones dans des environnements perturbés ou en présence de vent. Les premiers travaux se sont concentrés sur la dynamique sans vent pour appréhender une dynamique simplifiée. Nous avons pu proposer une modification non-linéaire du vecteur de commande pour rendre ce modèle linéaire en commande. De ce modèle, nous avons proposé une loi de commande locale-globale fondée sur une dynamique hybride à hystérésis. Elle permet d'étendre le domaine de stabilité de la loi de commande linéaire agressive à l'aide d'une loi non-linéaire avec une grande région d'attraction, mais moins agressive.

La suite des travaux s'est concentrée sur la stabilisation d'un \textit{tailsitters} soumis à des échelons de vent. Il en résulte une caractérisation des équilibres stationnaires pour un ensemble de conditions de vent et l'obtention de la représentation linéarisée de la dynamique du drone. À l'aide de ce modèle, il a été possible d'analyser les saturations des actionneurs et l'autorité disponible aux environs des points d'équilibre. Nous avons réalisé une stabilisation établie sur un retour de sortie, avec une action proportionnelle et intégrale. Cette commande n'utilise pas la mesure de l'angle de tangage du drone, car nous ne pouvons pas, a priori, connaître la valeur cible qui nécessiterait une estimation de la vitesse et de la direction du vent. L'optimisation de ce bouclage est effectuée à l'aide du logiciel "Systune" pour obtenir de bonnes propriétés de réjection de perturbation. Une approche incrémentale a été suivie, la loi de commande ayant été testée dans un premier temps sur une maquette à un degré de liberté face à une soufflerie ouverte. Une fois validée, la loi de commande a été implémentée dans le système de drone Paparazzi. Grâce à son architecture modulaire, il a été possible de nous interfacer avec les codes d'estimation et de commande des actionneurs. Ainsi, nous avons pu réaliser des vols sur le modèle complet à six degrés de liberté.

Enfin, nous avons proposé une architecture inspirée du \textit{tailsitter}, nommée \textit{freewing}. Nous avons développé un drone multicorps basé sur une aile en rotation libre sur son axe de tangage autour d'un fuselage. L'actionnement de l'aile est sensiblement le même que pour le \textit{tailsitters} et le fuselage possède deux actionneurs pour se maintenir horizontal. Nous recherchons, dans cette architecture, une passivité naturelle à la turbulence induite par le changement naturel de l'incidence de l'aile en fonction du vent incident. Il s'agit aussi d'installer une charge utile sur le fuselage horizontal sur le domaine de vol. De plus, nous avons réalisé un modèle de simulation où la dynamique est obtenue à l'aide des équations de Udwadia-Kalaba et de la phi-théorie. Enfin, nous nous sommes concentrés sur la stabilisation et le guidage du drone en utilisant une inversion incrémentale non-linéaire de la dynamique (INDI). Nous utilisons les actionneurs de l'aile et du fuselage pour obtenir une loi de stabilisation globale. Des vols ont validé l'intérêt de cette architecture.

{\large\textbf{Mots clés :}}
mots, clefs
\todo{Mots clés}

\noindent\rule[2pt]{\textwidth}{0.5pt}
\end{vcenterpage}

% English
\newpage
\begin{vcenterpage}
\noindent\rule[2pt]{\textwidth}{0.5pt}
% LTeX: language=en
{\large\textbf{Abstract:}}
Drones have become a prevalent tool in numerous fields, including inspection, surveillance, and maintenance. However, one area where they are currently lacking is autonomy. Tailsitter offer a viable solution, combining large flight envelopes with energy efficiency. Nonetheless, tailsitter unmanned aerial vehicles are particularly susceptible to disturbances, specifically turbulence during hover phases. This is because of the large vertical surface area of the wing, which offers a high degree of wind resistance. Furthermore, the drone's body rotates during transition, making it difficult to accurately measure airspeed. It is not possible to accurately measure wind speed, whether the drone is hovering or moving at a low speed. Additionally, this underactuated drone, has two motors on the wing and two aerodynamic surfaces on the trailing edge. The interaction of the airflow from the propellers with the elevons results in a coupling between the actuators.

The aim of this thesis is to examine the control of drone in challenging environments, with a particular emphasis on the impact of wind. However, initial research focused on windless dynamics in order to gain a better understanding of the simplified dynamics. We were able to propose a non-linear modification of the control vector to transform this nonlinear model into an input-affine model. Based on this model, we proposed a local-global control law based on hybrid dynamics with hysteresis, which allowed us to extend the stability domain of an aggressive linear control law by means of a non-linear law with a large region of attraction, but which was less aggressive. 

Further work was conducted on the stabilization of a tailsitter drone under wind steps. This resulted in a characterization of stationary equilibria for a range of wind conditions and a linearized representation of the drone's dynamics. Using this model, we were able to analyze the saturation levels of actuators and the authority available close to the equilibrium points. We have implemented a stabilization system based on output feedback with proportional and integral action, derived from the model. This control does not utilize the drones pitch angle measurement, since the target value is not known a priori. It would require an estimation of wind speed and direction, which is not feasible. The loop is optimized using "Systune" software in order to achieve effective disturbance rejection properties. We adopted an incremental approach by initially evaluating the control law on a one-degree-of-freedom model against an open wind tunnel. Upon validation, we proceeded to implement the control law within the Paparazzi UAV system. Due to its modular design, we were able to establish a connection with the state estimation and actuators, enabling us to execute flights on the entire model with six degrees of freedom. 

We then proposed an architecture inspired by the tailsitter, called freewing. We have developed a multi-body drone that is based on a wing that can freely rotate on its pitch axis around a fuselage. The wing is driven the same way as the tailsitter, and the fuselage has a motor and an aerodynamic surface to keep it horizontal. With this architecture, we aim to achieve a natural passivity to turbulence induced by the natural change in wing incidence as a function of the incident wind. However, there is also the possibility of installing a payload on the horizontal fuselage. To obtain a simulation model, we have modeled the drone's dynamics using the Udwadia-Kalaba equations and the phi-theory. We focused on stabilization and guidance of the UAV through the use of incremental nonlinear dynamic inversion (INDI). The wing and fuselage actuators are used to achieve a global stabilization law. In order to evaluate the benefits of this architecture, we conducted flight tests.

{\large\textbf{Keywords:}}
key, words
\todo{Keywords}

\noindent\rule[2pt]{\textwidth}{0.5pt}
\end{vcenterpage}


\end{document}
