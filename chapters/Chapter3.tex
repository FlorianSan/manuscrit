\chapter{Commande hybride}
\minitoc


\section{Motivation}
Basée sur les capacité d'un \textit{tailsitter}, il est legitime de se poser la question du mode de vol utilise pour rejoindre un point. Effectivement, le drone a la possibilité de se déplacer en stationnaire ou bien en vol d'avancement.



We propose in this section two control design strategies for stabilizing a hovering position. The first one is inspired by the nonlinear stabilizer presented in \cite{2020e-MicCenZacFra} and provides a large region of attraction, and the second one is based on the linearized dynamics and allow for more effective gain tuning in the final approaching phase. The two controllers are united via a hybrid mechanism that allows retaining the steady-state performance of the linearized design with the large region of attraction guaranteed by the nonlinear design. Our solution is tested by simulating the full nonlinear model.

\begin{remark}\label{rem:control_inputs}
We emphasize that vector $\boldsymbol{u}$ in \eqref{eq:vector_u} corresponds to a non-invertible transformation of the actual DarkO actuators corresponding to $\boldsymbol{u}_{\text{act}} := [\omega_1,\; \omega_2,\; \delta_1,\; \delta_2]^\top$. Nevertheless, when imposing the saturation constraints discussed in Remark~\ref{rem:saturation}, it is possible to uniquely determine $\boldsymbol{u}_{\text{act}}$ from a desired value of $\boldsymbol{u}$ in \eqref{eq:vector_u}, because nonzero positive values of $\omega_1$ and $\omega_2$ can be determined from the first two components of $\boldsymbol{u}$, and then $\delta_1$ and $\delta_2$ are easily constructed from the last two components of $\boldsymbol{u}$. 
\end{remark}

% While the main focus of this paper is to provide a comprehensive study of the dynamics of the DarkO drone, its simplified model, its equilibria and its linearized dynamics, we illustrate in this section how the special form of the obtained nonlinear and linearized models allow designing effective nonlinear and linear feedback control solutions providing desirable behavior in a number of practical scenario. 


% The control that we propose in this paper is based on a hybrid mechanism. The interest of this is explained by a need to converge the UAV to the target position with a robust control when the UAV is far from the target but also a desire for faster convergence, optimisation and disturbance rejection when the UAV is in the neighbourhood of the point. 

\subsection{Nonlinear dynamic feedback controller}

We illustrate in this section a nonlinear dynamic control law inspired by the
result of \cite{2020e-MicCenZacFra}. For the nonlinear control law of \cite{2020e-MicCenZacFra} to be applicable, matrices $F$ and $M$ reported in \eqref{eq:FandM} must allow defining a so-called zero moment direction $\boldsymbol{\bar u} \in \real^4$ ensuring $|F\boldsymbol{\bar u}| = 1$ and $M \boldsymbol{\bar u}=0$, and a right inverse $M^r$ of $M$ satisfying $M M^r = I$ and $FM^r=0$. In our case, it is immediate to see that the zero-moment direction
$\boldsymbol{\bar u} = \frac{\sqrt{2}}{2a_{\text{f}}}\smallmat{1&1&0&0}^\top$ satisfies the required conditions, whereas the fact that $\rank(F) = 2$ (so that $\ker F$ has dimension 2) makes it impossible to find a right inverse $M^r$ of $M$ completely contained in $\ker F$. Due to this fact, we determine $M^r$ by (conservatively) parametrizing the right pseudoinverses of $M$ as $M^r := KM^\top ( MKM^\top)^{-1}$ where parameter $K \in \real^{4\times 4}$ is symmetric and satisfies $MKM^\top \geq I$ (to ensure invertibility). Under this parametrization, the goal is to minimize the norm of $FM^r = F KM^\top ( MKM^\top)^{-1}$, which is well achieved by minimizing the norm of $F KM^\top$, due to the fact that the constraint on $MKM^\top \geq I$ ensures that the factor $( MKM^\top)^{-1}$ has norm smaller than 1.
Performing a Schur complement, this minimization is well obtained by solving the following semi-definite program:
\begin{align*}
&  \min_{K, \kappa} \kappa, \; \mbox{subject to:}
\;  M K M^\top\! \geq \!I, \; 
  \begin{bmatrix}
  \kappa I  &\! F K M^\top \\ 
   M K^\top F^\top &\! \kappa I
  \end{bmatrix}\! \geq\! 0,
\end{align*}
which minimizes $\kappa$ while ensuring $F K M^\top M K^\top F^\top \leq \kappa^2 I$. Solving this optimization, we obtain, for the specific matrices under consideration,
\begin{align*}
  K \!=\! \smallmat{  0   &   -737   &    171   &   -171\\
      -737  &  0   &  -171  &     171\\
       171  &    -171    &   1583.5    &   -43.73\\
      -171  &     171    &   -43.73    &   1583.5}\!, \,
  M^r \!=\! \smallmat{0     &          0   &   -3.19\\
                 0      &         0   &    3.19\\
               -4.51    &  -27.75    &   -1.48\\
                4.51    &  -27.75    &    1.48}
\end{align*}
leading to $\kappa = 39.7$. With this optimality-based selection, the nonlinear dynamic design of \cite{2020e-MicCenZacFra} can be effectively applied by obtaining responses that are almost indistinguishable from the fully decoupled case 
$FM^r=0$. Note that a similar approach, essentially neglecting the extra terms acting on the translational dynamics is also suggested in the survey paper \cite{hamel_minhduc}. 

Based on the above-described choice of $M^r$ and $\boldsymbol{\bar u}$, applying the feedback law in \cite[eqn (19)]{2020e-MicCenZacFra}, the input $\boldsymbol{u}$ becomes : 
\begin{align}
\label{eq:u_nonlin}
    \boldsymbol{u} = \boldsymbol{u_{\text{nl}}} := M^r \boldsymbol{\tau_{r}} + \boldsymbol{\bar u} \boldsymbol{f},
\end{align}
where $\boldsymbol{\tau_{r}}$ and $\boldsymbol{f}$ are provided by the dynamic feedback controller proposed in \cite{2020e-MicCenZacFra}.

The optimality-based selection of $M^r$ is prone to a few interesting interpretations when observing the product 
$M^r \boldsymbol{\tau_{r}} = M^r \smallmat{\tau_{r,x} & \tau_{r,y} & \tau_{r,z}}^\top$. First, 
% Three observations can be made. Indeed, we observe that 
to produce a moment $\tau_{r,z}$ about the $z$-axis we mainly use the thrust differential action; secondly, a moment $\tau_{r,y}$ about the $y$-axis is generated by an equal (additive) use of the two flaps, with great efficiency; finally a moment $\tau_{r,x}$ about the $x$-axis comes from a differential use of the flaps. 

As a final remark, as compared to the solution proposed in \cite{2020e-MicCenZacFra}, to partially take into account the saturation effects highlighted in Remark~\ref{rem:saturation}, the error feedback interconnection of the outer loop in \cite{2020e-MicCenZacFra} has been augmented with a simple error governor strategy never allowing the translational position error $\boldsymbol{\rm e}_p$ 
entering \cite[eqn. (22)]{2020e-MicCenZacFra} to exceed the maximum value of 3 meters. 
The remaining tuning gains required in the solution of \cite{2020e-MicCenZacFra} have been selected
following an intuitive PD tuning procedure
 as  $k_{pp} = 0.5$, $k_{pd} = 1.2$, $k_{ap} = 0.08$, $k_{ad} = 0.1$ and $k_{\Delta} = 1$.
 
Figure~\ref{fig_global_contol} shows the response of the system in terms of linear and angular positions (top two rows) and actuators efforts (bottom two rows) when the system starts from the initial condition  $\boldsymbol{x(0)} = [\boldsymbol{p(0)}~ \boldsymbol{v(0)}~ \boldsymbol{q(0)}~ \boldsymbol{\omega_b(0)}]^\top = [0~0~0 ~ 0~0~0 ~0.9140 ~0.1134~ -0.3728~ 0.1134~ 0~ 0~ 0]^\top $ with a target equilibrium position of $\boldsymbol{p_{\text{eq}}} = [4~5~6]^\top$ and $\boldsymbol{q_{\text{eq}}} = [\frac{\sqrt{2}}{2}~0~-\frac{\sqrt{2}}{2}~0]^\top$. 
% (this is associated to a large initial error of ??????? ).
A graceful response can be seen, which remains quite far from the actuator saturations (see Remark~\ref{rem:saturation}). Increasing the tuning gains can speed up the response but provides undesired attitude oscillations. Therefore it is interesting to combine this nonlinear controller (providing a large region of attraction) with a more aggressive controller, designed based on the linearized dynamics \eqref{eq:linearized} and to be used to improve the fail of the response.
\begin{figure}[ht!]
    \centering
    \includegraphics[trim=0cm 0.6cm 0cm 0.6cm,clip,width=1\columnwidth, height=7cm]{figures/global2.eps}
    \vspace*{-6mm}
    \caption{Simulation with the nonlinear dynamic feedback controller.}
    \label{fig_global_contol}
\end{figure}
\subsection{Linear control design}

Based on the performance-oriented observations of the previous section, given a target position corresponding to an equilibrium $\boldsymbol{p_{\text{eq}}}, \boldsymbol{q_{\text{eq}}}$ as characterized in Proposition~\ref{prop:eqs}, we design here a local linear feedback controller capable of inducing a more aggressive response. To this end, we focus on the linearized dynamics \eqref{eq:linearized} and recognize that we can design 
a state feedback controller 
\begin{align}
  \boldsymbol{u_{\text{lin}}} := \boldsymbol{u_{\text{eq}}} - K \boldsymbol{\tilde x},
\label{eq:u_lin}
\end{align}
where $\boldsymbol{\tilde x}$ has been introduced in \eqref{eq:xtilde} and $K \in \real^{4 \times 12}$ is a state feedback gain that can be designed, based on the matrices $A$ and $G$ appearing in \eqref{eq:linearized}, in such a way that the closed-loop linear feedback $A_{\text{cl}}:=A-GK$ be exponentially stable. 

For our design, we have used an LQR selection, associated with the simplest possible weight matrices selection $Q = I_{12}$ and $R = I_{4}$, which gives desirable closed-loop responses. The LQR design also provides a positive definite Lyapunov certificate matrix $S \in \real^{12 \times 12}$ (solution of the algebraic Riccati equation) ensuring that $A_{\text{cl}}^\top S + S A_{\text{cl}} <0$. In particular, it is well known from the linear approximation theorem that function $V(\boldsymbol{\tilde x}) = \boldsymbol{\tilde x}^\top S \boldsymbol{\tilde x}$ is also a Lyapunov function certifying local exponential stability of $\boldsymbol{x_{\rm eq}}$ for the nonlinear dynamics. More specifically, there exits a positive scalar $\bar v \in \real$ such that, along dynamics \eqref{eq:dyna_simp}, we have :
\begin{align}
\label{eq:Vdecrease}
  V(\boldsymbol{\tilde x}) \leq \bar v \quad \Rightarrow \quad \dot V(\boldsymbol{\tilde x}) := \langle 
\nabla V(\boldsymbol{\tilde x}), \boldsymbol{\dot{\tilde x}}\rangle <0,
\end{align}
for all $\boldsymbol{\tilde x} \neq 0$; in other words, the sublevel set $V(\boldsymbol{\tilde x}) \leq \bar v$ is contained in the basin of attraction of the equilibrium $\boldsymbol{x_{\text{eq}}}$.
 
Determining the largest possible scalar $\bar v$ ensuring \eqref{eq:Vdecrease} is a challenging problem and conservative lower bounds of this quantity can be determined by quantifying the effect of the nonlinearities on the dynamics. Since $\boldsymbol{\dot{\tilde x}}$ is a function of $\boldsymbol{x}$, then it is fairly easy to algebraically evaluate $\dot V(\boldsymbol{\tilde x})$ for a large amount of random extractions of the variable $\boldsymbol{\tilde x}$, so as to get a probabilistic estimate of the largest $\bar v$. Rigorous guarantees about these selections can be obtained by applying the results in \cite{tempo2013randomized}, which is out of the scope of this paper, but an evaluation of 10000 samples confirmed that the value $\bar v = 400$  is a good candidate selection satisfying \eqref{eq:Vdecrease}.

\begin{figure}[ht!]
    \centering
    \includegraphics[trim=0cm 0.6cm 0cm 1cm,clip,width=1\columnwidth, height= 7cm]{figures/converge2.eps}
    \vspace*{-5mm}
    \caption{Simulation of the full model (solid) and \eqref{eq:dyna_simp} (dashed) with $\boldsymbol{u} = \boldsymbol{u}_{\text{lin}}$ as in 
    \eqref{eq:u_lin} from an initial condition $\boldsymbol{\tilde x_0}$ within the basin of attraction.}
    \label{fig_linearize_conv}
\end{figure}
Figure \ref{fig_linearize_conv} shows a simulation starting at the origin with a zero orientation on the three axes (horizontal UAV)
and zero initial velocities,
with a target position $\boldsymbol{p_{\text{eq}}} = [4,~5,~6]$ with a hovering stabilization (vertical UAV) with $\beta = 0$. The dotted line represents the target position on each axis. Note that the initial linear and angular velocities are zero.  The last graph shows the desirable exponential decay of $V$
Figure \ref{fig_linearize_conv} shows both the simulation of the full model (solid) of \cite{sansou:stage} and of the simplified nonlinear model \eqref{eq:dyna_simp} (dashed) showing some significant differences in the initial response.
When providing a larger target position $\boldsymbol{p_{\text{eq}}} =[8,~9,~10]$ (with the same orientation), the initial condition is outside the basin of attraction and diverging solutions are experienced as shown in Figure~\ref{fig_linearize_div}.
\vspace*{-4mm}
\begin{figure}[ht!]
    \centering
    \includegraphics[trim=0cm 0cm 0cm 0cm,clip,width=\columnwidth, height=6cm]{figures/diverge2.eps}
    \vspace*{-7mm}
    \caption{Diverging simulation of the full model with $\boldsymbol{u} = \boldsymbol{u}_{\text{lin}}$ as in 
    \eqref{eq:u_lin} from an initial condition $\boldsymbol{\tilde x_0}$ outside the basin of attraction.}
    \label{fig_linearize_div}
\end{figure}



\subsection{Hysteresis-based local-global control design}
 
Inspired by the local/global strategies presented in \cite[Ex. 1.7]{65}, similar to the solution presented in \cite{DBLP:conf/IEEEcca/AndreettoFZ16}, we use a hybrid mechanism to switch between the high performance local feedback \eqref{eq:u_lin} (as long as the state is in the basin of attraction of the equilibrium) and the less aggressive nonlinear controller \eqref{eq:u_nonlin}, which provides a larger region of attraction (and can be called with an abuse of notation the ``global controller''). To this end, we augment the controller state with a logical state variable

$\ell \in \{0,1\}$, governing the choice of the control input 
between \eqref{eq:u_nonlin} and \eqref{eq:u_lin} as
\begin{align}
\label{eq:u_hybrid}
  \boldsymbol{u}=\boldsymbol{u}_{\text{hyb}} := \ell \boldsymbol{u}_{\text{nl}} + (1-\ell) \boldsymbol{u}_{\text{lin}},
\end{align}
We ensure, through the hybrid dynamics, that $\ell$ can only take values in $\{0,1\}$. 
Its dynamics is defined by: 
\begin{align*}
    \left\{
        \begin{array}{ll}
            \dot \ell = 0,& \chi \in \mathcal{C}\\
            \ell^{+} = 1-\ell,& \chi \in \mathcal{D}
        \end{array}
    \right.
\end{align*}
where $\chi = \left[\boldsymbol{p},~ \boldsymbol{v},~ \boldsymbol{q},~  \boldsymbol{\omega},~ l\right]$ is the complete closed loop state and $\mathcal{C}$ and $\mathcal{D}$ are, respectively, the flow and the jump sets, defined as
\begin{align*}
    & \mathcal{C} := \mathcal{C}_{0} \cup \mathcal{C}_{1}, ~ \mathcal{D} := \mathcal{D}_{0} \cup \mathcal{D}_{1},\\
   & \mathcal{C}_{0} :=\{\boldsymbol{\chi} \in \mathbb{R}^{14}:~ V(\boldsymbol{\tilde x}) \le \overline{v} \mbox{ and } \ell=0\}\\
   & \mathcal{C}_{1} :=\left\{\boldsymbol{\chi} \in \mathbb{R}^{14}:~ V(\boldsymbol{\tilde x}) \ge \underline{v} \mbox{ and } \ell=1 \right\}\\
   & \mathcal{D}_{0} :=\left\{\boldsymbol{\chi} \in \mathbb{R}^{14}:~ V(\boldsymbol{\tilde x}) \geq \overline{v}\mbox{ and } \ell=0 \right\}\\
   & \mathcal{D}_{1} :=\left\{\boldsymbol{\chi} \in \mathbb{R}^{14}:~ V(\boldsymbol{\tilde x}) \leq \underline{v}\mbox{ and } \ell=1 \right\}
\end{align*}
where $V(\boldsymbol{\tilde x}) := \boldsymbol{\tilde x}^\top S \boldsymbol{\tilde x}$  has been defined in the previous section, $\overline{v}=400$ has been determined in the previous section to satisfy \eqref{eq:Vdecrease},
 and $\underline{v}$ is any positive constant satisfying $\underline{v}<\overline{v}$ (a smaller choice of $\underline{v}$ increases the hysteresis margin but postpones the desirable high performance tail of the feedback response). In our case we choose $\underline{v}= 350$.
%
The following result is an immediate consequence of the results in \cite[Ex. 1.7]{65} and the properties of our linear and nonlinear designs.

\begin{proposition}
  Under the action of the hybrid feedback \eqref{eq:u_hybrid}, the closed loop exhibits the same basin of attraction as the one associated with the nonlinear controller \eqref{eq:u_nonlin}, while always using the high-performance linear feedback \eqref{eq:u_lin} in the tail of the response.
\end{proposition}

We performed several simulations of the closed loop using the Matlab toolbox \cite{sanfelice_2017}. The simulations are carried out with the complete model of the UAV \cite{sansou:stage}, including all the nonlinear aerodynamic effects. A sample simulation is reported in Figure~\ref{fig_sim}, where we initialize the UAV at the origin with zero roll and yaw orientation, and with a pitch angle of 45 degrees. The target orientation is in the vertical hovering configuration and the target position is assigned to $\boldsymbol{p_{\text{eq}}} = [50,~25,~12.5]$.

We observe that in the time phase $t \in \left[0,38\right]$, the UAV
exhibits a graceful but slow convergence to the desired target position using the global controller ($\ell=1$). At that time, the state enters set $\mathcal{D}_1$ and the more aggressive local controller is activated up to the convergence to the desired equilibrium.

To perform realistic simulations, the measurements are affected by 
sensors noise. The intrinsic robustness of the hybrid feedback, established in \cite[Chapter 7]{65} is confirmed by the graceful performance degradation as a function of the amplitude of the measurement noise.

\section{Schéma de commande hybride}

\section{Simulations}






