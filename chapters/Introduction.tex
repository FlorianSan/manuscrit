\chapter*{Introduction}
\addstarredchapter{Introduction} %Sinon cela n'apparait pas dans la table des matières
\markboth{Introduction}{Introduction} % headers


\section*{Contexte}
Ces dernières années, le domaine des drones s'est considérablement développé. En effet, de nombreux progrès ont été réalisés dans la conduite de vols autonomes, lesquels permettent de réaliser de nombreuses tâches longues, répétitives ou dangereuses, de manière plus sûre que des avions ou des systèmes télépilotés. Les drones ont fait leurs preuves dans de nombreuses applications civiles, alors qu'ils étaient auparavant conçus à des fins de surveillance et de destruction dans le secteur militaire. Tout leur intérêt réside dans leur capacité à se maintenir stabilisé sans intervention humaine. Ainsi, les opérateurs peuvent se concentrer sur la mission, sans devoir consacrer une grande attention au pilotage du drone. 

La possibilité d'utiliser des systèmes de vols autonomes dans le secteur civil a été rendu possible par l'accessibilité croissante, proposée par l'industrie, de solutions à faible coût pour les applications d'imagerie aérienne. Ainsi, ce sont dans des domaines aussi variés que l'agriculture de précision,  l'inspection des infrastructures civiles ou encore les opérations de sécurité que les drones autonomes sont aujourd'hui mobilisés, devenant alors un riche sujet de recherche.

La miniaturisation des équipements électroniques et mécaniques est à l'origine de l'essor d'une classe de drones de plus en plus petits. Souvent qualifiés de \textit{Micro Air Vehicle} (MAV) ou de \textit{Unmanned Aerial Vehicle} (UAV), leur petite taille leur permet d'intervenir dans des espaces confinés ou contraints. Ils n'ont, cependant, qu'une charge utile restreinte, souvent limitée à l'emport d'une caméra ou d'un colis de faible masse. 
\nomenclature[]{\(MAV\)}{Micro drone (\textit{Micro Air Vehicle})}
\nomenclature[]{\(UAV\)}{Drone autonomes (\textit{Unmanned Aerial Vehicle})}
Leur faible autonomie restreignant leur usage, la recherche s'est alors concentrée sur une solution permettant d'optimiser leur utilisation. En cela, les drones à décollage et atterrissage verticaux (\textit{Vertical take-off and landing}; VTOL) répondent aux exigences.
\nomenclature[]{\(VTOL\)}{Drones à décollage et atterrissage verticaux (\textit{Vertical Take-Off and landing})}

Dans l'ensemble de VTOL, plusieurs architectures existent et seront détaillé dans la section \ref{sec:archConvertible}. Toutefois, nos travaux se sont concentré sur les classes des \textit{tailsitters} et des \textit{freewings}.


De nombreux travaux ont été menés sur les \textit{tailsitters}, avec l'objectif de couvrir l'intégralité du domaine de vol. Ce dernier est constitué des phases de vol suivantes :
\begin{enumerate}
    \item Décollage vertical
    \item Transition entre le vol stationnaire et le vol d'avancement
    \item Vol d'avancement
    \item Transition entre le vol d'avancement et le vol stationnaire
    \item Atterrissage vertical
\end{enumerate}
Bien que l'on puisse observer une symétrie dans la phase \raisebox{.5pt}{\textcircled{\raisebox{-.9pt} {1}}} et \raisebox{.5pt}{\textcircled{\raisebox{-.9pt} {5}}}, qui correspondent au décollage et à l'atterrissage vertical, une différence fondamentale est observée. Lors du décollage, la vitesse du drone engendrera un flux d'air sur l'aile orienté dans le même sens que le flux d'air généré par les hélices. Cependant, lors de l'atterrissage, le flux d'air va se trouver inversé, le drone devant descendre, ce qui engendre une vitesse opposée à la direction du flux d'air des hélices. Cette inversion génère une instabilité qui doit être compensée par le contrôleur.

Le vecteur $\overrightarrow{W}$ représente la perturbation de vent qui peut affecter le vol sur l'intégralité des cinq phases de vol. Toutefois, on observe que dans les phases décollage \raisebox{.5pt}{\textcircled{\raisebox{-.9pt} {1}}}, de transition \raisebox{.5pt}{\textcircled{\raisebox{-.9pt} {2}}} et \raisebox{.5pt}{\textcircled{\raisebox{-.9pt} {4}}} et d'atterrissage \raisebox{.5pt}{\textcircled{\raisebox{-.9pt} {5}}}, le drone offre une grande surface verticale sujette au vent. Ainsi, il est nécessaire de traiter l'impact du vent sur cette architecture.

\begin{figure}[ht!]
    \centering
        \includegraphics[width=0.8\columnwidth]{figures/darko_transition.png}
        \caption{Phases de vol d'un drone \textit{tailsitters}, DarkO.}
        \label{fig:darko_flight}
\end{figure}


Nous pouvons citer en exemple le \textit{tailsitter} à double rotors appelé « T-Wing » \cite{Stone2002PreliminaryDO, TWing2008}, un autre tail-sitter appelé « MavIon » \cite{oatao14575}, ou le « JLion » et le « KH-Lion » \cite{8003167}. Ces drones partagent une architecture similaire basée sur une aile supportant deux moteurs sur le bord d'attaque et soufflant deux élevons situés sur le bord de fuite. Cette architecture offre une plus grande robustesse que les \textit{tiltrotors}, composés de pièces mobiles, ce qui les rend plus fragiles et d'un actionneur puissant pour faire tourner l'ensemble moteur-hélice.
La complexité inhérente à ces architectures nécessite un travail de modélisation en raison des nombreuses non-linéarités et couplages impliqués, en particulier en termes de modélisation des effets aérodynamiques. Dans ce contexte, l'interférence aérodynamique entre l'aile fixe et les rotors a été modélisée dans \cite{droandi_zanotti_gibertini_grassi_campanardi_2015, Simmons2022, aerospace5030079}, et les forces et moments d'hélice générés à des angles d'attaque élevés sont abordés dans \cite{Fernandez2023}. Cependant, ces modèles sont complexes et ne sont que partiellement utilisables pour la conception des commandes. 

Un autre point important est la représentation de l'attitude du drone. Aussi, il est possible de représenter son orientation par des angles d'Euler \cite{4177650, 5415267, 8003165}, ce qui permet une compréhension intuitive. Toutefois, une singularité apparaît dans certaines phases de vol. Compte tenu de la grande manœuvrabilité, il est préférable de représenter l'attitude par un quaternion unitaire, ce qui élimine toute singularité \cite{8027691}. De nombreuses publications modélisent les effets aérodynamiques en fonction de l'angle d'attaque et du dérapage générés par les hélices \cite{Escareno07, 8453301}. 
Il est possible de choisir un autre modèle pour les interactions aérodynamiques entre les moteurs, les ailes et les élevons, comme présenté dans \cite{lustosaHal-03035938}. La technique de modélisation présentée dans \cite{lustosaHal-03035938} permet de disposer d'un modèle global couvrant l'ensemble de l'enveloppe de vol, grâce à ce que l'on appelle l'approche $\Phi$-théorie. Bien que cette dernière ne permette pas de prédire la chute brutale de la force de portance avec un angle d'attaque (AoA) croissant (qui est causée par un flux d'air turbulent) \cite{tal2022global}, elle permet de représenter le drone avec suffisamment de précision pour capturer le comportement lors de manœuvres agressives. 


Actuellement, nous pouvons mentionner deux types d'architecture de commande ayant fonctionné sur ce tailsitter. La première est basée sur une inversion incrémentale non-linéaire de la dynamique du drone (\textit{Incremental Non-linear Dynamic Inversion}, INDI) \nomenclature[]{\(INDI\)}{Inversion incrémentale non-linéaire  (\textit{Incremental Non-linear Dynamic Inversion})} et la seconde est basée sur le technique sans modèle (\textit{Model free control}, MFC). \nomenclature[]{\(MFC\)}{Commande sans modèle (\textit{Model free control})}

Les deux architectures sur lesquelles se sont concentré nos recherches sont celle de DarkO, un \textit{tailsitters} et celle de Colibri un \textit{freewings} basé sur une aile inspiré de DarkO en rotation libre autour d'un fuselage qui sera maintenu horizontal.



\todo{rejet de perturbation et metrique de maintient de position }


\section*{Question de recherche}
Contrôleur unifié d'une architecture de drone fortement non-linéaire et couplé sur l'intégralité du domaine de vol en environnement perturbé.

\section*{Plan et contribution}
Notre exposé commencera par une description générale des architectures de drone convertible (\ref{chap:generalites}) avec une description des avantages et des inconvénients ainsi que leur mode de fonctionnement. Une description générale de la modélisation, de l'actionnement et des lois de commande proposé sur les \textit{tailsitters} et les \textit{freewings} ouvrira nos propos sur ces architectures.

Le chapitre \ref{chap:model} détaillera le modèle non-linéaire d'un drone \textit{tailsitters}, DarkO, à partir des travaux de \cite{lustosaHal-03035938} et de \cite{olszaneckibarthHal-02542982}. Nous proposons un modèle simplifier pour les basses vitesses ainsi que le détail des équilibres stationnaire en présence ou non de vent. De ces équilibres, nous présentons la dynamique linéarisée du drone paramétré par deux scalaires, le vent horizontal et vertical. Ce modèle étant le point de départ des chacun de nos travaux, il se retrouve expliquer dans \cite[Chapitre 2]{sansouStage} et \cite[Section II]{sansouECC} dans la condition de vent nulle, dans \cite[Section 2]{SANSOUACA} avec des conditions de vent non nulle et dans \cite[Section II]{sansouTCST} sous sa forme la plus complète.

Le chapitre \ref{chap:3DOF} fera l'objet d'une proposition de loi de commande hybride permettant d'augmenter le domaine de stabilité d'une loi linéaire avec une loi de commande non-linéaire basée sur une direction de zéro moment. Ces travaux ont été publiés dans  \cite{sansouStage} et \cite{sansouECC}.

Le chapitre \ref{chap:3DOF} permet de décrire une maquette expérimentale utilisée pour tester une loi de commande basée sur une architecture proportionnelle dérivative sur un retour de sortie proposé pour stabiliser une position stationnaire en présence de vent. Cette maquette restreint les degrés de liberté classique d'un drone pour se concentrer sur la réjection de perturbation de vent grâce à un changement d'incidence de la maquette. La description de la loi de commande, son optimisation et les résultats sont disponibles dans \cite{SANSOUACA}.

Le chapitre \ref{chap:6DOF} \cite{sansouTCST}

Le chapitre \ref{chap:colibri} \cite{sansouICUAS}

Publications

\section*{Objectif fixé pour la thèse}

Étudier le comportement d'un drone \textit{tailsitters} en environnement perturbé, saturation des actionneurs, dynamique linéarisé, impact des non-linéarité et cycle limite.

Proposer des architectures de commande basée modèle pour un drone \textit{tailsitters} permettant d'assurer une robustesse aux perturbations de vent, ouvrant la possibilité de certification de ce type d'architecture.

Utiliser des capteurs pour mesurer les perturbations en avance de phase pour les rejeter (sonde 5 trous, micro, Pitot, etc.) ce qui implique de développer une nouvelle architecture 
