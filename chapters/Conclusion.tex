\chapter*{Conclusion}
\addstarredchapter{Conclusion} 

\todo{Finir la conclusion}
{ \color{red}

Cette thèse est consacrée à la modélisation, à l'étude et au contrôle de drone convertible sujet à des forces aérodynamiques, des couplages entre les actionneurs et des dynamiques non linéaire. Elle propose, au travers de l'utilisation d'un modèle unifié représentant les forces aérodynamiques sur l'ensemble du domaine de vol, d'analyser le comportement d'un \textit{tailsitter} et de proposer des méthodes de commande. Notre travail à déboucher sur la proposition d'une nouvelle architecture de type \textit{freewing}.

La première partie du manuscrit propose un rapide aperçu des architectures de drone convertible avant de se focaliser sur les \textit{tailsitter} et les \textit{freewing}. Nous avons développé les principales caractéristiques d'actionnement, le comportement des drones ainsi que les méthodes de modélisation. Ce chapitre introductif a permis de développer l'architecture conventionnelle de commande ainsi qu'un tour d'horizon des méthodes de commande et d'optimisation des contrôleurs. Notre travail ayant une force composante expérimentale, une vision de l'architecture nécessaire à ces expérimentations a été évoqué avant d'être développé plus en profondeur dans l'annexe \ref{chap:annexe1}.  

Nous avons utilisé un modèle de la littérature, sans singularité sur l'ensemble du domaine de vol, ne faisant pas appel aux angles aérodynamiques $\alpha$ et $\beta$, appelé $\phi$-théorie. Ce modèle mathématique n'a d'utilité pratique que s'il est cohérent avec la réalité, ce qui a pu être démontré dans la littérature et qui est confirmé par nos travaux. De ce modèle non-linéaire, nous avons pu extraire des caractéristiques intéressantes pour les drones à décollage et atterrissage vertical. 
Nous avons caractérisé l'ensemble de point d'équilibre avec ou sans vent pour un \textit{tailsitter}. De ces équilibres, nous avons extrait la dynamique linéarisée, point de départ de la conception de toute loi de commande linéaire. Notre compréhension du comportement du drone a été augmenter par ces résultats qui nous informe sur la commandabilité du drone, sur les marges vis-à-vis des saturations et sur la capture du comportement du drone par une linéarisation autour des points d'équilibre. Nous avons donc pu valider la précision des linéarisations face aux nombreuses non-linéarités du modèle. Pour cela, nous avons effectué des simulations en boucle fermée, au vu du comportement instable du drone, du modèle linéaire et non linéaire.

Un travail préliminaire a permis de proposer une architecture de commande hybride avec un mécanisme d'hystérésis, basée sur une variable discrète sélectionnant la loi de commande la plus appropriée en fonction de la phase de vol. Les deux lois proposées dans ce cas sont une loi non linéaire basée sur une direction de zéro-moment et une loi linéaire LQR. Cette loi LQR est optimisé grâce au modèle obtenu plus précédemment.

De ce travail et à l'aide de la linéarisation, nous avons observé un comportement intéressant pour le rejet de perturbation de vent sur un \textit{tailsitter}. Ce comportement repose sur le changement de l'angle de tangage du drone pour compenser l'augmentation de la vitesse air qui engendre un déplacement du drone. Nous avons donc expérimenté à l'aide d'une maquette à trois degrés de liberté une loi de commande proportionnelle intégrale. Cette maquette, utilisant une architecture physique associé à un modèle de dynamique transactionnelle simuler, a permis de valider l'architecture de commande ainsi que son optimisation basée sur des contraintes $H_{\infty}$.
Bien que les résultats obtenus soient prometteur, nous avons étudié une méthode différente d'obtention des gains du contrôleur PI. Cette méthode, plus conservative, est basée sur une résolution successive de LMI. Les résultats ont pu être évalué sur l'architecture complète du drone, par un vol expérimental en volière.

Cette expérimentation a permis d'identifier des problèmes de sensibilité de la boucle fermée aux dynamiques non modélisée et au bruit. Nous avons donc proposé une extension du contrôleur PI pour augmenter sa robustesse. Une expérimentation face à un vent croissant par palier à valider notre travail.

Nos travaux nous ont amené vouloir installer un capteur de vent sur le drone pour pouvoir utiliser la mesure pour la transition. Toutefois, le corps du \textit{tailsitter} étant en rotation lors de la transition, nous ne pouvions pas fixer le capteur de manière satisfaisante. Nous avons donc étudié et développé une architecture \textit{freewing} procurant un fuselage maintenu horizontal permettant d'installer n'importe quel capteur ou charge utile. L'aile étant en rotation libre autour du fuselage, nous conservons de nombreuse propriété des \textit{tailsitter}. Dans cette démarche, nous avons modélisé le drone avec une dynamique multicorps, identifier les paramètres, fabriquer la maquette et réalisé des vols expérimentaux à l'aide de l'INDI.

\section{Limite de l'étude}
Les travaux préliminaires, menés au chapitre \ref{chap:hybrid}, ne sont que des résultats de simulation. Il serait souhaitable de réalisé des expérimentations du contrôleur non linéaire basée sur une direction de zéro-moment ainsi que de son utilisation dans l'architecture hybride avec une transition entre le contrôleur basée sur une direction de zéro-moment et le contrôleur PI étendu développé au chapitre \ref{chap:6DOF}.

Bien que nous souhaitions utiliser la mesure du vent, notre travail n'a pu aboutir par la richesse des questions que nous avons souhaitées développe en amont et par le temps nécessaire au développement de l'architecture nécessaire.
De plus, de nombreuses architectures auraient pu répondre au besoin. Nous avons choisi de nous concentrer sur une architecture inspirée du \textit{tailsitter} DarkO car nous avions de l'expérience dans sa modélisation et sa dynamique.

Tous nos résultats ont été expérimentés dans une atmosphère contrôlée avec un générateur de perturbation, il serait maintenant intéressant d'évaluer la précision de nos contrôleurs en extérieur. Ce travail possède une double complexité car le drone évoluerait dans un environnement plus turbulent, mais aussi avec une estimation d'état moins précise. Effectivement, en intérieur, nous avons accès à un système de positionnement millimétrique alors qu'en extérieur les GPS ne peuvent nous fournir une information de position aussi précise.


\section{Travaux futurs}
{
    \color{green}
La modélisation de Udwadia-Kalaba permet d'obtenir un modèle d'un drone multicorps. Il serait intéressant d'utiliser ces travaux pour concevoir un contrôleur 
mesure du vent 
guidage 
}
}