\chapter*{Conclusion}
\addstarredchapter{Conclusion} 
\todo{Faire la conclusion}
{ \color{red}

Nous avons utilisé un modèle de la littérature, sans singularité sur l'ensemble du domaine de vol, ne faisant pas appel aux angles aérodynamiques $\alpha$ et $\beta$. Ce modèle mathématique n'a d'utilité pratique que s'il est cohérent avec la réalité, ce qui a pu être démontré dans la littérature et qui est confirmé par nos travaux.

De ce modèle, nous avons pu extraire des caractéristiques intéressantes pour les drones à décollage et atterrissage vertical. 
Nous avons caractérisé l'ensemble de point d'équilibre avec ou sans vent pour un \textit{tailsitter}. De ces équilibres, nous avons extrait la dynamique linéarisée.



\section{Limite de l'étude}

\section{Travaux futurs}

Mesure de la perturbation avec une sonde cinq trous
Limite de sensibilité

}