\chapter{Généralités sur les drones convertibles}
\minitoc
\label{chap:generalites}


\section{Micro drones convertibles}

    \subsection{Efficacité énergétique et Domaine de vol} 
    Tout l'intérêt d'un drone convertible réside dans sa capacité à décoller et atterrir verticalement, tout en conservant une bonne efficacité énergétique en vol d'avancement, grâce à une aile. Cette aile a l'avantage de générer de la portance, laquelle s'oppose au poids du drone et permet d'assurer sa sustentation. La contrepartie de la génération de la portance est la traînée qui s'oppose à l'avancement et doit être contrée par une force de traction engendrée par les hélices. 
    
     Nous pouvons définir l'efficacité énergétique comme le ratio entre le temps de vol et l'énergie électrique nécessaire pour effectuer ce vol. Afin de souligner la prééminence de l'efficacité énergétique de ce modèle convertible, il convient de la comparer avec celle d'un drone quadrirotor qui assure sa sustentation uniquement grâce à des hélices (à l'instar d'un hélicoptère), et à celle d'un drone à voilure fixe.
    \begin{table}[ht]
        \centering
        \begin{tabular}{|c|c|c|c|c|}
            \hline
            Architecture & Vitesse $\SI{}{\meter\per\second}$  & Stationnaire & Temps de vol & Consommation\\
            \hline \hline
            Convertible & [0 - 30] & Possible & Quelques heures & Faible\\
            \hline
            Quadrirotor & [0 - 16] & Possible& Quelques minutes & Forte\\
            \hline
            Voilure fixe & [8 - 30] & Impossible & Plusieurs heures & Variable \\
            \hline
        \end{tabular}
        \caption{Comparaison des architectures de drone.}
    \end{table}

    Nous observons qu'un drone convertible possède un domaine de vol bien plus important qu'un drone à voilure fixe, lequel ne sera pas en mesure de voler à très basse vitesse, et qu'un quadrirotor, dont l'autonomie va être limitée par sa consommation.
    En alliant autonomie et vol stationnaire, le modèle convertible répond aux exigences des missions civiles et militaires.

    \subsection{Types d'architectures des drones convertibles}
    \label{sec:archConvertible}
    La conception structurelle et aérodynamique d'un drone est le facteur principal permettant des transitions stables et fluides. De plus, il est nécessaire d'optimiser l'architecture pour une mission, de manière à être le plus efficace. Au vu de la diversité des tâches à réaliser, un grand nombre de modèles ont été proposés, lesquels sont généralement catégorisables en trois classes : \textit{tiltrotor}, \textit{tailsitter}, \textit{tiltwing}. 
    En ce qu'il s'agit des prémices des convertibles, il paraît opportun d'ajouter aux trois grandes catégories précédemment citées dans les états de l'art \cite{saeed_survey_2018,ducard_review_2021, review_2022}, la catégorie \textit{quadplane}.

    
        \subsubsection*{\textit{Quadplane}}
        Les \textit{quadplanes} résultent de la fusion d'un avion et d'un quadrirotor (comme visible sur la figure \ref{fig:quadplane}), ce qui permet un découplage de l'actionnement en fonction de la phase de vol. Le premier système de propulsion est composé de quatre hélices générant une force verticale permettant le contrôle lors des phases de décollage, d'atterrissage et stationnaire. Le second système de propulsion est composé d'une hélice propulsive supplémentaire afin de maintenir le vol d'avancement.

        \begin{figure}[ht!]
            \centering
                \includegraphics[width=0.6\columnwidth]{figures/Avy-Drone-quadplane.jpg}
                \caption{Structure \textit{quadplanes} proposée par \cite{Avy_2023}.}
                \label{fig:quadplane}
        \end{figure}
        

        L'avantage de ce type d'architecture est sa grande robustesse. Effectivement, aucune pièce en mouvement n'est nécessaire pour réaliser la transition, ce que réduit le risque de défaillance mécanique. Toutefois, l'inconvénient réside dans son manque d'efficacité. Lors d'un vol d'avancement, la portance sera générée par l'aile. Ainsi, il est possible de désactiver les rotors qui génèrent des perturbations aérodynamiques et des trainées parasites. Les axes des moteurs se retrouvent orthogonaux au flux d'air généré par le déplacement du drone, ce qui correspond au cas le plus défavorable en termes de trainée. De plus, la surcharge engendrée par l'emport de moteurs supplémentaires se traduit par une diminution de la charge utile transportable. 

        Pour ce qui est du contrôle, un atout indéniable est la séparation des actionnements en fonction des phases de vol. Ainsi, l'architecture de commande sera basée sur un mécanisme de commutation permettant de choisir la loi de commande appropriée sur un critère de vitesse air \cite{LiSongZhang2021, MathurAtkins2021, okulski2022small}. Ce critère est pertinent en ce qu'il est lié à la capacité de l'aile à générer de la portance induite par le flux d'air. Ainsi, à basse vitesse, le drone se stabilise avec l'actionnement quadrirotor et la loi de commande associée. Dans les vitesses plus importantes, la commutation de la loi permet de contrôler le drone en mode avion. Toutefois, le passage d'une loi à l'autre reste le point clé de la commande et demeure complexe et critique.

        \subsubsection*{\textit{Tiltrotor}}

        Les \textit{tiltrotors} nécessitent l'utilisation d'un actionneur dédié à la modification de l'orientation des moteurs. Les rotors sont montés sur des arbres basculants actionnés et la transition du vol stationnaire au vol d'avancement (ou inversement) s'effectue progressivement en fonction de l'inclinaison du rotor. Les deux configurations sont représentées sur la figure \ref{fig:tiltrotor}. Ainsi, l'angle entre le souffle des hélices et l'aile peut être ajusté à chaque instant \cite{9836063, du2024numerical, nie2024hierarchical, schlatter2024longitudinal}. Cet angle joue un rôle important dans le contrôle des forces et des moments aérodynamiques : sa maîtrise permet, non seulement, de mieux gérer les performances aérodynamiques du vol lors des transitions, mais aussi la stabilité du système sur l'ensemble du domaine de vol. 
        Malgré le fait que les \textit{tiltrotor} embarquent un actionneur uniquement dédié à la transition, ce qui augmente la masse du drone, cette architecture est intéressante car elle permet d'utiliser les mêmes actionneurs pour assurer la sustentation en stationnaire et pour générer la traction en mode avion.
        \begin{figure}[ht!]
            \centering
            \resizebox{.9\textwidth}{!}{%
            \includegraphics[height=3cm]{figures/tiltrotorhover.png}
            \quad
            \includegraphics[height=3cm]{figures/tiltrotorforward.png}
            }
            \caption{Structure \textit{Tiltrotor}  proposée par \cite{7040348}, dans deux configurations, vol stationnaire et d'avancement.}
            \label{fig:tiltrotor}
        \end{figure}


        

        \subsubsection*{\textit{Tailsitter}}
        Contrairement aux \textit{tiltrotors} qui se posent sur le fuselage de l'avion (figure \ref{fig:tiltrotor}), les \textit{tailsitter} se posent à la verticale (voir figure centrale \ref{fig:tailsitter}). Durant la transition du mode stationnaire au vol d'avancement, la structure entière bascule vers l'avant, modifiant ainsi l'angle d'incidence de la voilure \cite{RobinRaffaello2017, VerlingWeibelSiegwart2016,smeurINDITail, ChiappinelliNahon2018, tal2022global}. Selon la configuration du \textit{tailsitter}, la transition peut être réalisée soit par la génération du moment aérodynamique créé par les élevons, soit par le couple créé par le système de propulsion. Pendant le vol d'avancement, en position horizontale, le \textit{tailsitter} vole comme un avion conventionnel sans dérive. En utilisant des techniques aérodynamiques classiques, les concepteurs peuvent optimiser le profil de l'aile du \textit{tailsitter} pour le rendre plus endurant, afin de réduire la consommation d'énergie. Grâce à ce processus d'optimisation aérodynamique, le \textit{tailsitter} peut effectuer des missions de vol de plus d'une heure.

        \begin{figure}[ht!]
            \centering
            \resizebox{.9\textwidth}{!}{%
            \includegraphics[height=3cm]{figures/cyclone.png}
            \quad
            \includegraphics[height=3cm]{figures/falcon.png}
            \quad
            \includegraphics[height=3cm]{figures/HoverEye.png}
            }
            \caption{Structure \textit{tiltrotor}  proposée par \cite{smeurINDITail,fernandez:hal-04612206,pflimlin:tel-00132352}.}
            \label{fig:tailsitter}
        \end{figure}
        
 
        Ce modèle semble être la configuration la plus intéressante énergétiquement, car il utilise les mêmes actionneurs dans tout le domaine de vol. Ainsi, aucune masse superflue n'est embarquée.

    
        \subsubsection*{\textit{Tiltwing}} 
        La particularité des \textit{tiltwings} réside dans le fait que les rotors sont inclinés en même temps que les ailes (voir figure \ref{fig:tiltwing}). Un actionneur supplémentaire et puissant est donc nécessaire pour surmonter le couple de l'aile, afin de la positionner dans l'orientation souhaitée \cite{holsten2011design, rohr2019attitude, ccetinsoy2012design}. La commande de cet actionneur doit être prise en compte lors de la conception des lois de commande {\color{red} car il permet de fixer l'orientation de l'aile et par conséquent l'incidence vis-à-vis du flux d'air incident. Toutefois, la dynamique de cet actionneur et l'inertie de l'aile limite les déplacements rapide lesquels permettrait de rejeter des turbulences.} Pendant le décollage, l'atterrissage et les vols stationnaires, les ailes doivent être positionnées vers le haut afin de produire une force de poussée opposée au vecteur gravité. Dans ces phases de vol, lorsque les ailes sont orientées vers le haut, l'aéronef est plus vulnérable au vent et les lois de commande doivent rejeter ces perturbations. Dans la littérature, il existe plusieurs configurations d'ailes basculantes et différentes approches de contrôle conçues pour stabiliser leur dynamique de vol.
        \begin{figure}[ht!]
            \centering
            \resizebox{.9\textwidth}{!}{%
            \includegraphics[height=3cm]{figures/tiltwing_aero2.png}
            \quad
            \includegraphics[height=3cm]{figures/avigle_tiltwing.png}
            }
            \caption{Structure \textit{tiltwings}  proposée par \cite{Aero2_2024, Ostermann2012ControlCO}.}
            \label{fig:tiltwing}
        \end{figure}

        Par ailleurs, les \textit{freewing} sont une gamme en cours de développement à l'intérieur de l'architecture des \textit{tiltwing}. Ils sont actionnés comme des \textit{tiltwing}, excepté au niveau de l'axe de rotation entre l'aile et le fuselage. Cette rotation est laissée libre, ce degré de liberté permettant à l'aile de changer librement son incidence. Il en résulte un gain de masse car il est possible de supprimer l'actionneur puissant et lourd nécessaire à la rotation de l'aile. De plus, l'aile étant libre de s'orienter, les turbulences ont un impact plus faible sur la structure, ce qui rend les vols plus stables \cite{freewing2012, Johnson2021, Johnson2023}. 


\section{Propriétés des \textit{tailsitters} et des \textit{freewings}}
    D'un point de vue mécanique, les \textit{tailsitters} et les \textit{freewings} sont caractérisés comme des systèmes sous-actionnés avec une dynamique fortement couplée. Ces caractéristiques mécaniques rendent le processus de modélisation et d'identification difficile. Cela peut s'expliquer par le fait que, pour ce type de système, une entrée de commande donnée agit simultanément sur différentes dynamiques. Ainsi, l'identification de l'influence d'une entrée de commande donnée sur une dynamique particulière reste un processus important qui nécessite plus d'attention.
    \subsection{Actionnement}
    Dans ces deux architectures, il est courant de trouver des actionneurs basés sur des effets aérodynamiques. Ces actionneurs ont l'avantage d'être peu consommateurs en énergie. Ils sont mus par des servomoteurs qui consomment peu d'électricité proportionnellement au couple qu'ils génèrent. Dans le cas des ailes volantes, les surfaces aérodynamiques sont souvent placées sur la partie arrière des ailes et peuvent être utilisées symétriquement à l'instar de volets, ou anti-symétriquement comme des ailerons. 
    Dans les phases de décollage, de vol stationnaire ou d'atterrissage, la plateforme est maintenue en vol par les hélices : il est donc nécessaire de dimensionner les groupes moteurs-hélices pour qu'ils puissent générer assez de force. En fonction des configurations, les moments peuvent être obtenus par des différentiels sur l'utilisation des moteurs ou bien par des surfaces aérodynamiques. Dans le cas de surfaces soufflées par le flux d'air des hélices, il existe un couplage des actionneurs qui complexifie la modélisation et le contrôle de ces architectures.


    \begin{definition}[Sous-actionnement]
        \label{def:sousacctionnement}
        Considérons la dynamique d'un système décrite par l'équation différentielle du second ordre :
        \begin{align*}
            \ddot{\boldsymbol{q}} = f_1(\boldsymbol{q}, \dot{\boldsymbol{q}}, t) + f_2(\boldsymbol{q}, \dot{\boldsymbol{q}}, t) \boldsymbol{u}
        \end{align*}
        où $\boldsymbol{q} \in \real^{n}$ est le vecteur d'état de dimension $n$, $\boldsymbol{u} \in \real^{m}$ est le vecteur de commande de dimension $m$ et $t$ le temps. 

        Lorsque le rang de $f_2$ est inférieur à la dimension du vecteur $\boldsymbol{q}$, on dit que le système est sous-actionné :
        \begin{align*}
            rank[f_2(\boldsymbol{q}, \dot{\boldsymbol{q}}, t)] < dim[\boldsymbol{q}]
        \end{align*}
    \end{definition}

    Nous observons au vu de la définition \ref{def:sousacctionnement} que beaucoup d'architectures présentées dans les catégories précédentes (\textit{tiltrotor}, \textit{tailsitter}, \textit{tiltwing}, \textit{quadplane}) sont sous-actionnées. La principale conséquence du sous-actionnement est l'impossibilité de générer un déplacement, sans modifier l'orientation du drone ou, autrement dit, tout déplacement sera engendré par un changement d'orientation du drone.
    

    \subsection{Aérodynamique}

    Lors d'un vol d'avancement, les \textit{tailsitters} et les \textit{freewings} assurent le maintien du vol en palier, en générant une force de portance grâce à leur surface alaire. Cette portance, qui s'oppose au poids, permet de maintenir une trajectoire. La force de trainée engendrée par l'aile et le fuselage est compensée par la composante horizontale de la poussée. 
    
    En vol stationnaire, le poids est compensé par la traction de l'hélice. La relation d'équilibre est plus complexe lorsque nous évaluons l'ensemble des points d'équilibre lors de la transition d'un mode à l'autre. La transition équilibrée est assurée par la composition correcte des forces aérodynamiques et de propulsion.

    Les interactions aérodynamiques entre l'hélice et la surface de l'aile sont connues pour être complexes et difficiles à modéliser. La vitesse induite par le souffle de l'hélice entraîne une variation locale de l'angle d'attaque et des variations de pression dynamique au niveau des sections d'aile immergées, d'où une distribution différente de la portance (voir figure~\ref{fig:darkoAirPress}). L'analyse de l'interaction entre l'hélice, l'aile et les élevons doit permettre d'obtenir de bonnes caractéristiques de vol et de tirer profit des combinaisons.

    \begin{figure}[ht!]
        \centering
            \includegraphics[width=0.6\columnwidth]{figures/Darko-air-pressure.png}
            \caption{Représentation de la pression dynamique sur la voilure de DarkO lors d'un vol d'avancement.}
            \label{fig:darkoAirPress}
    \end{figure}

     L'identification de ces effets aérodynamiques nécessite, pour chaque point de vol, les informations de l'air environnant, les valeurs des entrées de commande et les mesures des forces et moments aérodynamiques agissant sur le système. Le moyen le plus précis et le plus fiable d'identifier les phénomènes aérodynamiques est probablement de mener des campagnes en soufflerie, lesquelles sont particulièrement chronophages et coûteuses.


\section{Modélisations}

\subsection{Dynamique du système}
Dans le cas des drones convertibles basculant, le modèle aérodynamique doit être correct dans les phases de transition. Cela nécessite l'extension de l'aérodynamique classique (faible incidence) à un angle d'attaque élevé et à un fonctionnement à faible vitesse. En outre, l'effet du souffle de l'hélice sur les profils aérodynamiques du véhicule doit être compris et pris en compte dans le modèle du système. Par exemple, les travaux de \cite{9444145} sur un \textit{tiltwing} identifient les zones d'interaction entre l'hélice et l'aile. Ils séparent clairement la génération de force et de couple, lesquelles sont obtenues par la partie soufflée et non-soufflée de l'aile.

Des fonctions non linéaires continues décrivant les coefficients de portance et de traînée aérodynamiques sur toute la plage de l'angle d'attaque pour les  \textit{tiltrotors} ont été dérivées dans \cite{6981467} et pour les \textit{tiltwings} dans \cite{Lustosa2017LaP, lustosaHal-03035938}.

Il existe de nombreuses conceptions possibles pour un drone convertible. Bien que tous les modèles aient une structure commune, il existe des différences majeures dans la formulation réelle des termes de force et de couple. Celles-ci dépendent de la disposition des moteurs ou des hélices, de l'existence ou non de surfaces de contrôle aérodynamiques et de la forme du véhicule.

Un modèle précis est nécessaire pour les conceptions classiques de contrôle basées sur un modèle et en particulier pour les approches d'inversion dynamique ou de contrôle prédictif de modèle. Cependant, une modélisation précise va nécessiter une campagne d'identification approfondie en soufflerie ou un vol en environnement contraint. Il se peut également que la complexité du modèle ne permette pas une utilisation directe dans le contrôle, à cause d'une limitation matérielle des calculateurs embarqués.

\subsection{Modèle de perturbation}
\label{sec:perturbation}
Dans notre cas, nous nous sommes intéressés à l'impact du vent sur les architectures mentionnées précédemment. Il est donc nécessaire de modéliser le vent. Dans le cas de vent constant ou de cisaillement de vent, un modèle à échelons semble tout à fait approprié pour représenter le changement brusque de vitesse de vent. 
Pour les rafales, nous utiliserons plusieurs modèles, tel que le modèle "Chapeau mexicain" ou les "ondelettes de Morlet".

La fonction définissant le chapeau mexicain est :
\begin{align}
    \Psi_{mex}(t)= w_{m} - \frac{A_g}{2} \left(1-\cos(2 \pi f_g (t-t_{0,mex}))\right)\sin(3 \pi f_g (t-t_{0,mex}))
\end{align}
et la fonction ondelettes de Morlet est définie par :
\begin{align}
    \Psi_{mor}(t)=  w_{m} + A_g \exp(-(t+t_{0,mor})) \cos(5*(t-t_{0,mor}))
\end{align}
où $w_{m}$ est le vent moyen sans perturbation, $t_{0,mex}$ représente l'instant de début de perturbation et $t_{0,mor}$ l'instant où la perturbation est maximale, $A_g$ est l'intensité maximale de la perturbation et  $f_g$ est la fréquence de la perturbation. Les tracés de la figure \ref{fig:mexhat} montrent la représentation temporelle des deux fonctions pour les valeurs $w_{m} = \SI{1}{\meter\per\second}$, $t_{0,mex} = \SI{2}{\second}$,  $t_{0,mor} = \SI{5}{\second}$, $A_g = \SI{1}{\meter\per\second}$ et $f_g = \SI{0.8}{\radian\per\second}$.


\begin{figure}[ht!]
    \centerline{
    \includegraphics[trim=0cm 0cm 0cm 0cm,clip,width=0.9\columnwidth]{figures/mex_hat_morlet.png}}
    \caption{Représentation temporelle des modèles de perturbation de vent.}
    \label{fig:mexhat}
\end{figure}


% Le modèle de Dryden est défini :
% \begin{subequations}
%     \begin{align}
%         \Phi _{u_{g}}(\Omega )& =\sigma _{u}^{2}{\frac {2L_{u}}{\pi }}{\frac {1}{1+(L_{u}\Omega )^{2}}}\\
%         \Phi _{v_{g}}(\Omega )& =\sigma _{v}^{2}{\frac {2L_{v}}{\pi }}{\frac {1+12(L_{v}\Omega )^{2}}{\left(1+4(L_{v}\Omega )^{2}\right)^{2}}}\\
%         \Phi _{w_{g}}(\Omega )& =\sigma _{w}^{2}{\frac {2L_{w}}{\pi }}{\frac {1+12(L_{w}\Omega )^{2}}{\left(1+4(L_{w}\Omega )^{2}\right)^{2}}}
%     \end{align}
% \end{subequations}
% Le modèle de von Kármán :
% \begin{subequations}
%     \begin{align}
%         \Phi _{u_{g}}(\Omega )& =\sigma _{u}^{2}{\frac {2L_{u}}{\pi }}{\frac {1}{\left(1+(1.339L_{u}\Omega )^{2}\right)^{\frac {5}{6}}}}\\
%         \Phi _{v_{g}}(\Omega )& =\sigma _{v}^{2}{\frac {2L_{v}}{\pi }}{\frac {1+{\frac {8}{3}}(2.678L_{v}\Omega )^{2}}{\left(1+(2.678L_{v}\Omega )^{2}\right)^{\frac {11}{6}}}}\\
%         \Phi _{w_{g}}(\Omega )& =\sigma _{w}^{2}{\frac {2L_{w}}{\pi }}{\frac {1+{\frac {8}{3}}(2.678L_{w}\Omega )^{2}}{\left(1+(2.678L_{w}\Omega )^{2}\right)^{\frac {11}{6}}}}
%     \end{align}
% \end{subequations}



\section{Actionnements}

La figure \ref{fig:actionDarko} illustre les effets des actionneurs sur l'attitude d'un \textit{tailsitter}.
\begin{figure}[ht!]
    \centerline{
    \includegraphics[trim=0cm 0cm 0cm 0cm,clip,width=1\columnwidth]{figures/actionnement.png}}
    \caption{Actionneurs d'un \textit{tailsitter} et leurs actions sur les axes principaux.}
    \label{fig:actionDarko}
\end{figure}

La rotation autour de l'axe $x_{\text{b}}$ peut être contrôlée par des braquages de volets asymétriques, l'angle de tangage (axe $y_{\text{b}}$) par des braquages de volets symétriques et la rotation autour de l'axe $z_{\text{b}}$ est contrôlée par un dispositif de poussée différentielle, ce qui permet d'obtenir un angle de tangage plus élevé. La rotation autour de l'axe de lacet est contrôlée par une configuration de poussée différentielle, qui créé un couple temporaire non nul autour de son axe. Le dispositif de poussée différentielle engendre une différence entre les rotations de l'hélice gauche et de l'hélice droite, et modifie ainsi la vitesse de rotation de l'avion et donc le comportement aérodynamique autour des volets.


\section{Architectures de commande de vol}


Une approche en cascade est une méthode couramment utilisée dans le cadre de la commande de drone. Cette approche est basée sur l'imbrication de boucles de  commande, telle qu'illustrée sur la Figure \ref{fig:schemahiera}. 
Cette architecture s'appuie sur la séparation des dynamiques d'un drone sous actionné. Effectivement, le drone étant obligé de modifier son orientation pour changer de position, le schéma de contrôle prend en compte cette spécificité. Ainsi, la boucle interne est réglée pour avoir un temps de réponse faible vis-à-vis des autres lois. Cette boucle interne stabilise l'attitude du drone ($att_{m}$) en la faisant converger vers l'attitude de consigne ($att_{c}$). Cette attitude de consigne est le résultat de la boucle de vitesse faisant converger la vitesse du drone ($vel_{m}$) vers la vitesse de consigne ($vel_{c}$). La vitesse de consigne est la résultante de la boucle de position faisant converger la position du drone ($pos_{m}$) vers la position de consigne définie par le pilote ($pos_{c}$).
\begin{figure}[ht!]
    \centerline{
    \includegraphics[trim=0cm 0cm 0cm 0cm,clip,width=0.8\columnwidth]{figures/controlhierachique.png}}
    \caption{Architecture classique de contrôle hiérarchique pour les drones.}
    \label{fig:schemahiera}
\end{figure}


\section{Méthodes de commande}
Plusieurs méthodes de commande ont été proposées sur les drones convertibles. Nous pouvons notamment distinguer les architectures linéaires et non-linéaires. La figure~\ref{fig:methodecmd} propose une sélection des principales lois disponibles dans la littérature.

\begin{figure}[ht!]
    \centerline{
    \includegraphics[trim=0cm 0cm 0cm 0cm,clip,width=0.8\columnwidth]{figures/controle_methode.png}}
    \caption{Méthodes de commande utilisées sur les architectures \textit{tailsitters} et \textit{freewings}.}
    \label{fig:methodecmd}
\end{figure}


\subsection*{Contrôleur PID}
Le contrôle Proportionnel Intégral Dérivé (PID) (ou P/PD/PI) est couramment utilisé, car son réglage est intuitif et ne nécessite qu'une connaissance limitée du système. Le contrôle PID donne souvent de très bons résultats et constitue un excellent point de départ pour la conception de contrôleurs plus avancés.

La forme générale d'un contrôleur PID est la suivante :
\begin{align}
    u_{PID}(t) = K_{p} e(t) + K_{i} \int_{0}^{t} e(\tau) \,d\tau + K_{d} \frac{d e(t)}{dt}
\end{align}
{\color{red} où le vecteur d'erreur est $e(t) = x_{ref} (t) - x(t)$ et où les gains $K_{p}$, $K_{i}$ et $K_{d}$ sont respectivement le gain proportionnel, intégral et dérivatif. Ils peuvent être réglés individuellement pour obtenir le comportement souhaité du système en boucle fermée.}

\subsection*{Contrôleur LQR}
Les contrôleurs LQR sont un type de contrôle linéaire basé sur les systèmes linéarisés de la forme $\dot{x} = Ax+Bu$. La commande $u$ est
choisie comme :
\begin{align*}
    u = -Kx.
\end{align*}

où la matrice de gain $K$ est optimisée pour obtenir de bonnes performances pour le système en boucle fermée par rapport à la fonction de coût suivant :
\begin{align*}
    J = \int_{0}^{\infty} x^{\top}(\tau)Q x(\tau) + u^{\top}(\tau)R u(\tau) \,d\tau
\end{align*}

où Q et R sont, respectivement, les matrices de pondération de l'état et des commandes. 

Bien que le contrôleur LQR présente généralement de bonnes propriétés de robustesse, l'optimalité n'est plus assurée si des erreurs de modélisation et des perturbations sont présentes dans le système. Une discussion détaillée et une comparaison entre le contrôle PID sans modèle et le contrôle LQR basé sur le modèle d'un \textit{tailsitter} est disponible dans \cite{BarthCondomines2018}.
Les auteurs de \cite{Lustosa2017LaP} ont proposé l'approche de séquencement des gains obtenus pour un ensemble de modèles linéaires. Cette architecture de contrôle optimise le gain K en boucle fermée afin de répondre aux exigences de contrôle de la vitesse et de l'attitude définies par l'utilisateur. Grâce à des simulations de vol et à des vols expérimentaux, les auteurs soulignent et prouvent qu'une seule matrice de gains LQR n'est pas suffisante pour stabiliser le \textit{tailsitter} dans toute son enveloppe de vol, ce qui justifie l'utilisation de méthodes de séquencement des gains.

\subsection*{Contrôleur $H_{\infty}$}
La synthèse $H_{\infty}$ est une méthode qui sert à la conception de commandes optimales, en imposant des contraintes sur la norme $H_{\infty}$ d'un système. En se basant sur une synthèse $H_{\infty}$, les auteurs de \cite{SunYang2009} ont obtenu la stabilisation longitudinale d'un \textit{tiltrotor}. En combinant une synthèse $H_{\infty}$ à une approche de séquencement des gains, les auteurs de \cite{DickesonMix2005, DickesonCifdaloz2006,DickesonMiles2007} ont proposé la stabilisation d'un \textit{tiltwing} sur l'ensemble de son enveloppe de vol. La structure classique d'un contrôleur $H_{\infty}$ est représentée sur la figure \ref{fig:schemalft}. 

\begin{figure}[ht!]
    \centerline{
    \includegraphics[trim=0cm 0cm 0cm 0cm,clip,width=0.5\columnwidth]{figures/lft.png}}
    \caption{Architecture d'un contrôleur $H_{\infty}$.}
    \label{fig:schemalft}
\end{figure}

L'objectif est de minimiser l'effet de la perturbation $w$  (cas le moins favorable possible) sur la variable de performance $z$, par le biais d'un contrôleur par retour de sortie de la forme $u = K(s) y$. En d'autres termes, K est choisi de telle sorte que $\| F(P, K)\|_{\infty}$  soit minimisée, où $F(P, K)$ est la fonction de transfert de la perturbation $w$ vers le signal d'erreur $z$. 

Nous pouvons aussi citer le travail de \cite{SNYDER2021106621} qui utilise une approche de contrôle $H_{\infty}$ ainsi qu'une représentation du système \textit{Linear Parameter Varying} (LPV). Cette représentation permet d'avoir la dynamique linéaire du drone sur l'ensemble du domaine de vol par interpolation de dynamique parametrée par la vitesse totale du drone.
\nomenclature[]{\(LPV\)}{Système linéaire à paramètres variant (\textit{Linear Parameter Varying})}


Enfin, il existe une extension non-linéaire de l'approche $H_{\infty}$ en utilisant l'espace de Sobolev pondéré $W_{\infty}$ \cite{cardoso2018nonlinear, CardosoEsteban2019, cardoso2021robust, cardoso2024robust}. Cette méthode s'appuie sur la norme de Sobolev $W_{m,p}$  d'un signal, laquelle est définie par :
\begin{align*}
    \|\boldsymbol{z(t)}\|_{W_{m,p}} = \left( \int_{t_{0}}^{\infty}(\|\boldsymbol{z(t)}\|^{p} + \|\frac{d \boldsymbol{z(t)}}{dt}\|^{p} + ... + \|\frac{d^{m} \boldsymbol{z(t)}}{dt^{m}}\|^{p}) dt \right)^{\frac{1}{p}}.
\end{align*}  

Grâce à la nature de la norme $W_{m,p}$, la variable de coût et ses dérivées temporelles sont prises en compte dans la fonction de coût, ce qui permet d'obtenir des contrôleurs offrant de meilleures performances en régime transitoire et en régime permanent.

\subsection*{Contrôleur non linéaire}
{ \color{red}
L'approche "inversion dynamique non linéaire incrémentale" (INDI) utilise la connaissance de la dynamique d'un système. Une inversion incrémentale est réalisé pour d'obtenir les accélérations angulaires cibles. Théoriquement, le bouclage des accélérations angulaires élimine la sensibilité aux imperfections du modèle, ce qui augmente considérablement la robustesse du système par rapport à l'inversion dynamique non linéaire conventionnelle \cite{Sieberling2010, Binz2019}. Ce contrôleur nécessite l'identification des actionneurs afin de régler les paramètres du contrôleur, tels que l'efficacité des actionneurs. Étant donné que l'efficacité des surfaces de contrôle aérodynamique n'est pas constante pendant toute l'enveloppe de vol sur un \textit{tiltwing}, une méthode de programmation des gains a été mise au point \cite{smeurINDI,smeurINDITail}.
La limitation d'une telle loi de commande est la disponibilité et la qualité  des accélérations angulaires, en effet, les capteurs mesurant la vitesse angulaire sont très bruité et nous sommes obligé de dériver ce signal pour obtenir les accélérations angulaires.

}

\subsection*{Contrôleur sans modèle basée capteur}
{\color{red}
Contrairement à l'INDI basée sur l'inversion d'un modèle, la loi de commande sans modèle (MFC) \cite{Fliess_2013}  nécessite peu de connaissances préalables du système contrôlé pour concevoir le contrôleur. Cette commande est basée sur un estimateur non-linéaire qui estime la dynamique du système. Soulignons que cet estimateur ne modifie pas les gains de la commande en retour, contrairement aux méthodes de commande adaptative. Le processus adaptatif MFC est réalisé par l'estimateur qui fournir une sortie, laquelle est utilisé pour calculer la commande formant la boucle fermée. Avec cet estimateur, il est possible de mettre en œuvre un contrôleur robuste et adaptatif qui assure la stabilité des systèmes variables dans le temps en estimant en temps réel leur dynamique à partir des mesures périodiques. Nous pouvons citer le travail de \cite{olszaneckibarthHal-02542982} sur un \textit{tailsitter} qui continuer à voler malgré une modification importante de son actionnement et de sa dynamique (voir figure \ref{fig:darko_mfc}).

\begin{figure}[ht!]
    \centerline{
    \includegraphics[trim=0cm 0cm 0cm 0cm,clip,width=0.5\columnwidth]{figures/darko_MFC.png}}
    \caption{\textit{Tailsitter} DarkO lors de test de robustesse \cite{olszaneckibarthHal-02542982}.}
    \label{fig:darko_mfc}
\end{figure}


\nomenclature[]{\(MFC\)}{Commande sans modèle (\textit{ Model Free Control})}
}
\subsection*{Contrôleur séquencé}
L'approche "Diviser et Conquérir" permet de passer d'une loi de contrôle à une autre, de manière discrète, de sorte qu'un seul contrôleur ne fonctionne à la fois. Un exemple d'un contrôle séquencé hybride de \textit{tailsitters} est donnée dans \cite{Casau2011}.

Contrairement à "Diviser et Conquérir", la pondération de contrôle fusionne en continu les commandes de deux contrôleurs, sur la base d'un poids dépendant d'une variable de sélection. Les auteurs de \cite{Liang2016} ont développé deux contrôleurs : un pour le mode VTOL et un second pour la croisière, lesquels fonctionnent simultanément. La commande globale est calculée comme  :
\begin{align*}
    u(\Delta t) = (1 - \frac{\Delta t}{T_{T}})u_{VTOL}(\Delta t) + \frac{\Delta t}{T_{T}}u_{cruise}(\Delta t).
\end{align*}
Cet exemple est une transition entre le mode vol stationnaire et le mode croisière. Dans ce cas, $T_{T}$ est le paramètre de durée de la transition et $\Delta t$ est le temps écoulé depuis le début de la manœuvre.

\subsection*{Contrôleur adaptatif}
{\color{red}
\nomenclature[]{\(MRAC\)}{Commande adaptative à référence de modèle (\textit{ Model Reference Adaptive Control})}
La commande adaptative à référence de modèle (MRAC) utilise un modèle de référence avec les performances de suivi souhaitées et adapte les paramètres du contrôleur en fonction de la différence entre le modèle réel $y$ et le modèle de référence $y_{m}$ (voir figure \ref{fig:schemaMRAC}). 

\begin{figure}[ht!]
    \centerline{
    \includegraphics[trim=0cm 0cm 0cm 0cm,clip,width=0.5\columnwidth]{figures/Mrac.png}}
    \caption{Architecture d'un contrôleur MRAC.}
    \label{fig:schemaMRAC}
\end{figure}


Le mécanisme d'ajustement peut être développé à partir de la règle de MIT \ref{Jain2013DesignOA}, de la théorie de Lyapunov \ref{GE1999741, CHAKRABARTY2016213} et d'autres approches mathématiques, telles que les fonctions d'ajustement de la covariance, etc. 

La technique de mécanisme d'ajustement proposée par \ref{CHAKRABARTY2016213}, basée sur les fonctions de Lyapunov, vise à suivre la sortie du système et ses états avec le modèle de référence. Pour ce faire, on suppose que tous les états sont disponibles pour la mesure, ce qui n'est pas toujours le cas dans les applications pratiques de contrôle.

Selon la règle du MIT \ref{Shekhar2018}, le mécanisme d'ajustement fonctionne en minimiser l'erreur entre la sortie du système contrôlé $y$ et la sortie du modèle de référence $y_{m}$ via une fonction de coût qui met à jour le gain du contrôleur. Ces paramètres sont modifiés dans le sens négatif du gradient de la fonction de coût. Notons que le modèle de référence joue un rôle important dans les performances du contrôleur car la règle MIT tente de faire correspondre la sortie du système contrôlé avec la sortie du modèle de référence. Si ce modèle est mal modélisé, les performances de contrôle seront donc directement impacté.
En outre, la règle MIT est très sensible aux changements d'amplitude de l'entrée de référence.

Pour l'estimation des paramètres en temps réel, les méthodes les plus couramment utilisées sont les dérivés de la méthode des moindres carrés moyens, tels que la méthode des moindres carrés moyens normalisés, la méthode des moindres carrés moyens récursifs et le filtre de Kalman étendu. 
}
\section{Technologies et réalisations}

Nos expérimentations sont réalisées grâce au système de drone Paparazzi (voir Annexe \ref{chap:annexe1}), lequel est constitué d'une partie matérielle et logicielle. La partie matérielle est principalement composée de l'autopilote (Apogee ou Tawaki) et la partie logicielle comporte l'ensemble des codes, embarqués et au sol, permettant de faire voler un drone. 

La figure \ref{fig:schemaComposent} présente de manière schématique les principaux composants nécessaires à la conduite d'un vol, tous étant reliés à l'autopilote. L'autopilote se charge d'exécuter périodiquement les codes associés aux différentes fonctions du drone (estimation d'état, navigation, guidage, contrôle (figure~\ref{fig:schemahiera}), charge utile, etc.).

\begin{figure}[ht!]
    \centerline{
    \includegraphics[trim=0cm 0cm 0cm 0cm,clip,width=0.7\columnwidth]{figures/arch_materiel.png}}
    \caption{Architecture d'un autopilote et son environnement.}
    \label{fig:schemaComposent}
\end{figure}

Nos principales contributions dans cet environnement sont le codage des lois de commande proposées dans les chapitres \ref{chap:hybrid} et \ref{chap:6DOF} et le codage de l'estimation d'état du chapitre \ref{chap:colibri}. 

Un travail supplémentaire a été réalisé sur les contrôleurs moteurs. Les ESC ont été flashés avec le code open-source AM32 (voir Annexe \ref{sec:AM32}). L'avantage de ce \textit{firmware}, par rapport au code commercial, est qu'il exploite un retour PID \nomenclature[]{\(PID\)}{Proportionnel Intégral Dérivé} de bas niveau de la vitesse de rotation du moteur, calculé à la même vitesse que la commutation de phase du moteur. Nous avons adapté le code de la boucle de vitesse dans le \textit{firmware}, en suivant l'approche de \cite{franchi2017}, avec un algorithme de biais et de gain adaptatif (ABAG) \nomenclature[]{\(ABAG\)}{Biais adaptatif et gain adaptatif \textit{Adaptive Bias and Adaptive Gain}}. Les avantages de la solution ABAG sont une grande réactivité et une grande adaptabilité, puisque les dimensions de l'hélice peuvent être modifiées sans qu'il ne soit nécessaire de modifier les gains d'actionnement.
Un module a aussi été développé pour pouvoir utiliser des servomoteurs Feetech (le modèle STS3032), qui ont la possibilité d'être chaînés et d'être commandés via une liaison série bidirectionnelle. Cela permet notamment d'avoir un retour sur la position du servomoteur.

\begin{figure}[ht!]
    \centering
    \resizebox{.9\textwidth}{!}{%
    \includegraphics[height=3cm]{figures/aikon-ak32-esc.jpg}
    \quad
    \includegraphics[height=3cm]{figures/feetech-sts3032.jpg}
    }
    \caption{Contrôleur moteur et servomoteur.}
    \label{fig:ESCServo}
\end{figure}




\section{Conclusion}
{\color{red}
Ce chapitre présente un résumé de la littérature relative à la modélisation et au contrôle des \textit{tailsitters}.

Le domaine des drones convertibles est vaste et chaque architecture cherche à répondre à une mission. Les diverses complexités engendrées sont autant de challenges pour assurer la stabilité du drone. Les défis aérodynamiques liés aux différents domaines de vol, tels que les écoulements à faible nombre de Reynolds, l'interaction hélice-ailes et l'identification des coefficients aérodynamiques par des campagnes en soufflerie, ont été résumés. Plusieurs publications expliquent la complexité de développer un modèle dynamique pour les \textit{tailsitters}. Cette difficulté notamment liée à la dynamique instable dans les vols de transition induite par des changements rapides des forces et des moments aérodynamiques, appelle au développement de stratégies de contrôle innovantes et robuste. 

L'analyse de la littérature sur les lois de commande indique que la mise en œuvre d'une loi robuste abordant directement l'impact du vent sur le drone n'a jamais été réalisé.

Nous allons donc nous focaliser sur un drone \textit{tailsitter} et plus précisément sa modélisation dans le but de comprendre les effets des perturbations sur sa dynamique.




}