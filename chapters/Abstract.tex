\pagestyle{empty} % remove headers on 2e page
\newgeometry{left=1.5in,right=1.3in,top=1.1in,bottom=1.1in} % remove header/footer space

% French
\begin{vcenterpage} % note: vcenterpage is not needed for long abstrat
\noindent\rule[2pt]{\textwidth}{0.5pt}

{\large\textbf{Résumé :}}
Les drones sont aujourd'hui devenus un outil dans de nombreux domaines tels que l'inspection, la surveillance ou la maintenance. Cependant, ils souffrent d'une autonomie limitée. Les \textit{tailsitters} apportent une solution grâce à leur grande enveloppe de vol et à leur efficacité énergétique. Toutefois, les \textit{tailsitters} sont grandement sujets aux perturbations aérologiques et notamment aux turbulences dans les phases stationnaires principalement. Cela est dû à la grande surface d'aile verticale, laquelle possède une grande prise au vent. De plus, leur corps tournant lors de la transition, il est donc compliqué de mesurer la vitesse de l'air.  Ainsi, en stationnaire ou à faible vitesse, le vent n'est pas connu. Ce type de drone est sous-actionné puisque l'on trouve deux moteurs sur l'aile et deux surfaces aérodynamiques sur le bord de fuite. Le flux d'air des hélices soufflant les élevons, nous avons un couplage entre les actionneurs.

Cette thèse cherche à étudier la commande de drones dans des environnements perturbés ou en présence de vent. Les premiers travaux se sont concentrés sur la dynamique sans vent pour appréhender une dynamique simplifiée. Nous avons pu proposer une modification non-linéaire du vecteur de commande pour rendre ce modèle linéaire en commande. De ce modèle, nous avons proposé une loi de commande locale-globale fondée sur une dynamique hybride à hystérésis. Elle permet d'étendre le domaine de stabilité de la loi de commande linéaire agressive à l'aide d'une loi non-linéaire avec une grande région d'attraction, mais moins agressive.

La suite des travaux s'est concentrée sur la stabilisation d'un \textit{tailsitters} soumis à des échelons de vent. Il en résulte une caractérisation des équilibres stationnaires pour un ensemble de conditions de vent et l'obtention de la représentation linéarisée de la dynamique du drone. À l'aide de ce modèle, il a été possible d'analyser les saturations des actionneurs et l'autorité disponible aux environs des points d'équilibre. Nous avons réalisé une stabilisation établie sur un retour de sortie, avec une action proportionnelle et intégrale. Cette commande n'utilise pas la mesure de l'angle de tangage du drone, car nous ne pouvons pas, a priori, connaître la valeur cible qui nécessiterait une estimation de la vitesse et de la direction du vent. L'optimisation de ce bouclage est effectuée à l'aide du logiciel "Systune" pour obtenir de bonnes propriétés de réjection de perturbation. Une approche incrémentale a été suivie, la loi de commande ayant été testée dans un premier temps sur une maquette à un degré de liberté face à une soufflerie ouverte. Une fois validée, la loi de commande a été implémentée dans le système de drone Paparazzi. Grâce à son architecture modulaire, il a été possible de nous interfacer avec les codes d'estimation et de commande des actionneurs. Ainsi, nous avons pu réaliser des vols sur le modèle complet à six degrés de liberté.

Enfin, nous avons proposé une architecture inspirée du \textit{tailsitter}, nommée \textit{freewing}. Nous avons développé un drone multicorps basé sur une aile en rotation libre sur son axe de tangage autour d'un fuselage. L'actionnement de l'aile est sensiblement le même que pour le \textit{tailsitters} et le fuselage possède deux actionneurs pour se maintenir horizontal. Nous recherchons, dans cette architecture, une passivité naturelle à la turbulence induite par le changement naturel de l'incidence de l'aile en fonction du vent incident. Il s'agit aussi d'installer une charge utile sur le fuselage horizontal sur le domaine de vol. De plus, nous avons réalisé un modèle de simulation où la dynamique est obtenue à l'aide des équations de Udwadia-Kalaba et de la phi-théorie. Enfin, nous nous sommes concentrés sur la stabilisation et le guidage du drone en utilisant une inversion incrémentale non-linéaire de la dynamique (INDI). Nous utilisons les actionneurs de l'aile et du fuselage pour obtenir une loi de stabilisation globale. Des vols ont validé l'intérêt de cette architecture.

{\large\textbf{Mots clés :}}
mots, clefs
\todo{Mots clés}

\noindent\rule[2pt]{\textwidth}{0.5pt}
\end{vcenterpage}

% English
\newpage
\begin{vcenterpage}
\noindent\rule[2pt]{\textwidth}{0.5pt}
% LTeX: language=en
{\large\textbf{Abstract:}}
Drones have become a prevalent tool in numerous fields, including inspection, surveillance, and maintenance. However, one area where they are currently lacking is autonomy. Tailsitter offer a viable solution, combining large flight envelopes with energy efficiency. Nonetheless, tailsitter unmanned aerial vehicles are particularly susceptible to disturbances, specifically turbulence during hover phases. This is because of the large vertical surface area of the wing, which offers a high degree of wind resistance. Furthermore, the drone's body rotates during transition, making it difficult to accurately measure airspeed. It is not possible to accurately measure wind speed, whether the drone is hovering or moving at a low speed. Additionally, this underactuated drone, has two motors on the wing and two aerodynamic surfaces on the trailing edge. The interaction of the airflow from the propellers with the elevons results in a coupling between the actuators.

The aim of this thesis is to examine the control of drone in challenging environments, with a particular emphasis on the impact of wind. However, initial research focused on windless dynamics in order to gain a better understanding of the simplified dynamics. We were able to propose a non-linear modification of the control vector to transform this nonlinear model into an input-affine model. Based on this model, we proposed a local-global control law based on hybrid dynamics with hysteresis, which allowed us to extend the stability domain of an aggressive linear control law by means of a non-linear law with a large region of attraction, but which was less aggressive. 

Further work was conducted on the stabilization of a tailsitter drone under wind steps. This resulted in a characterization of stationary equilibria for a range of wind conditions and a linearized representation of the drone's dynamics. Using this model, we were able to analyze the saturation levels of actuators and the authority available close to the equilibrium points. We have implemented a stabilization system based on output feedback with proportional and integral action, derived from the model. This control does not utilize the drones pitch angle measurement, since the target value is not known a priori. It would require an estimation of wind speed and direction, which is not feasible. The loop is optimized using "Systune" software in order to achieve effective disturbance rejection properties. We adopted an incremental approach by initially evaluating the control law on a one-degree-of-freedom model against an open wind tunnel. Upon validation, we proceeded to implement the control law within the Paparazzi UAV system. Due to its modular design, we were able to establish a connection with the state estimation and actuators, enabling us to execute flights on the entire model with six degrees of freedom. 

We then proposed an architecture inspired by the tailsitter, called freewing. We have developed a multi-body drone that is based on a wing that can freely rotate on its pitch axis around a fuselage. The wing is driven the same way as the tailsitter, and the fuselage has a motor and an aerodynamic surface to keep it horizontal. With this architecture, we aim to achieve a natural passivity to turbulence induced by the natural change in wing incidence as a function of the incident wind. However, there is also the possibility of installing a payload on the horizontal fuselage. To obtain a simulation model, we have modeled the drone's dynamics using the Udwadia-Kalaba equations and the phi-theory. We focused on stabilization and guidance of the UAV through the use of incremental nonlinear dynamic inversion (INDI). The wing and fuselage actuators are used to achieve a global stabilization law. In order to evaluate the benefits of this architecture, we conducted flight tests.

{\large\textbf{Keywords:}}
key, words
\todo{Keywords}

\noindent\rule[2pt]{\textwidth}{0.5pt}
\end{vcenterpage}
